\documentclass[french,a4paper,10pt]{article}
\makeatletter % Allows the use of @ in command names

% Font and Language Settings ----------------------------------------------------
\usepackage[T1]{fontenc} % Use T1 font encoding for better character representation
\usepackage[french]{babel} % Set document language to French
\usepackage{lmodern} % Use Latin Modern font

% List Customization ------------------------------------------------------------
\usepackage[shortlabels]{enumitem} % Enhanced control over lists
\setlist[itemize,1]{label={\color{gray}\small \textbullet}} % Customize the bullet style for itemize

% Header and Footer Customization -----------------------------------------------
\usepackage{fancyhdr} % Create custom headers and footers

% Mathematical Symbols and Enhancements -----------------------------------------
\usepackage{centernot} % Provides the \centernot command for centered negation
\usepackage{stmaryrd} % Provides extra mathematical symbols like \llbracket
\usepackage[overload]{abraces} % Provides extensible braces
\usepackage{latexsym} % Standard LaTeX symbols
\usepackage{amsmath} % American Mathematical Society mathematical features
\usepackage{amsfonts} % AMS font package
\usepackage{amssymb} % Additional AMS symbols
\usepackage{amsthm} % Theorem environments
\usepackage{mathtools} % Mathematical tools to extend amsmath
\usepackage{mathrsfs} % Provides \mathscr for script letters
\usepackage{MnSymbol} % More math symbols
\usepackage{etoolbox} % Conditional macros
\usepackage{hyperref} % Hyperlinks within the document

% Graphing Tools ----------------------------------------------------------------
\usepackage{tikz} % Create graphics programmatically
\usepackage{pgfplots} % Create plots using TikZ
\pgfplotsset{compat=1.18} % Set compatibility level for pgfplots
\usetikzlibrary{arrows} % Add arrow tips to TikZ

% Color Customization -----------------------------------------------------------
\usepackage{xcolor} % Color management
\usepackage{colortbl} % Color management for tables
\definecolor{astral}{RGB}{46,116,181} % Define astral color
\definecolor{verdant}{RGB}{96,172,128} % Define verdant color
\definecolor{algebraic-amber}{RGB}{255,179,102} % Define algebraic-amber color
\definecolor{calculus-coral}{RGB}{255,191,191} % Define calculus-coral color
\definecolor{divergent-denim}{RGB}{130,172,211} % Define divergent-denim color
\definecolor{matrix-mist}{RGB}{204,204,204} % Define matrix-mist color
\definecolor{numeric-navy}{RGB}{204,204,204} % Define numeric-navy color
\definecolor{quadratic-quartz}{RGB}{204,153,153} % Define quadratic-quartz color

% Custom Theorem Styles ---------------------------------------------------------
\usepackage[]{mdframed} % Framing for environments


% Redeclare math operators for customization ------------------------------------
\makeatletter
\newcommand\RedeclareMathOperator{%
	\@ifstar{\def\rmo@s{m}\rmo@redeclare}{\def\rmo@s{o}\rmo@redeclare}%
}
\newcommand\rmo@redeclare[2]{%
	\begingroup \escapechar\m@ne\xdef\@gtempa{{\string#1}}\endgroup
	\expandafter\@ifundefined\@gtempa
	{\@latex@error{\noexpand#1undefined}\@ehc}%
	\relax
	\expandafter\rmo@declmathop\rmo@s{#1}{#2}}

\newcommand\rmo@declmathop[3]{%
	\DeclareRobustCommand{#2}{\qopname\newmcodes@#1{#3}}%
}
\@onlypreamble\RedeclareMathOperator
\makeatother


% Miscellaneous Commands --------------------------------------------------------
\setlength{\parindent}{0pt} % Remove paragraph indentation

% Define emphasis in text
\providecommand{\defemph}[1]{{\sffamily\bfseries\color{astral}#1}}

% Section styling
\usepackage{sectsty} % Allows customizing section fonts
\allsectionsfont{\color{astral}\normalfont\sffamily\bfseries} % Style all section headers

\newcommand{\skipline}{\vspace{\baselineskip}} % Skip a line
\newcommand{\noi}{\noindent} % No indentation

\makeatother % End of @ usage
\makeatletter
% Custom Commands for fast mathematical notation --------------------------------
% General shortcuts
\newcommand{\scr}[1]{\mathscr{#1}} % Script font shortcut
\newcommand{\bb}[1]{\mathbb{#1}} % Blackboard bold shortcut
\newcommand{\ol}[1]{\overline{#1}} % Overline shortcut
\newcommand{\ul}[1]{\underline{#1}} % Underline shortcut
\newcommand{\act}{\circlearrowleft} % Action symbol

% Interval notation
\newcommand{\oo}[1]{\mathopen{}\left]#1\right[\mathclose{}} % Open interval % chktex 9
\newcommand{\of}[1]{\mathopen{}\left]#1\right]\mathclose{}} % Half-open interval (open first) % chktex 9 % chktex 10
\newcommand{\fo}[1]{\mathopen{}\left[#1\right[\mathclose{}} % Half-open interval (open last)
\newcommand{\ff}[1]{\mathopen{}\left[#1\right]\mathclose{}} % Closed interval % chktex 9

% Set notation
\newcommand{\R}{\mathbb{R}} % Real numbers
\newcommand{\Z}{\mathbb{Z}} % Integers
\newcommand{\N}{\mathbb{N}} % Natural numbers
\newcommand{\C}{\mathbb{C}} % Complex numbers
\newcommand{\Q}{\mathbb{Q}} % Rational numbers
\newcommand{\bbR}{\mathbb{R}} % Real numbers
\newcommand{\bbZ}{\mathbb{Z}} % Integers
\newcommand{\bbN}{\mathbb{N}} % Natural numbers
\newcommand{\bbC}{\mathbb{C}} % Complex numbers
\newcommand{\bbQ}{\mathbb{Q}} % Complex numbers
\newcommand{\scrP}{\mathscr{P}} % Set of subsets
\newcommand{\barR}{\overline{\bb{R}}} % R with infinities
\newcommand{\gln}{\text{GL}_n} % General linear group of degree n
\newcommand{\glx}[1]{\text{GL}_{#1}} % General linear group
%
\newcommand{\sub}{\subset} % Subset
\newcommand{\cequiv}[1]{\mathopen{}[#1\mathclose{}]} % Equivalence class
\newcommand{\restr}[2]{#1\mathop{}\!|_{#2}} % Restriction
\newcommand{\adh}[1]{\mathring{#1}} % Adherence
\newcommand{\Adh}[1]{\mathring{\overbrace{#1}}} % Big adherence
\newcommand{\comp}[1]{{#1}^C} % Complementary of a set

% Differential notation
\newcommand{\der}{\mathop{}\!{d}} % Differential operator
\newcommand{\p}{\mathop{}\!{\partial}} % Partial derivative operator
\providecommand{\dpar}[2]{\frac{\partial{#1}}{\partial{#2}}}

% Topology notation
\newcommand{\bolo}[1]{B\left({#1}\mathopen{}\right[\mathclose{}} % Open ball
\newcommand{\bolf}[1]{B\left({#1}\mathopen{}\right]\mathclose{}} % Closed ball

% Limits notation
\newcommand{\limi}{\underline{\lim}}
\newcommand{\lims}{\overline{\lim}}

% Norm notation
\newcommand{\norm}{\mathcal{N}} % Norm
\newcommand{\nn}[1]{\mathopen{}\left\|#1\right\|\mathclose{}} % Double bar norm
\newcommand{\nnn}[1]{\mathopen{}\left\||#1|\right\|\mathclose{}} % Double bar norm
\newcommand{\n}[1]{\mathopen{}\left|#1\right|\mathclose{}} % Single bar norm


% Other operators
\providecommand{\1}{\mathds{1}} % Identity operator
\DeclareMathOperator{\im}{\mathsf{Im}} % Imaginary part
\DeclareRobustCommand{\re}{\mathsf{Re}} % Real part
\RedeclareMathOperator{\ker}{\mathsf{Ker}} % Kernel
\RedeclareMathOperator{\det}{\mathsf{det}} % Determinant
\DeclareMathOperator{\vect}{\mathsf{Vect}} % Vector space
\DeclareMathOperator{\diam}{\mathsf{Diam}} % Diameter
\DeclareMathOperator{\orb}{\mathsf{orb}} % Orbit
\DeclareMathOperator{\st}{\mathsf{st}} % Standard part
\DeclareMathOperator{\spr}{\mathsf{SP_{\bb{R}}}} % Real spectrum
\DeclareMathOperator{\aut}{\mathsf{Aut}} % Automorphism group
\DeclareMathOperator{\bij}{\mathsf{Bij}} % Bijection group
\DeclareMathOperator{\rank}{\mathsf{rank}} % Rank
\DeclareMathOperator{\tr}{\mathsf{tr}} % Trace
\DeclareMathOperator{\id}{\mathsf{Id}} % Identity
\DeclareMathOperator{\var}{\mathsf{Var}} % Variance
\DeclareMathOperator{\cov}{\mathsf{Cov}} % Covariance
\providecommand{\B}{\mathsf{B}} % Bold symbol

\usepackage{dsfont}
\newcommand{\one}{\mathds{1}}

\makeatother
\input{src/common/macros/theorems.tex}

\mytheorem{lemma}{Lemme}{astral}{section}{o}
\mytheorem{theorem}{Théorème}{astral}{section}{o}
\usepackage[a4paper,hmargin=30mm,vmargin=30mm]{geometry}

\title{\color{astral} \sffamily \bfseries Étude de \( \langle (i~j),(1~\ldots~n) \rangle \)}
\author{Ivan Lejeune}
\date{\today}


\begin{document}
    \maketitle

    \section*{Introduction}

    Dans ce document, nous allons étudier le sous-groupe de \( \mathfrak S_n \) engendré par
    \begin{equation*}
        \sigma = \langle (i~j),(1~\ldots~n) \rangle
    \end{equation*}
    pour \( 1 \leq i < j \leq n \). On travaillera aussi ``modulo \(n\)'' 
    où les indices iront de \(1\) à \(n\) et où \(n+1 = 1\).

    \subsection*{Quelques résultats préliminaires}

    \begin{lemma}
        Soit \( \sigma \in \mathfrak S_n \) et \( i_1~\ldots~i_k \) une permutation de \( 1, \ldots, k \). Alors
        \begin{equation*}
            \sigma (i_1~\ldots~i_k) \sigma^{-1} = (\sigma(i_1)~\ldots~\sigma(i_k)).
        \end{equation*}
    \end{lemma}

    \begin{lemma}
        Soit \( \sigma \in \mathfrak S_n \) et \( i~j \) une transposition. Alors
        \begin{equation*}
            \langle \sigma, (i~j) \rangle = \langle \sigma, (1~k) \rangle
        \end{equation*}
        où \( k = j - i + 1 \).
    \end{lemma}

    \begin{lemma}
        Soit \( \sigma = (1~\ldots~n) \). Alors
        \begin{equation*}
            \sigma (1~k) \sigma^{-1} = (2~k+1)
        \end{equation*}
        pour \( k \in \bb N \).
    \end{lemma}

    \section*{Le choix de \(i\) et \(j\)}

    \subsection*{Le cas \(j = i+1\)}

    Comme on l'a déjà vu en TD, si \(i = 1, j = 2\), alors
    \begin{equation*}
        \langle (1~2), (1~\ldots~n) \rangle = \mathfrak S_n.
    \end{equation*}

    Si on ne fixe pas \(i\) et on prend \(j = i+1\), on peut alors se ramener à ce cas en utilisant le lemme 2.

    Ainsi, un des premiers résultats que l'on peut obtenir est le suivant.

    \begin{theorem}
        Soit \( \sigma = \langle (i~i+1), (1~\ldots~n) \rangle \) pour \( 1 \leq i < n \). Alors
        \begin{equation*}
            \sigma = \mathfrak S_n.
        \end{equation*}
    \end{theorem}

    \begin{proof}
        On a
        \begin{align*}
            \sigma &= \langle (i~i+1), (1~\ldots~n) \rangle \\
            &= \langle (1~2), (1~\ldots~n) \rangle \\
            &= \mathfrak S_n.
        \end{align*}
    \end{proof}

    \subsection*{Le cas \(j \neq i+1\)}

    C'est ici que les choses se compliquent. On va essayer de trouver des conditions sur \(i\) et \(j\) pour que
    \begin{equation*}
        \langle (i~j), (1~\ldots~n) \rangle = \mathfrak S_n.
    \end{equation*}

    La première chose qu'on peut faire est de se ramener à un cas plus simple en utilisant le lemme 2.

    On étudie alors
    \begin{equation*}
        \sigma = \langle (1~k), (1~\ldots~n) \rangle
    \end{equation*}

    \(\triangleright\) Le premier point important à aborder est que
    pour engendrer l'ensemble des transpositions adjacentes (qui engendrent \(\mathfrak S_n\)), il suffit
    d'en engendrer une seule. En effet, on a
    \begin{equation*}
        (i+1~i+2) = (1~\ldots~n) (i~i+1) {(1~\ldots~n)}^{-1},\quad 1 \leq i < n.
    \end{equation*}
    De plus, on a un ``décalage de l'écart entre les indices'' qu'on peut effectuer. En effet, 
    à partir de \( (i~j) \) et \( (j~k) \), on peut obtenir \( (i~k) \) en utilisant
    \begin{equation*}
        (i~j) (j~k) (i~j) = (i~k).
    \end{equation*}

    Essayons maintenant de voir comment on peut obtenir une telle transposition à partir de \( (1~k) \).

    \begin{lemma}
        Soit \( \sigma = \langle (1~k+1), (1~\ldots~n) \rangle \). Alors
        \begin{equation*}
            k\wedge n = 1 \implies \sigma = \mathfrak S_n.
        \end{equation*}
    \end{lemma}

    \begin{proof}
        On peut énumérer certains éléments de \( \sigma \):
        \begin{equation*}
            \begin{aligned}
                (1~k+1) &\to (2~k+2) \\
                (2~k+2) &\to (3~k+3) \\
                &\vdots \\
                &\to (k+1~2k+1) \\
                &\vdots \\
                &\to (2k+1~3k+1) \\
                &\vdots
            \end{aligned}
        \end{equation*}
        On voit alors qu'à partir de \( (1~k+1) \), on peut obtenir
        \begin{equation*}
            \left\{ (1~\lambda k+1) \mid \lambda \in \bb N \right\}.
        \end{equation*}

        D'autre part, si \( k\wedge n = 1 \), alors il existe \( a, b \in \bb Z \) tels que
        \begin{equation*}
            ak + bn = 1.
        \end{equation*}
        En particulier, pour \( \lambda = a \), on a
        \begin{equation*}
            \begin{aligned}
                (1~\lambda k+1) 
                &= (1~ak+1) \\
                &= (1~1-bn +1) \\
                &= (1~2).
            \end{aligned}
        \end{equation*}
        On a donc \( (1~2) \in \sigma \) et on peut obtenir toutes les transpositions adjacentes.
    \end{proof}

    On a donc une condition suffisante pour que \( \langle (i~j), (1~\ldots~n) \rangle = \mathfrak S_n \).
    
    On peut facilement vérifier que si \( k\wedge n \neq 1 \), alors on
    ne peut pas obtenir de transposition adjacente à partir de seulement \( (1~k+1) \) et \( (1~\ldots~n) \).

    Cependant, il est difficile de prouver que c'est une condition nécessaire.

    \section*{Conclusion}

    On a vu que si \( j = i+1 \), alors \( \langle (i~j), (1~\ldots~n) \rangle = \mathfrak S_n \).

    Si \( j \neq i+1 \), on a vu qu'il est possible d'obtenir \( \mathfrak S_n \) si \( k\wedge n = 1 \).

    Cependant, on n'a pas réussi à déterminer ce qu'il se passe si \( k\wedge n \neq 1 \).

    Ainsi, on peut dire que

    \begin{theorem}
        Soit \( \sigma = \langle (i~j), (1~\ldots~n) \rangle \) pour \( 1 \leq i < j \leq n \). Alors
        \begin{equation*}
            \sigma = \mathfrak S_n
        \end{equation*}
        si \( j = i+1 \) ou \( (j-i+1)\wedge n = 1 \).
    \end{theorem}

    \section*{Remarque}

    Dans certains cas, on peut vérifier ce qui se passe pour \( k\wedge n \neq 1 \) 
    mais cela est difficile pour des valeurs de \( n \) élevées.

    \subsection*{Exemple}

    On peut vérifier que pour \( n = 4\) et \(k = 2\), on a
    \begin{equation*}
        \begin{aligned}
            \langle (1~3), (1~\ldots~4) \rangle
            = \{&(1~3), (2~4),\\
            &(1~2)(3~4), (1~3)(2~4), (1~4)(2~3),\\
            &(1~2~3~4), (1~4~3~2)
            \}\neq \mathfrak S_4.
        \end{aligned}
    \end{equation*}
    Cela ne prouve pas que si \( k\wedge n \neq 1 \), alors \( \langle (1~k+1), (1~\ldots~n) \rangle \neq \mathfrak S_n \)
    mais le montre pour \( n = 4 \).
\end{document}
