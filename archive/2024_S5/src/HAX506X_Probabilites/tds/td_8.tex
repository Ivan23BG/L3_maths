\begin{td-exo}[]\, % 1
    Soit \({(X_n)}_{n\geq 1}\) une suite de variables aléatoires
    de loi donnée par
    \begin{equation*}
        \bb P(X_n=n)=\bb P(X_n=-n)=\frac{1}{2n}\quad\text{et}\quad \bb P(X_n=0)=1-\frac{1}{2n}
    \end{equation*}
    \begin{enumerate}
        \item Montrer que la suite \({(X_n)}_{n\geq 1}\) 
        converge en probabilité vers 0. Y a-t-il convergence
        dans \(\scr L^p\)?

        \item On suppose maintenant que les variables aléatoires
        \({(X_n)}_{n\geq 1}\) sont indépendantes. On considère
        l'événement \(A_n=\{X_n=n\}\). Montrer que
        \begin{equation*}
            \bb P\left(\limsup_{n\to\infty} A_n\right) = 1
        \end{equation*}
        puis en déduire que
        \begin{equation*}
            \bb P\left(\lim_{n\to\infty} X_n = +\infty\right) = 1
        \end{equation*}
    \end{enumerate}
\end{td-exo}

\iftoggle{showsolutions}{
    \begin{td-sol}[]\, % 1
        \begin{enumerate}
            \item Soit \(\varepsilon>0\) et \(n\geq\varepsilon\). Alors
            \begin{equation*}
                \bb P(|X_n|\geq\varepsilon) = \bb P(X_n=n)+\bb P(X_n=-n)=\frac{1}{n}\cvn 0
            \end{equation*}
            donc
            \begin{equation*}
                X_n\cvpn 0
            \end{equation*}
            mais
            \begin{equation*}
                \begin{aligned}
                    \bb E(|X_n|^p) 
                    &= |n|^p\frac{1}{2n}+|-n|^p\frac{1}{2n} + 0^p\left(1-\frac{1}{n}\right) \\
                    &= n^{p-1}\not\cvn 0\quad\text{pour }p\geq 1
                \end{aligned}
            \end{equation*}
            donc \({(X_n)}_{n\geq 1}\) ne converge pas dans \(\scr L^p\).

            \item Les événements \({(A_n)}_{n\geq 1}\) sont indépendants et
            \begin{equation*}
                \bb P(A_n) = \bb P(X_n=n) = \frac{1}{2n}
            \end{equation*}
            qui est le terme général d'une série divergente. Donc
            Borel-Cantelli assure que
            \begin{equation*}
                \bb P\left(\limsup_{n\to\infty} A_n\right) = 1
            \end{equation*}

            De plus,
            \begin{equation*}
                \begin{aligned}
                    1
                    &= \bb P\left(\bigcap_{n=1}^\infty\bigcup_{k=n}^\infty A_k\right) \\
                    &= \bb P\left(\forall n\geq 1,\exists k\geq n,X_k=k\right)\\
                    &\leq \bb P\left(\forall n\geq 1,\sup_{k\geq n}X_k\geq n\right)\\
                    &\leq \bb P\left(\left(\limsup_{n\to\infty}X_n\right)=+\infty\right)
                \end{aligned}
            \end{equation*}
            donc
            \begin{equation*}
                \left(\lim_{n\to\infty}X_n\right)=+\infty \text{ p.s.}
            \end{equation*}
            et \({(X_n)}_{n\geq 1}\) ne converge pas p.s..
        \end{enumerate}
    \end{td-sol}
}{}

\begin{td-exo}[]\, % 2
    \begin{enumerate}
        \item Soit \({(X_n)}_{n\geq 1}\) une suite de variables aléatoires
        indépendantes. Montrer que \((X_n)\) converge vers 0 presque
        sûrement si et seulement si pour tout \(\varepsilon>0\),
        la série
        \begin{equation*}
            \sum_{n=1}^\infty \bb P(|X_n|>\varepsilon)
        \end{equation*}
        est convergente.

        \item On suppose maintenant que \(X_n\) suit une loi
        de Bernoulli de paramètre \(p_n\).
        \begin{enumerate}
            \item Montrer que la suite \((X_n)\) converge
            vers 0 en probabilité si et seulement si
            \(p_n\to 0\) quand \(n\to\infty\).

            \item Montrer que la suite \((X_n)\) converge
            vers 0 dans \(\scr L^p\) si et seulement si
            \(p_n\to 0\) quand \(n\to\infty\).

            \item On suppose que les \(X_n\) sont indépendantes.
            Montrer que la suite \((X_n)\) converge vers 0
            presque sûrement si et seulement si la série
            \(\sum_{n=1}^\infty p_n\) converge.
        \end{enumerate}

        \item Soit \({(X_n)}_{n\geq 1}\) une suite de variables aléatoires
        indépendantes telles que \(X_n \sim \mathcal E(n)\). Montrer
        que la suite \((X_n)\) converge vers 0 presque sûrement.
        Y a-t-il convergence dans \(\scr L^p\)?
    \end{enumerate}
\end{td-exo}

\iftoggle{showsolutions}{
    \begin{td-sol}[]\, % 2
        \begin{enumerate}
            \item Montrons que
            \begin{equation*}
                X_n\cvpsn 0 \iff \forall \varepsilon>0,\quad \sum_{n=1}^\infty \bb P(|X_n|>\varepsilon)<\infty
            \end{equation*}
            \begin{itemize}[\ptr{}]
                \item Sens direct:\\
                On rappelle que \(X_n\cvpsn 0\) si et seulement si
                \begin{equation*}
                    \bb P\left(\forall\varepsilon>0,\exists k\geq 1,\forall n\geq k,|X_n|<\varepsilon\right) = 1.
                \end{equation*}
                On pose
                \begin{equation*}
                    B_\varepsilon = \bigcup_{k\geq 1}\bigcap_{n\geq k}\{|X_n|<\varepsilon\}
                \end{equation*}
                où
                \begin{equation*}
                    \bb P\left(\bigcap_{\varepsilon>0}B_\varepsilon\right) = 1.
                \end{equation*}
                Soit \(\varepsilon>0\). Alors
                \begin{equation*}
                    \bigcap_{\alpha>0}B_\alpha \subset B_\varepsilon
                \end{equation*}
                donc
                \begin{equation*}
                    \bb P(B_\varepsilon) \geq \bb P\left(\bigcap_{\alpha>0}B_\alpha\right) = 1
                \end{equation*}
                Ainsi,
                \begin{equation*}
                    \bb P(B_\varepsilon) = \bb P\left(\liminf_{n\to\infty}\{|X_n|<\varepsilon\}\right) = 1
                \end{equation*}
                d'où
                \begin{equation*}
                    \bb P\left(\limsup_{n\to\infty}\{|X_n|\geq\varepsilon\}\right) = 0.
                \end{equation*}
                Par Borel-Cantelli, on a donc
                \begin{equation*}
                    \sum_{n=1}^\infty \bb P(|X_n|\geq\varepsilon)<\infty
                \end{equation*}

                \item Sens réciproque:\\
                Par Borel-Cantelli, on a
                \begin{equation*}
                    \begin{aligned}
                        & \forall\varepsilon>0,\bb P\left(\limsup_{n\to\infty}\{|X_n|\geq\varepsilon\}\right) = 0\\
                        & \forall\varepsilon>0,\bb P\left(\bigcup_{k\geq 1}\bigcap_{n\geq k}\{|X_n|<\varepsilon\}\right) = 1
                    \end{aligned}
                \end{equation*}
                \begin{remark}
                    Si \(\bb P(A_n)=0\) pour tout \(n\geq 1\), alors
                    \begin{equation*}
                        \bb P\left(\bigcup_{n=1}^\infty A_n\right) \leq \sum_{n=1}^\infty \bb P(A_n)=0
                    \end{equation*}
                    De même, si \(\bb P(A_n)=1\) pour tout \(n\geq 1\), alors
                    \begin{equation*}
                        \bb P\left(\bigcap_{n=0}^\infty A_n\right)=1
                    \end{equation*}
                \end{remark}
                Ainsi, on a
                \begin{equation*}
                    \bb P\left(\bigcap_{\substack{\varepsilon>0\\\varepsilon\in\bb Q}}\bigcup_{k\geq 1}\bigcap_{n\geq k}\{|X_n|<\varepsilon\}\right)=1
                \end{equation*}
                c'est-à-dire
                \begin{equation*}
                    \bb P(\forall\varepsilon\in\bb Q_+^*,\exists k\geq 1,\forall n\geq k,|X_n|<\varepsilon)=1
                \end{equation*}
                et alors
                \begin{equation*}
                    \bb P\left(\lim_{n\to\infty}X_n=0\right)=1
                \end{equation*}
                donc \(X_n\cvpsn 0\).
            \end{itemize}

            \item On a \(X_n\sim\scr B(p_n)\).
            \begin{enumerate}
                \item On montre dans les deux sens
                \begin{itemize}[\ptr{}]
                    \item Soit \(\varepsilon>0\), alors
                    \begin{equation*}
                        \bb P\left(|X|\geq \varepsilon\right)\leq \bb P(X_n=1)=p_n
                    \end{equation*}
                    dinc si \(p_n\to 0\), alors \(X_n\cvpn0\).

                    \item Réciproquement, si \(X\cvpn0\), alors
                    \begin{equation*}
                        \bb P\left(|X_n|\geq 1\right)\cvn 0\quad (\varepsilon=1)
                    \end{equation*}
                    et donc \(p_n\cvn 0\).
                \end{itemize}

                \item On a
                \begin{equation*}
                    \begin{aligned}
                        \bb E \ff{{\n{X_n}}^p}
                        &= 0^p\times (1-p_n) + 1^p \times p_n\\
                        &= p_n
                    \end{aligned}
                \end{equation*}
                Donc \(X_n \cvlpn 0\) si et seulement si \(p_n\cv0\)

                \item On a
                \begin{equation*}
                    \bb P\left(\n{X_n}\geq\varepsilon\right)=
                    \begin{cases}
                        0&\text{ si }\varepsilon>1\\
                        1&\text{ sinon}
                    \end{cases}
                \end{equation*}
                Donc d'après la question 2, on a \(X_n\cvpsn0\) si
                et seulement si \(\sum_{n=1}^{\infty}p_n<\infty\)
            \end{enumerate}

            \item On rappelle que \(X_n\cvpn0\) par le cours.\\
            Soit \(\varepsilon>0\), on a
            \begin{equation*}
                \begin{aligned}
                    \bb P\left(\n{X_n}\geq\varepsilon\right)
                    &=\bb P\left(X_n\geq\varepsilon\right)\\
                    &=\int_{\varepsilon}^\infty ne^{-nt}\der t\\
                    &=\ff{-e^{-nt}}_{\varepsilon}^{\infty}\\
                    &= e^{-\varepsilon}
                \end{aligned}
            \end{equation*}
            Comme \(\sum_{n=1}^\infty e^{-n\varepsilon}<\infty\), on en déduit que 
            \(X_n\cvpsn 0\).
        \end{enumerate}
    \end{td-sol}
}{}