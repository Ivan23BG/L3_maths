\begin{td-exo}[] % 1
    On considère l'espace probabilisé 
    \((\ff{0,1},\mathcal B(\ff{0,1}),\bb P)\) où
    \(\bb P\) est la mesure de Lebesgue et on pose
    \(A_n = \of{0,\frac1n}\) pour tout entier \(n\geq 1\).

    \begin{enumerate}
        \item Expliciter l'événement
        \begin{equation*}
            \limsup_{n\to\infty} A_n.
        \end{equation*}

        \item Calculer
        \begin{equation*}
            \sum_{n=1}^\infty \bb P(A_n).
        \end{equation*}
        Commenter.
    \end{enumerate}
\end{td-exo}

\iftoggle{showsolutions}{
    \begin{td-sol}[]\,
        \begin{enumerate}
            \item On a
            \begin{equation*}
                \begin{aligned}
                    \limsup_{n\to\infty} A_n
                    &=\bigcap_{n\geq 1}\bigcup_{k\geq n} A_k\\
                    &= \bigcap_{n\geq 1}\bigcup{k\geq n}\of{0,\frac1k}\\
                    &=\bigcap_{n\geq 1} \of{0,frac1n}\\
                    &=\varnothing
                \end{aligned}
            \end{equation*}

            \item On a
            \begin{equation*}
                \sum_{n=1}^\infty \bb P(A_n) = \sum_{n=1}^\infty\frac1n=\infty
            \end{equation*}
            et
            \begin{equation*}
                \bb P\left(\limsup_{n\to\infty} A_n\right) = \bb P(\varnothing) = 0
            \end{equation*}

            Les \(A_n\) ne sont pas indépendants:
            \begin{equation*}
                \begin{aligned}
                    &\bb P(A_2\cap A_3) = \bb P(A_3) = \frac13\\
                    &\bb P(A_2)\bb P(A_3) = \frac12\frac13=\frac16
                \end{aligned}
            \end{equation*}
        \end{enumerate}
    \end{td-sol}
}{}

\begin{td-exo}[]\, % 2
    On lance une infinite de fois une pièce de monnaie équilibrée et
    on considère un entier \(k\in\bb N^\ast\). Montrer qu'avec probabilité
    1, on obtiendra une infinité de fois \(k\) Pile consécutifs.
\end{td-exo}

\iftoggle{showsolutions}{
    \begin{td-sol}[]\, % 2
        \ptr{} On considère les événements suivants:
        \begin{itemize}
            \item \(A_1=\) \og{} on n'obtient que des Pile entre \(1\) et \(k\)\fg{},
            \item \(A_2=\) \og{} on n'obtient que des Pile entre \(k+1\) et \(2k\)\fg{},
            \item \(A_3=\) \og{} on n'obtient que des Pile entre \(2k+1\) et \(3k\)\fg{},
            \item \(A_n=\) \og{} on n'obtient que des Pile entre \((n-1)k+1\) et \(nk\)\fg{}.
        \end{itemize}

        Les événements \({(A_n)}_{n\geq 1}\) sont indépendants et
        \begin{equation*}
            \sum_{n=1}^\infty \bb P(A_n)=\sum_{n=1}^\infty\frac{1}{2^k}=\infty
        \end{equation*}
        Donc Borel-Cantelli assure que
        \begin{equation*}
            \bb P\left(\limsup_{n\to\infty} A_n\right) = 1
        \end{equation*}

    \end{td-sol}
}{}

\begin{td-exo}[]\, % 3
    Soit \({(X_n)}_{n\geq 1}\) une suite de variables aléatoires réelles
    indépendantes telles que
    \begin{equation*}
        \forall n\geq 1,\quad \bb P(X_n=1)=p=1-\bb P(X_n=-1),\quad p\in\oo{0,1}
    \end{equation*}
    On pose \(S_0=0\) et \(S_n = X_1+\cdots+X_n\). Enfin, on considère l'événement
    \(A_n = \{S_n = 0\}\).
    \begin{enumerate}
        \item Déterminer la probabilité de \(A_n\). On pourra distinguer
        suivant la parité de \(n\).

        \item A l'aide de la formule de Stirling, déterminer un équivalent
        de \(\bb P(A_{2n})\).

        \item On pose \(A=\limsup_{n\to\infty} A_n\). Que représente cet
        événement? Déterminer \(\bb P(A)\) dans le cas \(p\neq\frac12\).

        \item On suppose maintenant que \(p=\frac12\). On va montrer que
        \(\bb P(A)=1\).
        \begin{enumerate}
            \item Expliquer pourquoi le lemme de Borel-Cantelli ne
            s'applique pas.

            \item Montrer que pour tout \(n\geq 1\) et pour tout \(k\geq 1\),
            les vecteurs aléatoires \((X_{k+1},\ldots,X_{k+n})\) et
            \(X_1,\ldots,X_n\) ont la même loi.

            \item On rappelle que \(A=\limsup_{n\to\infty}A_n\). Montrer que
            \begin{equation*}
                A^c = \bigcup_{k=0}^\infty\left\{ S_k=0\text{ et }\forall n\geq 1,S_{n+k}\neq 0\right\}
            \end{equation*}
            et que cette union est formée d'événements disjoints.

            \item En déduire que \(\bb P(A)=1\).
        \end{enumerate}
    \end{enumerate}
\end{td-exo}

\iftoggle{showsolutions}{
    \begin{td-sol}[]\, % x
        \begin{enumerate}
            \item Si \(n\) est impair, alors \(\bb P(A_n) = 0\).\\
            Si \(n=2k\) est pair, alors
            \begin{equation*}
                \begin{aligned}
                    \bb P(A_{2k}) 
                    &= \bb P(S_{2k}=0)\\
                    &= \underbrace{\binom{2k}{k}}_{\substack{\text{nb de}\\\text{chemins}}} \underbrace{p^k}_{\text{ch }\nearrow}\underbrace{{(1-p)}^{n-k}}_{\text{ch }\searrow}
                \end{aligned}
            \end{equation*}

            \item On veut montrer que
            \begin{equation*}
                n!\sim{n\to\infty} \frac{n^n}{e^n}\sqrt{2\pi n}
            \end{equation*}
            Alors
            \begin{equation*}
                \begin{aligned}
                    \bb P(A_{2n})
                    &= \frac{(2n)!}{{(n!)}^2}p^n{(1-p)}^n\\
                    &\underset{n\to\infty}{\sim}\frac{{(2n)}^{2n}}{e^{2n}}\sqrt{4\pi n}\frac{{(e^n)}^2}{{(n^n)}^2 {\sqrt{2\pi n}}^2}p^n{(1-p)}^n\\
                    &\underset{n\to\infty}{\sim} \frac{4^n p^n{(1-p)}^n}{\sqrt{\pi n}}\\
                    &\underset{n\to\infty}{\sim}\frac{{(4p(1-p))}^n}{\sqrt{\pi n}}
                \end{aligned}
            \end{equation*}

            \item On considère 
            \begin{equation*}
                A = \limsup_{n\to\infty} A_n = \text{ on repasse une infinité de fois par }0
            \end{equation*}
            alors, comme \(p\neq\frac12\), la série des \(\bb P(A_n)\) converge
            par le critère de D'Alembert et donc
            \begin{equation*}
                \bb P(\limsup_{n\to\infty} A_n) = 0.
            \end{equation*}

            Si \(p=\frac12\), alors
            \begin{enumerate}
                \item on a
                \begin{equation*}
                    \sum_{n=1}^\infty \bb P(A_n)=+\infty
                \end{equation*}
                mais les \(A_n\) ne sont pas indépendants donc Borel-Cantelli 
                ne s'applique pas.

                \item On a
                \begin{equation*}
                    \begin{aligned}
                        &~\bb P\left((X_1,\ldots,X_n)\in B_1\times\cdots B_n\right)\\
                        &=\bb P\left(X_1\in B_1,\ldots,X_n\in B_n\right)\\
                        &=\bb P(X_1\in B_1)\cdots \bb P(X_n\in B_n)
                    \end{aligned}
                \end{equation*}
                et
                \begin{equation*}
                    \begin{aligned}
                        &~\bb P\left((X_{k+1},\ldots,X_{k+n})\in B_1\times\cdots B_n\right)\\
                        \smol{indep}&=\bb P(X_{k+1}\in B_1)\cdots \bb P(X_{k+n}\in B_n)\\
                        \smol{mm loi}&=\bb P(X_1\in B_1)\cdots \bb P(X_n\in B_n)
                    \end{aligned}
                \end{equation*}
                Donc \((X_1,\ldots,X_n)\) et \((X_{k+1},\ldots,X_{k+n})\) ont
                la même loi

                \item On a
                \begin{equation*}
                    \begin{aligned}
                        A^c
                        &= {\left(\bigcap_{n\geq 0}\bigcup_{k\geq n}A_k\right)}^c\\
                        &= \bigcup_{n\geq 0}\bigcap{k\geq n}{A_k}^c\\
                        &=\bigcup_{n\geq 0}\bigcap_{k=0}^\infty {A_{k+n}}^c\\
                        &= \bigcup_{n\geq 0}\bigcap_{k=0}^\infty \left\{S_{n+k}\neq 0\right\}\\
                        &\iff \substack{\text{il existe un pas de temps à partir}\\\text{duquel on ne repasse plus par }0}\\
                        &=\bigcup_{n=0}^\infty \color{red}\underbrace{\color{black}\left\{S_n=0,\forall k\geq 1,S_{k+n}\neq 0\right\}}_{{B_n}}\color{black}
                    \end{aligned}
                \end{equation*}
                Alors
                \begin{equation*}
                    B_n= \left\{ X_1+\cdots+X_n = 0\text{ et }\forall k\geq 1,X_{n+1}+\cdots X_{n+k}\neq 0\right\}
                \end{equation*}
                Si \(n'>n\), alors dans \(B_n\) on a \(S_{n'}\neq 0\) (car \(n'>n\)) et
                dans \(B'_n\) on a \(S_{n'}=0\).

                \item On a
                \begin{equation*}
                    \begin{aligned}
                        \bb P(A^c) 
                        &= \sum_{n=0}^\infty \bb P(S_n=0,\forall k\geq 1, X_{n+1}+\cdots X_{n+k}\neq 0)\\
                        \smol{indep des \(1\ldots n\),\(n+1\ldots n+k\)}&= \sum_{n=0}^\infty\bb P(S_n = 0)\bb P(\forall k \geq 1, X_{n+1}+\cdots+ X_{n+k}\neq 0)\\
                        &= \sum_{n=0}^\infty \bb P(S_n = 0)\bb P(\forall k\geq 1,X_1+\cdots X_k \neq 0)
                    \end{aligned}
                \end{equation*}
                Bref, on a montré que
                \begin{equation*}
                    \bb P(A^c) = \bb P(\forall k\geq 1,S_k\neq 0)\underbrace{\sum_{n=0}^\infty \underbrace{\bb P(S_n = 0)}_{\bb P(A_n)}}_{=\infty}
                \end{equation*}
                Nécessairement, comme \(\bb P(A^c)\in\ff{0,1}\), on obtient
                \(\bb P(A^c)=0\) et donc \(\bb P(A)=1\).

            \end{enumerate}
        \end{enumerate}
    \end{td-sol}
}{}