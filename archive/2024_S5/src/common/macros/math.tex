\makeatletter
% Custom Commands for fast mathematical notation --------------------------------
% General shortcuts
\newcommand{\scr}[1]{\mathscr{#1}} % Script font shortcut
\newcommand{\bb}[1]{\mathbb{#1}} % Blackboard bold shortcut
\newcommand{\ol}[1]{\overline{#1}} % Overline shortcut
% \newcommand{\ul}[1]{\underline{#1}} % Underline shortcut

% Interval notation
\newcommand{\oo}[1]{\mathopen{}\left]#1\right[\mathclose{}} % Open interval % chktex 9
\newcommand{\of}[1]{\mathopen{}\left]#1\right]\mathclose{}} % Half-open interval (open first) % chktex 9 % chktex 10
\newcommand{\fo}[1]{\mathopen{}\left[#1\right[\mathclose{}} % Half-open interval (open last)
\newcommand{\ff}[1]{\mathopen{}\left[#1\right]\mathclose{}} % Closed interval % chktex 9

% Set notation
\newcommand{\R}{\mathbb{R}} % Real numbers
\newcommand{\Z}{\mathbb{Z}} % Integers
\newcommand{\N}{\mathbb{N}} % Natural numbers
\newcommand{\C}{\mathbb{C}} % Complex numbers
\newcommand{\Q}{\mathbb{Q}} % Rational numbers
\newcommand{\barR}{\overline{\bb{R}}} % R with infinities
% \newcommand{\gln}{\mathsf{GL}_n} % General linear group
\newcommand{\glx}[1]{\mathsf{GL}_{#1}} % General linear group
% \newcommand{\set}[1]{\left\{#1\right\}} % Set
\newcommand{\sub}{\subset} % Subset
% \newcommand{\cequiv}[1]{\mathopen{}[#1\mathclose{}]} % Equivalence class
\newcommand{\restr}[2]{#1\mathop{}\!|_{#2}} % Restriction
% \newcommand{\adh}[1]{\mathring{#1}} % Adherence
% \newcommand{\Adh}[1]{\mathring{\overbrace{#1}}} % Big adherence
\newcommand{\comp}[1]{{#1}^{\mathsf{c}}} % Complement

% Differential notation
\newcommand{\der}{\mathop{}\!{d}} % Differential operator
\newcommand{\p}{\mathop{}\!{\partial}} % Partial derivative operator
%\providecommand{\dpar}[2]{\frac{\partial{#1}}{\partial{#2}}}

% Topology notation
\newcommand{\bolo}[1]{\mathsf{B}\left({#1}\mathopen{}\right[\mathclose{}} % Open ball
\newcommand{\bolf}[1]{\mathsf{B}\left({#1}\mathopen{}\right]\mathclose{}} % Closed ball

% Limits notation
\newcommand{\limi}{\underline{\lim}}
\newcommand{\lims}{\overline{\lim}}

% Norm notation
\newcommand{\norm}{\mathcal{N}} % Norm
\newcommand{\nn}[1]{\mathopen{}\left\|#1\right\|\mathclose{}} % Double bar norm
\newcommand{\n}[1]{\mathopen{}\left|#1\right|\mathclose{}} % Single bar norm


% Other operators
\providecommand{\1}{\mathds{1}} % Identity operator
\DeclareMathOperator{\im}{\mathsf{Im}} % Imaginary part
\DeclareRobustCommand{\re}{\mathsf{Re}} % Real part
\RedeclareMathOperator{\ker}{\mathsf{Ker}} % Kernel
\RedeclareMathOperator{\det}{\mathsf{det}} % Determinant
\DeclareMathOperator{\vect}{\mathsf{Vect}} % Vector space
\DeclareMathOperator{\diam}{\mathsf{Diam}} % Diameter
\DeclareMathOperator{\orb}{\mathsf{orb}} % Orbit
\DeclareMathOperator{\st}{\mathsf{st}} % Standard part
\DeclareMathOperator{\spr}{\mathsf{SP_{\bb{R}}}} % Real spectrum
\DeclareMathOperator{\aut}{\mathsf{Aut}} % Automorphism group
\DeclareMathOperator{\bij}{\mathsf{Bij}} % Bijection group
\DeclareMathOperator{\rank}{\mathsf{rank}} % Rank
\DeclareMathOperator{\tr}{\mathsf{tr}} % Trace
\DeclareMathOperator{\id}{\mathsf{Id}} % Identity
\DeclareMathOperator{\var}{\mathsf{Var}} % Variance
\DeclareMathOperator{\cov}{\mathsf{Cov}} % Covariance
\providecommand{\B}{\mathsf{B}} % Bold symbol

\newcommand{\one}{\mathds{1}}

\newcommand{\smol}[1]{\text{\scriptsize{#1}}}

\makeatother