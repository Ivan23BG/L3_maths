\subsection*{Equations différentielles et Cauchy-Lipschitz}

\begin{td-exo}[] % 1
	La vitesse de déplacement des ions entre deux électrodes immergées dans un électrolyte
	vérifie l'équation différentielle
	\begin{equation*}
		\frac{\der v}{\der t}+\frac Rmv=\frac Fm
	\end{equation*}

	où \(m,F,R\) sont des constantes. Calculer \(v\).
\end{td-exo}

\iftoggle{showsolutions}{
		\begin{td-sol}[] % 1
			On reconnait une équation différentielle linéaire d'ordre 1 de la forme
			\begin{equation*}
				y'(x)+a(x)y(x)=b(x)
			\end{equation*}
			avec
			\begin{equation*}
				y(x)=v(t),\quad a(x)=\frac Rm,\quad b(x) = \frac Fm,\quad x=t
			\end{equation*}
			Pour résoudre une équation de cette forme, on résout l'équation homogène puis on
			cherche une solution particulière.
			
			\(\triangleright\) Résolvons l'équation homogène
			\begin{equation*}
				\frac{\der v}{\der t}+\frac Rmv=0
			\end{equation*}

			\begin{itemize}
				\item Une primitive de la fonction \(\frac Rm\) est
				\begin{equation*}
					\frac Rm t
				\end{equation*}

				\item Les solutions de l'équation homogène sont donc
				\begin{equation*}
					v_h = \lambda e^{-{\frac Rm t}}
				\end{equation*}
				avec \(\lambda\in\R\).
			\end{itemize}

			\(\triangleright\) Trouvons une solution particulière à l'équation
			\begin{equation*}
				\frac{\der v}{\der t}+\frac Rmv=\frac Fm
			\end{equation*}
			\begin{itemize}
				\item Ici le second membre est une constante donc on cherche une solution
				évidente sous forme de constante.

				On voit que \(v_p=\frac FR\) convient:
				\begin{equation*}
					\begin{aligned}
						\frac{\der v}{\der t}+\frac Rmv&=\frac Fm\\
						0+\frac Rm\times \frac FR &= \frac Fm\\
						\frac Fm &= \frac Fm
					\end{aligned}
				\end{equation*}
			\end{itemize}

			\(\triangleright\) Les solutions de l'équation s'écrivent comme somme de la 
			solution particulière et des solutions de l'équation homogène, soit ici:
			\begin{equation*}
				v(t)= v_h(t) + v_p(t) = \lambda e^{-{\frac Rm t}} + \frac FR
			\end{equation*}
		\end{td-sol}
}{}

\begin{td-exo}[] % 2
	Résoudre les problèmes de Cauchy suivants:
	\begin{enumerate}
		\item \(x'-2x=e^{2t}t^2\) avec \(x(0)=0\),
		\item \(x' -\frac1{1+t}x=2t^2\) avec \(x(0)=-3\),
		\item \(x'-(1+t)x=-2t-t^2\) avec \(x(0)=2\).
	\end{enumerate}

\end{td-exo}

\iftoggle{showsolutions}{
	\begin{td-sol}[]\, % 1
		\begin{enumerate}
			\item \(\triangleright\) Résolvons l'équation homogène
			\begin{equation*}
				x'-2x=e^{2t}t^2
			\end{equation*}

			\begin{itemize}
				\item Une primitive de la fonction \(-2\) est
				\begin{equation*}
					-2t
				\end{equation*}

				\item Les solutions de l'équation homogène sont donc
				\begin{equation*}
					x_h = \lambda e^{2t}
				\end{equation*}
				avec \(\lambda\in\R\).
			\end{itemize}

			\(\triangleright\) Trouvons une solution particulière à l'équation
			\begin{equation*}
				x'-2x=e^{2t}t^2
			\end{equation*}
			\begin{itemize}
				\item Ici le second membre est de la forme polynome fois exponentielle
				donc on cherche une solution sous la forme
				\begin{equation*}
					x_p = t\left(at^2+bt+c\right) e^{2t}
				\end{equation*}
				Ce facteur \(t\) est nécessaire car on a déjà une solution homogène de la
				forme \(e^{2t}\) et on ne veut pas de redondance.
				On intègre dans l'équation pour avoir
				\begin{equation*}
					\begin{aligned}
						&x'-2x=e^{2t}t^2\\
						\implies &\left(3at^2+2bt+c\right)e^{2t}+2\left(at^3+bt^2+ct\right)e^{2t} - 2\left(at^3+bt^2+ct\right)e^{2t}=e^{2t}t^2\\
						\implies &\left(2at^3 + (3a+2b)t^2 + (2b+c)t + 2c\right)e^{2t}=e^{2t}t^2
					\end{aligned}
				\end{equation*}
				On a donc
				\begin{equation*}
					\begin{aligned}
						a&=0\\
						b&=\frac12\\
						c&=0
					\end{aligned}
				\end{equation*}
				Donc la solution particulière est
				\begin{equation*}
					x_p = \frac12t^2e^{2t}
				\end{equation*}
			\end{itemize}

			\(\triangleright\) Les solutions de l'équation s'écrivent comme somme de la
			solution particulière et des solutions de l'équation homogène, soit ici:
			\begin{equation*}
				x(t)= x_h(t) + x_p(t) = \lambda e^{2t} + \frac12t^2e^{2t}
			\end{equation*}

			\item 2e partie
		\end{enumerate}
	\end{td-sol}
}{}

\begin{td-exo}[] % 3
	Résoudre les équations différentielles suivantes en en donnant toutes les 
	solutions maximales:
	\begin{enumerate}
		\item \(x'+x=\sin t\),
		\item \(x'=3t^2-\frac xt\).
	\end{enumerate}
\end{td-exo}

\begin{td-exo}[] % 4
	On considère le système linéaire d'équations différentielles suivant:
	\begin{equation}\label{eq:1}
		\begin{cases}
			x'&=-5x+8y-4\\
			y'&=-4x+7y+3
		\end{cases}
		\quad\text{avec les conditions initiales}\quad
		\begin{cases}
			x(0)&=0\\
			y(0)&=1
		\end{cases}
	\end{equation}

	\begin{enumerate}
		\item Trouver les vecteurs propres et valeurs propres de la matrice suivante:
		\begin{equation*}
			\cal A=
			\begin{pmatrix}
				-5&8\\
				-4&7
			\end{pmatrix}
		\end{equation*}

		\item Expliquer pourquoi résoudre \(\eqref{eq:1}\) revient à résoudre le système
		suivant:
		\begin{equation}\label{eq:2}
			\begin{cases}
				a'&=3a+10\\
				b'&=-b-7
			\end{cases}
			\quad\text{avec les conditions initiales}\quad
			\begin{cases}
				a(0)&=2\\
				b(0)&=-1
			\end{cases}
		\end{equation}

		\begin{remark}[]
			Il est possible d'arriver à un système différent, ce qui n'est pas grave du
			moment que les calculs sont corrects et que le résultat est un système découplé.
		\end{remark}

		\item Trouver la solution du \(\eqref{eq:2}\) puis du \(\eqref{eq:1}\).
	\end{enumerate}
\end{td-exo}
% ----- Solutions exo 4
\iftoggle{showsolutions}{
	\begin{td-sol}[]\, % 4
		\begin{enumerate}
			\item \(\triangleright\) Les valeurs propres de la matrice \(\cal A\) sont les
			solutions de l'équation caractéristique
			\begin{equation*}
				\begin{aligned}
					\det(\cal A-\lambda I)&=0\\
					\begin{vmatrix}
						-5-\lambda&8\\
						-4&7-\lambda
					\end{vmatrix}&=0\\
					\lambda^2-2\lambda-3&=0
				\end{aligned}
			\end{equation*}
			Les solutions de cette équation sont \(\lambda_1=3\) et \(\lambda_2=-1\).

			\(\triangleright\) Les vecteurs propres sont les solutions du système
			\begin{equation*}
				\begin{cases}
					(-5-3)x+8y&=0\\
					-4x+(7-3)y&=0
				\end{cases}
			\end{equation*}
			On trouve alors les vecteurs propres
			\begin{equation*}
				\begin{aligned}
					\begin{pmatrix}
						1\\
						1
					\end{pmatrix}
					\quad\text{et}\quad
					\begin{pmatrix}
						2\\
						1
					\end{pmatrix}
				\end{aligned}
			\end{equation*}
			Enfin, on peut écrire la matrice \(\cal A\) sous forme diagonale
			\begin{equation*}
				\cal A = PDP^{-1}
			\end{equation*}
			avec
			\begin{equation*}
				\begin{aligned}
					P&=
					\begin{pmatrix}
						1&2\\
						1&1
					\end{pmatrix}\\
					D&=
					\begin{pmatrix}
						3&0\\
						0&-1
					\end{pmatrix}
				\end{aligned}
			\end{equation*}

			\item \(\triangleright\) On peut réécrire le système \(\eqref{eq:1}\) sous la forme
			\begin{equation*}
				\begin{pmatrix}
					x'\\
					y'
				\end{pmatrix}
				=\cal A
				\begin{pmatrix}
					x\\
					y
				\end{pmatrix}
				+\begin{pmatrix}
					-4\\
					3
				\end{pmatrix}
			\end{equation*}
			On peut alors poser
			\begin{equation*}
				\begin{pmatrix}
					a\\
					b
				\end{pmatrix}
				=\begin{pmatrix}
					x+2y\\
					x+y
				\end{pmatrix}
			\end{equation*}
			On a alors
			\begin{equation*}
				\begin{aligned}
					a'&=x'+2y'\\
					b'&=x'+y'
				\end{aligned}
			\end{equation*}
			On a donc
			\begin{equation*}
				\begin{aligned}
					a'&=-5a+8b-4\\
					b'&=-4a+7b+3
				\end{aligned}
			\end{equation*}
			qui est le système \(\eqref{eq:2}\).

			Comme 
			\begin{equation*}
				\begin{pmatrix}
					a\\
					b
				\end{pmatrix}
				=P^{-1}X
			\end{equation*}
			on a
			\begin{equation*}
				\begin{pmatrix}
					a(0)\\
					b(0)
				\end{pmatrix}
				=P^{-1}
				\begin{pmatrix}
					x(0)\\
					y(0)
				\end{pmatrix}
				=P^{-1}
				\begin{pmatrix}
					0\\
					1
				\end{pmatrix}
				=
				\begin{pmatrix}
					2\\
					-1
				\end{pmatrix}
			\end{equation*}
			et donc on a bien \(a(0)=2\) et \(b(0)=-1\).

			\item \(\triangleright\) Résolvons le système \(\eqref{eq:2}\)
			\begin{itemize}
				\item Pour \(a\), on a 
				\begin{equation*}
					a(t)=c e^{3t}-\frac{10}3
				\end{equation*}
				et avec la condition initiale \(a(0)=2\), on trouve \(c=\frac{16}3\).

				\item Pour \(b\), on a
				\begin{equation*}
					b(t)=d e^{-t}+7
				\end{equation*}
				et avec la condition initiale \(b(0)=-1\), on trouve \(d=6\).

				\item Les solutions du système \(\eqref{eq:2}\) sont donc
				\begin{equation*}
					\begin{aligned}
						a(t)&=\frac{16}3 e^{3t}-\frac{10}3\\
						b(t)&=6 e^{-t}+7
					\end{aligned}
				\end{equation*}
				Ce qui, traduit pour \(x\) et \(y\) donne:
				\begin{equation*}
					\begin{aligned}
						x(t)&=\frac{16}3 e^{3t}-\frac{10}3+ 12 e^{-t}-14=\frac{16}3 e^{3t}+12 e^{-t}-\frac{52}3\\
						y(t)&=\frac{16}3 e^{3t}-\frac{10}3+ 6 e^{-t}-7=\frac{16}3 e^{3t}+6 e^{-t}-\frac{31}3
					\end{aligned}
				\end{equation*}
			\end{itemize}
		\end{enumerate}
	\end{td-sol}
}{}

\begin{td-exo}[] % 5
	Pour \(x\in\R\) on pose
	\begin{equation*}
		\varphi(x)=\int_{-\infty}^{+\infty}e^{tx-t^2}\der t
	\end{equation*}

	\begin{enumerate}
		\item Montrer que \(\varphi\) est bien définie et continue sur \(\R\).
		\item Montrer que \(\varphi\) est \(C^1\) sur \(\R\) et exprimer 
		\(\varphi'(x)\) comme une intégrale à paramètre.
		\item En intégrant par parties l'expression de \(\varphi'(x)\), trouver
		un problème de Cauchy linéaire vérifié par \(\varphi\).
		\item Calculer \(\varphi(x)\).
	\end{enumerate}

\end{td-exo}

\begin{td-exo}[] % 6
	Calculer la solution maximale du problème de Cauchy suivant:
	\begin{equation*}
		\begin{cases}
			x'=1+x^2\\
			x(0)=0
		\end{cases}
	\end{equation*}

\end{td-exo}

\begin{td-exo}[] % 7
	Résoudre sur un intervalle à préciser le problème suivant:
	\begin{equation*}
		\begin{cases}
			xx'=\frac12\\
			x(0)=1
		\end{cases}
	\end{equation*}

\end{td-exo}

\begin{td-exo}[] % 8
	On considère le \defemph{modèle de Gompertz}, utilisé en dynamique des
	populations, dans lequel l'évolution de la population \(N(t)\) considérée
	est décrite par l'équation suivante:
	\begin{equation*}
		N'(t)=rN(t)\ln\left(\frac K{N(t)}\right)
	\end{equation*}

	où \(r\) et \(K>0\) sont des constantes données. Donner une expression de la
	population en fonction de la population initiale \(N(0)=N_0>0\). Que se 
	passe-t-il quand \(t\to\infty\)?
\end{td-exo}
% ----- Solutions exo 8
\iftoggle{showsolutions}{
	\begin{td-sol}[]\, % 8
		\begin{itemize}
			\item Utilisons Cauchy-Lipschitz pour résoudre 
			l'équation différentielle.

			\(\triangleright\) On cherche \(\varepsilon>0\) et
			\(\lambda\in\R\) tels que
			\begin{equation*}
				\begin{aligned}
					\left\{
						\begin{aligned}
							N'(t)&=rN(t)\ln\left(\frac K{N(t)}\right)\\
							N(0)&=N_0
						\end{aligned}
					\right .
				\end{aligned}
			\end{equation*}
			Soit \(t\in\R\), on a
			\begin{equation*}
				u\colon x\mapsto f(t,x)=r x\ln\left(\frac Kx\right)
			\end{equation*}
			Pour \(x_1,x_2\in\R_+^*\), il existe \(x_3\) tel que
			\begin{equation*}
				u'(x_3)=\frac{u(x_2)-u(x_1)}{x_2-x_1}
			\end{equation*}
			par le théorème des accroissements finis.

			Alors, 
			\begin{equation*}
				\n{u(x_2)-u(x_1)}=\n{u'(x_3)}\n{x_2-x_1}
			\end{equation*}

			On prend \(u'\) continue sur \(ff{x_0-\varepsilon,x_0+\varepsilon}\). Soit
			\begin{equation*}
				M=\sup_{x\in[x_0-\varepsilon,x_0+\varepsilon]}\n{u'(x)}
			\end{equation*}
			Si \(x_1,x_2\in[x_0-\varepsilon,x_0+\varepsilon]\), on a
			\begin{equation*}
				\begin{aligned}
					\n{u(x_2)-u(x_1)}&=\n{u'(x_3)}\n{x_2-x_1}\\
					&\leq M\n{x_2-x_1}
				\end{aligned}
			\end{equation*}

			Soit \((t_0,x_0)\in\R\times\R_+^*\). Soit \(\varepsilon=\frac{x_0}{2}>0\)
			et \(\lambda=\sup\limits_{\ff{x_0-\varepsilon,x_0+\varepsilon}}\n{u'}\).

			Soient \(t,x_1,x_2\) tels que
			\begin{equation*}
				\sup{\{\n{t-t_0},\n{x_1-x_0},\n{x_2-x_0}\}}<\varepsilon
			\end{equation*}
			On a \(x_1,x_2\in\ff{x_0-\varepsilon,x_0+\varepsilon}\) donc, par le
			T.A.F.

			\begin{equation*}
				\n{f(t,x_1)-f(t,x_2)}=\n{u(x_1)-u(x_2)}\le \lambda \n{x_1-x_2}
			\end{equation*}

			\(f\) est continue et localement Lipschitzienne en espace donc il existe
			une unique solution maximale

		\end{itemize}
	\end{td-sol}
}{}

\begin{td-exo}[] % n
\end{td-exo}


