\begin{td-exo}[]\, % 1
    \begin{enumerate}
        \item Soit \(\Omega\) un ensemble muni d'une tribu \(\scr F\) et 
        \(x\in\Omega\). Montrer que
        \begin{equation*}
            \delta_x(A) = \one_A(x)
        \end{equation*}
        définit une probabilité sur \((\Omega,\scr F)\).

        \item Soit \({(\bb P_n)}_{n\ge 1}\) une suite de mesures
        de probabilité sur un espace mesurable \((\Omega,\scr F)\) et
        \({(a_n)}_{n\ge 1}\) une suite de réels dans \(\ff{0,1}\) telle que
        \begin{equation*}
            \sum_{n=1}^{\infty} a_n = 1.
        \end{equation*}
        Montrer que
        \begin{equation*}
            \sum_{n=1}^{\infty} a_n \bb P_n
        \end{equation*}
        est une probabilité sur \((\Omega,\scr F)\).

        \item Soit \(I\) un intervalle de \(\R\) de mesure de Lebesgue
        \(\lambda(I)\) finie et strictement positive. Montrer que
        \begin{equation*}
            \bb P(A) = \frac{\lambda(A)}{\lambda(I)}
        \end{equation*}
        définit une probabilité sur \((I,\scr B(I))\).

        \item Soit \((\Omega,\scr F,\mu)\) un espace mesuré (pas forcément
        de probabilité) et \(f\colon \Omega\to\fo{0,\infty}\) une fonction
        mesurable telle que
        \begin{equation*}
            \int_{\Omega} f(\omega) \der\mu(\omega) = 1.
        \end{equation*}
        Montrer que l'application
        \begin{equation*}
            \begin{aligned}
                \bb P\colon \scr F &\to \bb R\\
                A &\mapsto \int_{\Omega} f(\omega) \one_A(\omega) \der\mu(\omega)
            \end{aligned}
        \end{equation*}
        est une probabilité sur \((\Omega,\scr F)\).
    \end{enumerate}
\end{td-exo}
% ----- Solutions exo 1
\iftoggle{showsolutions}{
    \begin{td-sol}[]\, % 1
        \begin{enumerate}
            \item Commençons par montrer que
            \begin{equation*}
                \delta_x(A)
            \end{equation*}
            est une mesure, puis que c'est une probabilité.

            \begin{itemize}[label=\(\triangleright\)]
                \item On a \(\delta_x(A)\) une mesure par définition.
                \item Comme \(x\in\Omega\), on a toujours \(\delta_x(\Omega)=1\).
            \end{itemize}

            Ainsi, comme \(\delta_x\) est une mesure et \(\delta_x(\Omega)=1\),
            on a bien que \(\delta_x\) est une probabilité.
        \end{enumerate}
    \end{td-sol}
}{}

\begin{td-exo}[] % 2
    On considère la mesure \(\bb P\) sur \((\R,\scr B(\R))\) définie par
    \begin{equation*}
        \bb P(A) = \frac 13 \delta_0 + \frac 13 \one_{oo{0,2}}(x)\lambda,
    \end{equation*}
    où \(\lambda\) est la mesure de Lebesgue sur \((\R,\scr B(\R))\).

    On peut imaginer que cette mesure représente le temps d'attente à un
    carrefour composé de trois feux piétons (rouge, vert), chaque feu
    restant au vert pendant une minute.

    \begin{enumerate}
        \item Montrer que \(\bb P\) est une probabilité sur \((\R,\scr B(\R))\).

        \item Calculer \(\bb P(\ff{a,b})\) pour tout intervalle \(0\le a<b\le 2\).

        \item Déterminer
        \begin{equation*}
            \int_{\R} x \der\bb P(x),\qquad\text{puis}\qquad\int_{\R} x^2 \der\bb P(x).
        \end{equation*}
    \end{enumerate}
\end{td-exo}

% ----- Solutions exo 2
\iftoggle{showsolutions}{
    \begin{td-sol}[] % 2
        ~
    \end{td-sol}
}{}

\begin{td-exo}[] % 3
    Soit \((\Omega,\scr F,\bb P)\) un espace probabilisé.

    \begin{enumerate}
        \item Soient \(A\) et \(B\) deux événements.
        \begin{enumerate}
            \item Montrer que
            \begin{equation*}
                \bb P(A) + \bb P(B) - 1 \le \bb P(A\cap B) \le \min\{\bb P(A),\bb P(B)\}.
            \end{equation*}

            \item On considère le lancer d'un dé équilibré. Proposer
            un exemple d'événements \(A\) et \(B\) (d'intersection non vide)
            pour lequel l'inégalité de gauche est une égalité. Même question
            pour l'inégalité de droite.
        \end{enumerate}

        \item Montrer que si \(A_1,\ldots,A_n\) sont des \(n\) événements,
        alors
        \begin{equation*}
            \bb P(A_1) + \cdots + \bb P(A_n) - (n-1) \le \bb P(A_1\cap\cdots\cap A_n)
            \le \min_{1\le i\le n} \bb P(A_i).
        \end{equation*}
    \end{enumerate}
    
\end{td-exo}