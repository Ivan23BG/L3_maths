\begin{td-exo}[]\, % 1
    \begin{enumerate}
        \item Soit \(\Omega\) un ensemble muni d'une tribu \(\scr F\) et 
        \(x\in\Omega\). Montrer que
        \begin{equation*}
            \delta_x(A) = \one_A(x)
        \end{equation*}
        définit une probabilité sur \((\Omega,\scr F)\).

        \item Soit \({(\bb P_n)}_{n\ge 1}\) une suite de mesures
        de probabilité sur un espace mesurable \((\Omega,\scr F)\) et
        \({(a_n)}_{n\ge 1}\) une suite de réels dans \(\ff{0,1}\) telle que
        \begin{equation*}
            \sum_{n=1}^{\infty} a_n = 1.
        \end{equation*}
        Montrer que
        \begin{equation*}
            \sum_{n=1}^{\infty} a_n \bb P_n
        \end{equation*}
        est une probabilité sur \((\Omega,\scr F)\).

        \item Soit \(I\) un intervalle de \(\R\) de mesure de Lebesgue
        \(\lambda(I)\) finie et strictement positive. Montrer que
        \begin{equation*}
            \bb P(A) = \frac{\lambda(A)}{\lambda(I)}
        \end{equation*}
        définit une probabilité sur \((I,\scr B(I))\).

        \item Soit \((\Omega,\scr F,\mu)\) un espace mesuré (pas forcément
        de probabilité) et \(f\colon \Omega\to\fo{0,\infty}\) une fonction
        mesurable telle que
        \begin{equation*}
            \int_{\Omega} f(\omega) \der\mu(\omega) = 1.
        \end{equation*}
        Montrer que l'application
        \begin{equation*}
            \begin{aligned}
                \bb P\colon \scr F &\to \bb R\\
                A &\mapsto \int_{\Omega} f(\omega) \one_A(\omega) \der\mu(\omega)
            \end{aligned}
        \end{equation*}
        est une probabilité sur \((\Omega,\scr F)\).
    \end{enumerate}
\end{td-exo}
% ----- Solutions exo 1
\iftoggle{showsolutions}{
    \begin{td-sol}[]\, % 1
        \begin{enumerate}
            \item Commençons par montrer que
            \begin{equation*}
                \delta_x(A)
            \end{equation*}
            est une mesure, puis que c'est une probabilité.

            \begin{itemize}[label=\(\triangleright\)]
                \item Pour faire un rappel, vérifions les propriétés d'une mesure:
                \begin{itemize}
                    \item On a bien \(\delta_x(\varnothing) = 0\) pour tout \(x\in\Omega\)

                    \item On considère \({(A_n)}_{n\in\N}\) une suite d'ensembles 2 à 2
                    disjoints de \(\scr F\). Alors
                    \begin{equation*}
                        \begin{aligned}
                            \delta_x\left(\bigcup_{n\in\N}A_n\right)
                            &= \one_{\cup_{n=1}^\infty A_n}(x)\\
                            &=\sum_{n=1}^\infty \one_{A_n}(x)\\
                            &=\sum_{n=0}^\infty \delta_x A_n
                        \end{aligned}
                    \end{equation*}
                \end{itemize}
                Ainsi, \(\delta_x(A)\) est bien une mesure.
                \item Comme \(x\in\Omega\), on a toujours \(\delta_x(\Omega)=1\).
            \end{itemize}

            Ainsi, comme \(\delta_x\) est une mesure et \(\delta_x(\Omega)=1\),
            on a bien que \(\delta_x\) est une probabilité.

            \item On considère
            \begin{equation*}
                \bb P = \sum_{n=1}^\infty a_n\bb P_n
            \end{equation*}
            Vérifions les hypothèses d'une probabilité:
            \begin{itemize}[label=\(\triangleright\)]
                \item On a
                \begin{equation*}
                    \bb P(\varnothing) = \sum_{n=1}^\infty a_n\underbrace{\bb P_n(\varnothing)}_{=0}=0
                \end{equation*}

                \item On considère \({(A_k)}_{k\in\N}\) une suite d'ensembles 2 à 2
                disjoints. Alors
                \begin{equation*}
                    \begin{aligned}
                        \bb P\left(\bigcup_{k=0}^\infty A_k\right)
                        &= \sum_{n=1}^\infty a_n\bb P_n\left(\bigcup_{k=0}^\infty A_k\right) \\
                        &= \sum_{n=1}^\infty a_n \sum_{k=0}^\infty \bb P(A_k)\\
                        &= \sum_{k=0}^\infty \underbrace{\sum_{n=1}^\infty a_n\bb P_n(A_k)}_{=\bb P(A_k)}\\
                        &= \sum_{k=0}^\infty \bb P(A_k)
                    \end{aligned}
                \end{equation*}
                On rappelle qu'on peut intervertir les deux sommes car tous les éléments sont positifs

                \item Enfin, on a
                \begin{equation*}
                    \bb P(\Omega) = \sum_{n=1}^\infty a_n\underbrace{\bb P_n(\Omega)}_{=1} = \sum_{n=1}^\infty a_n = 1.
                \end{equation*}
            \end{itemize}

            \item On considère
            \begin{equation*}
                \bb P(A) = \frac{\lambda(A)}{\lambda(I)}
            \end{equation*}

            Comme \(\lambda\) est une mesure et \(\frac{1}{\lambda(I)}\ge 0\),
            on a bien que \(\bb P\) est une mesure. Montrons maintenant que c'est
            une probabilité:

            \(\triangleright\) Comme
            \begin{equation*}
                \bb P(I) = \frac{\lambda(I)}{\lambda(I)} = 1
            \end{equation*}
            on a bien que \(\bb P\) est une probabilité.

            \item On considère
            \begin{equation*}
                \bb P(A) = \int_{\Omega}f(\omega)\one_A(\omega)\der\mu(\omega) = \int_A f(\omega)\der\mu(\omega)
            \end{equation*}
            On vérifie les hypothèses:
            \begin{itemize}[label=\(\triangleright\)]
                \item On a
                \begin{equation*}
                    \bb P(\varnothing) = \int_{\Omega} f(\omega)\one_{\varnothing}(\omega)\der\mu(\omega)=0
                \end{equation*}

                \item On considère \({(A_n)}_{n\in\N}\) une suite d'ensembles 2 à 2 disjoints de \(\scr F\). Alors
                \begin{equation*}
                    \begin{aligned}
                        \bb P\left(\bigcup_{n=0}^\infty A_n\right)
                        &= \int_{\Omega}f(\omega) \one_{\cup_{n=0}^\infty A_n}(\omega)\der\mu(\omega)\\
                        &= \int_{\Omega}f(\omega)\sum_{n=0}^\infty \one_{A_n}(\omega)\der\mu(\omega)
                    \end{aligned}
                \end{equation*}
                Comme on a la positivité et la somme est finie, on peut appliquer le théorème
                de convergence monotone pour intervertir somme et integrale. Alors
                \begin{equation*}
                    \begin{aligned}
                        \bb P\left(\bigcup_{n=0}^\infty A_n\right)
                        &= \int_{\Omega}f(\omega)\sum_{n=0}^\infty \one_{A_n}(\omega)\der\mu(\omega)\\
                        &= \sum_{n=0}^{\infty} \int_{\Omega} f(\omega)\one_{A_n}(\omega)\der\mu(\omega)\\
                        &= \sum_{n=0}^{\infty} \bb P(A_n)
                    \end{aligned}
                \end{equation*}

                \item On a
                \begin{equation*}
                    \begin{aligned}
                        \bb P(\Omega)
                        &= \int_{\Omega} f(\omega)\underbrace{\one_{\Omega}(\omega)}_{=1}\der\mu(\omega)\\
                        &= \int_{\Omega} f(\omega)\der\mu(\omega)\\
                        &= 1
                    \end{aligned}
                \end{equation*}
            \end{itemize}

            On a donc montré que \(\bb P\) est une probabilité
        \end{enumerate}
    \end{td-sol}
}{}

\begin{td-exo}[] % 2
    On considère la mesure \(\bb P\) sur \((\R,\scr B(\R))\) définie par
    \begin{equation*}
        \bb P(A) = \frac 13 \delta_0 + \frac 13 \one_{oo{0,2}}(x)\lambda,
    \end{equation*}
    où \(\lambda\) est la mesure de Lebesgue sur \((\R,\scr B(\R))\).

    On peut imaginer que cette mesure représente le temps d'attente à un
    carrefour composé de trois feux piétons (rouge, vert), chaque feu
    restant au vert pendant une minute.

    \begin{enumerate}
        \item Montrer que \(\bb P\) est une probabilité sur \((\R,\scr B(\R))\).

        \item Calculer \(\bb P(\ff{a,b})\) pour tout intervalle \(0\le a<b\le 2\).

        \item Déterminer
        \begin{equation*}
            \int_{\R} x \der\bb P(x),\qquad\text{puis}\qquad\int_{\R} x^2 \der\bb P(x).
        \end{equation*}
    \end{enumerate}
\end{td-exo}

% ----- Solutions exo 2
\iftoggle{showsolutions}{
    \begin{td-sol}[]\, % 2
        On rappelle que
        \begin{equation*}
            \bb P = \frac 13 \delta_0 + \frac 13 \one_{\oo{0,2}}(x)\lambda
        \end{equation*}
        et
        \begin{equation*}
            \begin{aligned}
                \bb P(A) 
                &= \frac 13 \delta_0(A) + \frac 13 \int_A \one_{\oo{0,2}}(x) \der\lambda(x)\\
                &= \frac 13 \delta_0(A) + \frac 13 \lambda\left(\oo{0,2}\cap A\right)
            \end{aligned}
        \end{equation*}

        \begin{enumerate}
            \item On vérifie que c'est une probabilité:
            \begin{equation*}
                \bb P = \frac 13 \bb P_1 + \frac 23 \bb P_2
            \end{equation*}
            avec \(\bb P_1=\delta_0\) et \(\bb P_2 = \frac 12\lambda\left(\oo{0,2}\cap\cdot\right)\).

            L'exercice 1 assure que \(\bb P\) est une probabilité

            \item On a
            \begin{equation*}
                \bb P_1\left(\ff{a,b}\right) = \delta_0\left(\ff{a,b}\right) = \one_{\{0\}}(a)
            \end{equation*}
            et
            \begin{equation*}
                \bb P_2\left(\ff{a,b}\right) = \frac 12 \lambda\left(\oo{0,2}\cap\ff{a,b}\right) = \frac{b-a}2
            \end{equation*}
            Alors
            \begin{equation*}
                \bb P\left(\ff{a,b}\right) = \frac 13 \one_{\{0\}}(a) = \frac 13(b-a)
            \end{equation*}

            \item On a
            \begin{equation*}
                \begin{aligned}
                    \int_{\bb R} x\der \bb P(x)
                    &= \frac 13 \int_{\bb R} x\der\delta_0(x) + \frac 13 \int_{\bb R} x\one_{\oo{0,2}}(x)\der\lambda(x)\\
                    &=\frac 13\times 0 + \frac 13 {\left[\frac{x^2}2\right]}_0^2\\
                    &= \frac 23
                \end{aligned}
            \end{equation*}
            Il suit
            \begin{equation*}
                \begin{aligned}
                    \int_{\bb R} x^2\der\bb P(x)
                    &= \frac 13 \int_{\bb R} x^2\der\delta_0(x) + \frac 13 \int_{\bb R} x^2\one_{\oo{0,2}}(x)\der\lambda(x)\\
                    &= \frac 13\times 0^2 + \frac 13{\left[\frac{x^3}3\right]}_0^2\\
                    &=\frac 89
                \end{aligned}
            \end{equation*}
        \end{enumerate}
    \end{td-sol}
}{}

\begin{td-exo}[] % 3
    Soit \((\Omega,\scr F,\bb P)\) un espace probabilisé.

    \begin{enumerate}
        \item Soient \(A\) et \(B\) deux événements.
        \begin{enumerate}
            \item Montrer que
            \begin{equation*}
                \bb P(A) + \bb P(B) - 1 \le \bb P(A\cap B) \le \min\{\bb P(A),\bb P(B)\}.
            \end{equation*}

            \item On considère le lancer d'un dé équilibré. Proposer
            un exemple d'événements \(A\) et \(B\) (d'intersection non vide)
            pour lequel l'inégalité de gauche est une égalité. Même question
            pour l'inégalité de droite.
        \end{enumerate}

        \item Montrer que si \(A_1,\ldots,A_n\) sont des \(n\) événements,
        alors
        \begin{equation*}
            \bb P(A_1) + \cdots + \bb P(A_n) - (n-1) \le \bb P(A_1\cap\cdots\cap A_n)
            \le \min_{1\le i\le n} \bb P(A_i).
        \end{equation*}
    \end{enumerate}
\end{td-exo}

% ----- Solutions exo 2
\iftoggle{showsolutions}{
    \begin{td-sol}[]\, % 2
        \begin{enumerate}
            \item On a
            \begin{enumerate}
                \item \(\bb P(A) + \bb P(B) - \bb P(A\cap B)\le 1\)
                et \(A\cap B \subset A,\quad A\cap B\subset B\), on en déduit
                facilement les inégalités.

                \item On peut considérer
                \begin{equation*}
                    A = \{1, 2, 3\},\quad B = \{3, 4, 5, 6\}
                \end{equation*}
                Alors 
                \begin{equation*}
                    \begin{aligned}
                        \bb P(A) + \bb P(B) - 1
                        &= \frac 36 + \frac 46 - \frac 66
                        &= \frac {7-6}6
                        &= \frac 16
                        &= \bb P(\{3\})
                        &= \bb P(A\cap B)
                    \end{aligned}
                \end{equation*}
                
                Pour la deuxième partie, on peut considérer
                \begin{equation*}
                    A= \{1\}, \quad B = \{1,2\}
                \end{equation*}
                Alors
                \begin{equation*}
                    \begin{aligned}
                        \bb P(A\cap B)
                        &= \bb P(\{1\})
                        &= \frac 16
                        &\le \frac 16
                        &= \min\left(\bb P(A), \bb P(B)\right)
                    \end{aligned}
                \end{equation*}
            \end{enumerate}

            \item On fait une preuve par récurrence:

            \begin{itemize}[label=\(\triangleright\)]
                \item Hypothèse: 
                \begin{equation*}
                    \bb P(n) = \sum_{i=1}^n\bb P(A_i)-(n-1)\le\bb P(\cap_{i=1}^n A_i)
                \end{equation*}
                l'initialisation est montrée à l'exercice 1.

                \item Hérédité:

                On suppose \(\bb P(n)\) vrai pour un \(n\in\bb N\) fixé, prouvons \(\bb P(n+1)\):
                \begin{equation*}
                    \begin{aligned}
                        \bb P(\cap_{i=1}^{n+1} A_i)
                        &= \bb P(\cap_{i=1}^n A_i\cap A_{n+1})\\
                        &\geq \bb P(\cap_{i=1}^n A_i) + \bb P(A_{n+1}) -1\\
                        &\geq \sum_{i=1}^{n+1} \bb P(A_i) - n
                    \end{aligned}
                \end{equation*}
                Pour la deuxième inégalité on a
                \begin{equation*}
                    \begin{aligned}
                        \bb P(A_1\cap\cdots\cap A_n)
                        &\le \min\left(\bb P(A_1\cap\cdots\cap A_{n-1}), \bb P(A_n)\right)\\
                        &\le \min\left(\min \bb P(A_1\cap\cdots\cap A_{n-1})\right)\cdots\\
                        &\le \min_{i\in\{1,\dots,n\}} \bb P(A_i)
                    \end{aligned}
                \end{equation*}
            \end{itemize}
        \end{enumerate}
    \end{td-sol}
}{}