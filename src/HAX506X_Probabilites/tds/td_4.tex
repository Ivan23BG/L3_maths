\begin{td-exo}[] % 1
    Rappeler la définitions de la fonctions caractéristique
    \(\varphi_X\) d'une variable aléatoire \(X\) et 
    calculer \(\varphi_X\) pour les lois suivantes:
    \begin{enumerate}
        \item \(X\) suit une loi uniforme sur \(\ff{a,b}\),
        \item \(X\) suit une loi exponentielle de paramètre \(\lambda\),
        \item \(X\) suit une loi géométrique de paramètre \(p\in\oo{0,1}\),
        \item \(X\) suit une loi normale centrée réduite. On pourra
        dériver (en justifiant) la fonction caractéristique \(\varphi_X\)
        puis, après intégration par parties, en déduire que \(\varphi_X\)
        est solution d'une équation différentielle que l'on pourra résoudre.
    \end{enumerate}
\end{td-exo}
% ----- Solutions exo 1
\iftoggle{showsolutions}{
    \begin{td-sol}[] % 1
        On rappelle dans le cas général que la fonction caractéristique
        d'une variable aléatoire \(X\) est définie par
        \begin{equation*}
            \varphi_X(t) = \bb E\left(e^{itX}\right), \quad t\in\bb R
        \end{equation*}
        \begin{enumerate}
            \item On a
            \begin{equation*}
                f_X(x) = \frac{1}{b-a}\one_{\ff{a,b}}(x)
            \end{equation*}
            alors
            \begin{equation*}
                \begin{aligned}
                    \varphi_X(t)
                    &= \bb E\left(e^{itX}\right)\\
                    &= \int_{\bb R} e^{itx} \underbrace{f_X(x) \der\lambda_1(x)}_{\der\bb P_X(x)}\\
                    &= \int_{a}^{b} \frac{e^{itx}}{b-a} \der \lambda_1(x)\\
                    &= \frac{1}{b-a} {\left[\frac{e^{itx}}{it}\right]}_a^b\\
                    &= \frac{e^{itb} - e^{ita}}{it(b-a)}\\
                \end{aligned}
            \end{equation*}

            \item On a
            \begin{equation*}
                f_X(t) = \theta e^{-\theta t}\one_{\bb R_+}(t)
            \end{equation*}
            alors
            \begin{equation*}
                \begin{aligned}
                    \varphi_X(t)
                    &= \bb E\left(e^{itX}\right)\\
                    &= \int_{\bb R} e^{itx} \underbrace{f_X(x) \der\lambda_1(x)}_{\der\bb P_X(x)}\\
                    &= \int_{0}^{+\infty} e^{itx} \theta e^{-\theta x} \der \lambda_1(x)\\
                    &= \theta {\left[\frac{e^{(it-\theta)x}}{it-\theta}\right]}_0^{+\infty}\\
                    &= \frac{\theta}{\theta - it}
                \end{aligned}
            \end{equation*}

            \item On a
            \begin{equation*}
                \bb P(X = n) = p{(1-p)}^{n-1}, \quad n\in\bb N
            \end{equation*}
            et
            \begin{equation*}
                \bb P_X = \sum_{n=1}^{+\infty} p{(1-p)}^{n-1}\delta_n
            \end{equation*}
            alors
            \begin{equation*}
                \begin{aligned}
                    \varphi_X(t)
                    &= \bb E\left(e^{itX}\right)\\
                    &= \int_{\bb R} e^{itx} \der \left(\sum_{n=1}^{+\infty} p{(1-p)}^{n-1}\delta_n(x)\right)\\
                    &= \sum_{n=1}^{+\infty} p{(1-p)}^{n-1} \underbrace{\int_{\bb R} e^{itx} \der \delta_n(x)}_{e^{itn}}\\
                    &= e^{it} p \sum_{n=1}^{+\infty} {(1-p)}^{n-1} e^{it(n-1)}\\
                    &= \frac{pe^{it}}{1-(1-p)e^{it}}
                \end{aligned}
            \end{equation*}

            \item On a
            \begin{equation*}
                f_X(x) = \frac{e^{-\frac{x^2}{2}}}{\sqrt{2\pi}}
            \end{equation*}
            alors
            \begin{equation*}
                \begin{aligned}
                    \varphi_X(t)
                    &= \bb E\left(e^{itX}\right)\\
                    &= \int_{\bb R} \underbrace{e^{itx} \frac{e^{-\frac{x^2}{2}}}{\sqrt{2\pi}}}_{h(t,x)} \der(x)\\
                \end{aligned}
            \end{equation*}
            \(\triangleright\) À \(t\) fixé, on a \(x\mapsto h(t,x)\) intégrable
            par rapport à \(\lambda_1\).

            \(\triangleright\) À \(x\) fixé, on a \(t\mapsto h(t,x)\) dérivable
            par rapport à \(t\)

            \(\triangleright\) On dérive par rapport à \(t\)
            \begin{equation*}
                \begin{aligned}
                    \n{\frac{\partial h}{\partial t}(t,x)} 
                    &= \n{ix e^{itx} \frac{e^{-\frac{x^2}{2}}}{\sqrt{2\pi}}}\\
                    &= \frac{\n x e^{-\frac{x^2}{2}}}{\sqrt{2\pi}}\\
                    &= o_{\pm \infty}(\frac{1}{x^2})
                \end{aligned}
            \end{equation*}

            D'après le théorème de dérivation sous l'intégrale, on a
            \begin{equation*}
                \begin{aligned}
                    \varphi_X'(t)
                    &= \int_{\bb R} i x e^{itx} \frac{e^{-\frac{x^2}{2}}}{\sqrt{2\pi}}\der x\\
                    &= {\n{-i e^{itx} \frac{e^{-\frac{x^2}{2}}}{\sqrt{2\pi}}}}_{-\infty}^{+\infty} - \int_{\bb R} -i \frac{e^{-\frac{x^2}{2}}}{\sqrt{2\pi}} i t e^{itx} \der x\\
                    &= -t \varphi_X(t)
                \end{aligned}
            \end{equation*}

            Cette équation différentielle a pour solution
            \begin{equation*}
                \varphi_X(t) = C e^{-\frac{t^2}{2}}, \quad C\in\bb R, t\in\bb R
            \end{equation*}
            Comme \(\varphi_X(0) = 1\), on trouve
            \begin{equation*}
                \varphi_X(t) = e^{-\frac{t^2}{2}}
            \end{equation*}
        \end{enumerate}
    \end{td-sol}
}{}

\begin{td-exo}[] % 2
    Soit \(X\simeq\mathcal N(0,1)\).
    \begin{enumerate}
        \item Calculer la fonction de répartition de \(-X\) en
        fonction de celle de \(X\). Qu'en déduit-on?

        \item On pose \(Y = X^2\). Déterminer la fonction de répartition
        de \(Y\) en fonction de celle de \(X\). En déduire que \(Y\)
        est une variable aléatoire à densité, puis calculer \(\bb E\ff Y\).

        \item Reprendre la questions précédente avec \(Z = \exp X\).
    \end{enumerate}
\end{td-exo}
% ----- Solutions exo 2
\iftoggle{showsolutions}{
    \begin{td-sol}[]\, % 2
        \begin{enumerate}
            \item On prend \(t\in\bb R\), alors
            \begin{equation*}
                \begin{aligned}
                    F_{-X}(t) 
                    &= \bb P(-X\leq t) = \bb P(X\geq -t) \\
                    & \bb P(X\in \fo{-t,+\infty}) = \int_{-t}^{+\infty} f_X(x) \der x\\
                    \text{\footnotesize{\(x=-u\)}}&= \int_{t}^{-\infty} f_X(-u) -\der u\\
                    \text{\footnotesize{\(f_X\) est paire}}&= \int_{-\infty}^{t} f_X(u) \der u\\
                    &= \bb P(X\leq t) = F_X(t)
                \end{aligned}
            \end{equation*}
            Donc \(X\) et \(-X\) ont la même fonction de répartition
            et donc la même loi.

            \item On sait que \(Y = X^2\) est à valeurs dans \(\bb R_+\).
            Donc 
            \begin{equation*}
                F_Y(t) = \bb P(Y\leq t) = 0, \quad \text{si } t<0
            \end{equation*}
            Dans le cas où \(t\geq 0\), on a
            \begin{equation*}
                \begin{aligned}
                    \bb P(Y\leq t)
                    &= \bb P(X^2\leq t) = \bb P(\n X\leq \sqrt t)\\
                    &= \bb P(-\sqrt t\leq X\leq \sqrt t)\\
                    &= 2 \bb P(0\leq X\leq \sqrt t)\\
                    &= 2 \int_{0}^{\sqrt t} \frac{e^{-\frac{x^2}{2}}}{\sqrt{2\pi}} \der x\\
                    &= 2 \int_{-\infty}^{\sqrt t} \frac{e^{-\frac{x^2}{2}}}{\sqrt{2\pi}} \der x - \underbrace{2 \int_{-\infty}^{0} \frac{e^{-\frac{x^2}{2}}}{\sqrt{2\pi}} \der x}_{1}\\
                    &= 2 F_X(\sqrt t) - 1
                \end{aligned}
            \end{equation*}
            Il suit
            \begin{equation*}
                F_Y(t) = \begin{cases}
                    0 & \text{si } t<0\\
                    2 F_X(\sqrt t) - 1 & \text{si } t\geq 0
                \end{cases}
            \end{equation*}
            et donc \(F_Y\) est dérivable sur \(\bb R^*\) et
            \begin{equation*}
                F_Y'(t) = \frac{F_X'(\sqrt t)}{\sqrt t} \one_{\bb R_+}(t) = \frac{e^{-\frac{t}{2}}}{\sqrt{t2\pi}} \one_{\bb R_+}(t)
            \end{equation*}
            et donc \(Y\) a pour densité
            \begin{equation*}
                f_Y(t) = \frac{e^{-\frac{t}{2}}}{\sqrt{t2\pi}} \one_{\bb R_+}(t)
            \end{equation*}

            On a
            \begin{equation*}
                \begin{aligned}
                    \bb E\ff Y 
                    &= \int_0^{+\infty} y \frac{e^{-\frac{y}{2}}}{\sqrt{2\pi y}} \der y\\
                    \bb E\ff Y
                    &= E\ff{X^2} = \int_{\bb R} x^2 \frac{e^{-\frac{x^2}{2}}}{\sqrt{2\pi}} \der x\\
                    &= 1
                \end{aligned}
            \end{equation*}
            car on a la variance de \(X\) est égale à 1.

            \item On sait que \(Z = \exp X\) est à valeurs dans \(\bb R_+\).
            \begin{equation*}
                F_Z(t) = 0, \quad \text{si } t<0
            \end{equation*}
            Dans le cas où \(t\geq 0\), on a
            \begin{equation*}
                \begin{aligned}
                    F_Z(t) 
                    &= \bb P(Z\leq t)\\
                    &= \bb P(\exp X\leq t) = \bb P(X\leq \ln t)\\
                    &= F_X(\ln t)
                \end{aligned}
            \end{equation*}
            Donc \(F_Z\) est dérivale sur \(\bb R^*\) et
            \begin{equation*}
                \begin{aligned}
                    F_Z'(t) 
                    &= \frac{F_X'(\ln t)}{t} \one_{\bb R_+^*}(t)\\
                    &= \frac{e^{-\frac{{(\ln t)}^2}{2}}}{t\sqrt{2\pi}} \one_{\bb R_+^*}(t)\\
                    &= f_Z(t)
                \end{aligned}
            \end{equation*}

            On a
            \begin{equation*}
                \begin{aligned}
                    \bb E\ff Z 
                    &= \bb E\ff{e^X} = \int_{\bb R} e^x \frac{e^{-\frac{x^2}{2}}}{\sqrt{2\pi}} \der x\\
                    &= \int_{\bb R} \frac{e^{-\frac{1}{2}{(x-1)}^2 + \frac12}}{\sqrt{2\pi}} \der x\\
                    &= \sqrt e \int_{\bb R} \frac{e^{-\frac{1}{2}{(x-1)}^2}}{\sqrt{2\pi}} \der x\\
                    &= \sqrt e
                \end{aligned}
            \end{equation*}
        \end{enumerate}
    \end{td-sol}
}{}

\begin{td-exo}[] % 3
    On considère une variable aléatoire \(X\) à valeurs dans \(\bb N\)
    de fonction génératrice
    \begin{equation*}
        G_X(s) = \alpha {(3 + 2s^2)}^3, \quad s\in\ff{0,1}
    \end{equation*}
    \begin{enumerate}
        \item Déterminer la valeur de \(\alpha\).

        \item Déterminer la loi de \(X\).

        \item À partir de \(G_X\), donner les valeurs de
        l'espérance et de la variance de \(X\).
    \end{enumerate}
\end{td-exo}
% ----- Solutions exo 3
\iftoggle{showsolutions}{
    \begin{td-sol}[] % 3
        On rappelle
        \begin{equation*}
            G_X(s) = \bb E\ff{s^X} = \sum_{n=0}^{+\infty} s^n \bb P(X = n)
        \end{equation*}
        \begin{enumerate}
            \item On a \(G_X(1) = 1\) donne \(\alpha = \frac{1}{125}\).

            \item On a
            \begin{equation*}
                \begin{aligned}
                    G_X(s) 
                    &= \frac{1}{125} {(3 + 2s^2)}^3\\
                    &= \frac{1}{125} \left(27 + 3\times 3^2 \times 2s^2 + 3\times 3 \times {(2s^2)}^2 + 2^3 s^6\right)\\
                    &= \frac{27}{125} + \frac{54}{125} s^2 + \frac{36}{125} s^4 + \frac{8}{125} s^6
                \end{aligned}
            \end{equation*}
            D'autre part, on a
            \begin{equation*}
                \begin{aligned}
                    G_X(s) 
                    &= \sum_{n=0}^{+\infty} s^n \bb P(X = n)\\
                    &= \bb P(X = 0) + s^2 \bb P(X = 2) + s^4 \bb P(X = 4) + s^6 \bb P(X = 6)
                \end{aligned}
            \end{equation*}
            Par unicité du développement en série entière d'un polynôme,
            la loi de \(X\) est donnée par
            \begin{equation*}
                \begin{aligned}
                    \bb P(X = 0) &= \frac{27}{125}\\
                    \bb P(X = 2) &= \frac{54}{125}\\
                    \bb P(X = 4) &= \frac{36}{125}\\
                    \bb P(X = 6) &= \frac{8}{125}
                \end{aligned}
            \end{equation*}

            \item On a
            \begin{equation*}
                \begin{aligned}
                    G_X(s) = \bb E\ff{s^X}\\
                    G_X'(s) = \bb E\ff{Xs^{X-1}}\\
                \end{aligned}
            \end{equation*}
            donc \(G_X'(1) = \bb E\ff{X}\).
        \end{enumerate}
    \end{td-sol}
}{}