\begin{td-exo}[]\, % 1
    On tire deux fois avec remise dans une urne contenant trois boules
    numérotées \(1, 2, 3\). On désigne par \(X\) la somme des résultats
    obtenus. Montrer que \(X\) est une variable aléatoire discrète entre
    un espace probabilisé et un espace mesurable à déterminer. Donner la
    loi de \(X\).
\end{td-exo}
% ----- Solutions exo 1
\iftoggle{showsolutions}{
    \begin{td-sol}[]\, % 1
        On veut montrer que \(X\) est une application mesurable de
        \(\Omega\) dans \(E\). Déterminons d'abord l'espace de départ
        \(\left(\Omega, \scr F, \bb P\right)\).

        \(\triangleright\) L'espace de départ correspond à l'ensemble
        des résultats possibles pour chaque boule tirée, donc
        \begin{equation*}
            \Omega = {\left\{1,2,3\right\}}^2
        \end{equation*}
        qu'on munit de la tribu pleine.

        \(\triangleright\) L'espace d'arrivée correspond aux résultats
        étudiés, soit ici
        \begin{equation*}
            E = \left\{2,3,4,5,6\right\}
        \end{equation*}
        qu'on munit aussi de la tribu pleine.

        \(\triangleright\) Alors, on a
        \begin{equation*}
            \begin{aligned}
                X:\Omega &\to E\\
                (a,b) &\mapsto a+b
            \end{aligned}
        \end{equation*}

        \(\triangleright\) La loi de \(X\) est donc
        \begin{equation*}
            \bb P_X(\omega) =
            \begin{cases}
                \frac 19 &\text{ si } \omega\in\{2,6\}
                \frac 29 &\text{ si } \omega\in\{3,5\}
                \frac 13 &\text{ si } \omega\in\{4\}
            \end{cases}
        \end{equation*}
    \end{td-sol}
}{}

\begin{td-exo}[] % 2
    On suppose que la basketteuse française Marine Johannés a une 
    probabilité 0.8 de marquer un lancer franc.
    \begin{enumerate}
        \item Lors d'un entraînement, elle tente une série de 10 lancers
        francs. On désigne par \(X\) la variable aléatoire donnant le
        nombre de paniers marqués.
        \begin{enumerate}
            \item Donner la loi de \(X\).
            
            \item Déterminer la probabilité que Marine Johannés marque
            huit paniers ou plus.

            \item Donner l'espérance et la variance de \(X\).
        \end{enumerate}

        \item Lors d'un autre entraînement, Marine Johannés décide de
        tirer jusqu'à ce qu'elle inscrive un panier. On désigne par \(Y\)
        la variable aléatoire donnant le nombre d'essais nécessaires.
        \begin{enumerate}
            \item Donner la loi de \(Y\).

            \item Déterminer la probabilité que Marine Johannés ait besoin
            de strictement plus de 3 essais pour marquer son premier panier.

            \item Donner l'espérance et la variance de \(Y\).
        \end{enumerate}
    \end{enumerate}
\end{td-exo}
% ----- Solutions exo 2
\iftoggle{showsolutions}{
    \begin{td-sol}[]\, % 2
        \begin{enumerate}
            \item Marine Johannés fait 10 lancers avec une probabilité
            de réussite de 0.8.
            \begin{enumerate}
                \item La loi de \(X\) est donnée par \(\mathcal B(10, 0.8)\).

                \item La probabilité de marquer 8 paniers ou plus vaut
                \begin{equation*}
                    ({0.8}^8 * {0.2}^2) + ({0.8}^9 * {0.2}) + ({0.8}^{10})
                \end{equation*}

                \item L'espérance et la variance de \(X\) sont respectivement
                \begin{equation*}
                    \begin{aligned}
                        \bb E(X) &= 10*0.8 = 8\\
                        \bb V(X) &= 10*0.8*0.2 = 1.6
                    \end{aligned}
                \end{equation*}
            \end{enumerate}

            \item On considère la variable aléatoire \(Y\) donnant le
            nombre d'essais nécessaires pour marquer un panier.
            \begin{enumerate}
                \item La loi de \(Y\) est donnée par
                \begin{equation*}
                    \bb P_Y(n) = 0.8 * {0.2}^{n - 1}
                \end{equation*}

                \item La probabilité que Marine Johannés ait besoin de
                strictement plus de 3 essais pour marquer son premier
                panier est équivalente à la probabilité qu'elle ne marque
                pas son premier panier en 1, 2 ou 3 essais, soit
                \begin{equation*}
                    1 - \left(0.8 + 0.8*0.2 + 0.8*{0.2}^2\right)
                \end{equation*}

                \item L'espérance et la variance de \(Y\) sont respectivement
                \begin{equation*}
                    \begin{aligned}
                        \bb E(Y) &= \frac 1{0.8} = 1.25\\
                        \bb V(Y) &= \frac {0.2}{0.8^2} = 0.3125
                    \end{aligned}
                \end{equation*}
            \end{enumerate}
        \end{enumerate}
    \end{td-sol}
}{}

\begin{td-exo}[] % 3
    On tire deux cartes dans un jeu de 52 cartes. On considère la
    variable aléatoire \(X = \left(X_1, X_2\right)\) où \(X_1\) donne le
    nombre de cartes rouges tirées et \(X_2\) le nombre de cartes noires.
    \begin{enumerate}
        \item Quelles valeurs peut prendre la variable aléatoire \(X\)?

        \item Déterminer la loi de \(X\)
    \end{enumerate}
\end{td-exo}
% ----- Solutions exo 3
\iftoggle{showsolutions}{
    \begin{td-sol}[]\, % 3
        \begin{enumerate}
            \item La variable aléatoire \(X\) prend les valeurs
            \begin{equation*}
                X \in \left\{(0,2), (1,1), (2,0)\right\}
            \end{equation*}
            car on tire deux cartes et qu'il n'y a que deux couleurs.

            \item La loi de \(X\) est donnée par
            \begin{equation*}
                \begin{aligned}
                    \bb P_X(0,2) &= \frac {26}{52} * \frac {25}{51}\\
                    \bb P_X(1,1) &= \frac {26}{52} * \frac {26}{51} + \frac {26}{52} * \frac {26}{51}\\
                    \bb P_X(2,0) &= \frac {26}{52} * \frac {25}{51}
                \end{aligned}
            \end{equation*}
        \end{enumerate}
    \end{td-sol}
}{}