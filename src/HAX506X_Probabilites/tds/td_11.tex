\begin{td-exo}[]\,
    \begin{enumerate}
        \item Rappeler les quatres modes de convergence vus en cours,
        leurs implications, et réciproques (sous certaines hypothèses)
        éventuelles. On se restreindra au cas de variables aléatoires
        réelles.

        \item Enoncer le théorème central-limite. On écrira la convergence
        en loi qui apparaît dans le théorème avec la définition puis avec
        la caractérisation via les fonctions de répartition.
    \end{enumerate}
\end{td-exo}

\iftoggle{showsolutions}{
    \begin{td-sol}\,
        \begin{enumerate}
            \item Pour \({(X_n)}_{n\geq 1}\) une suite de variables aléatoires réelles,
            et \(X\) une variable aléatoire réelle, on a
            \begin{itemize}
                \item Convergence presque sure:
                \begin{equation*}
                    X_n\cvpsn X,\quad \bb P(\{X_n\cvn X\}) = \bb P(\{\omega\in\Omega,X_n(\omega)\cv X(\omega)\})=1
                \end{equation*}

                \item Convergence en probabilité:
                \begin{equation*}
                    X_n\cvpn X,\quad \forall \varepsilon>0,\quad \bb P(\{|X_n-X|>\varepsilon\})\cvn 0
                \end{equation*}

                \item Convergence dans \(\mathcal L^p\):
                \begin{equation*}
                    X_n\cvlpn X,\quad \bb E[{|X_n-X|}^p]\cvn 0
                \end{equation*}

                \item Convergence en loi:
                \begin{equation*}
                    X_n\cvlawn X,\quad \forall h\in C_b^0(\bb R,\bb R),\quad \bb E[h(X_n)]\cvn \bb E[h(X)]
                \end{equation*}
            \end{itemize}

            \usetikzlibrary{arrows}
\begin{center}
    \begin{tikzpicture}[
        implies/.style={double equal sign distance, -implies},
        every node/.style={align=center}
    ]
        % Define the circle centers with labels
        \node (c1) at (-3, 2) {convergence\\presque sûre};
        \node (c2) at (3, 2) {convergence\\sur $\mathcal{L}^p$};
        \node (c3) at (0, -2.5) {convergence\\en probabilité};
        
        % Draw bigger circles around the centers
        \draw (c1) circle (1.5);
        \draw (c2) circle (1.5);
        \draw (c3) circle (1.5);
        
        % Red implication arrows (double lined) from top to bottom - now straight
        \draw[-implies, double equal sign distance, red] (-2, 0.8) -- (-0.8, -1.2);
        \draw[-implies, double equal sign distance, red] (2, 0.8) -- (0.8, -1.2);
        
        % Blue curved arrows between sets
        \draw[<-, blue] (-1.8, 1) to[bend right=20] node[above] {extraction} (1.8, 1);
        
        % Top extraction arrow now lower, mirroring the domination arrow
        \draw[<-, blue] (1.8, 3) to[bend right=20] node[above] {domination} (-1.8, 3);
        
        % Bottom arrows now start from circle sides
        \draw[->, blue] (-1.5, -2.5) to[bend left=40] node[left] {extraction} (-3, 0.5);
        
        \draw[->, blue] (1.5, -2.5) to[bend right=40] node[right] {hypothèse\\de moment} (3, 0.5);
    \end{tikzpicture}
\end{center}

            \item Soit \({(X_n)}_{n\geq 1}\) une suite de variables aléatoires
            \iid{} dans \(\mathcal L^2(\Omega,\mathcal F,\bb P)\) (de variance
            finie non nulle). Alors
            \begin{equation*}
                \begin{aligned}
                    &S_n = X_1 + \cdots + X_n,\\
                    &\frac{S_n-n\bb E[X_1]}{\sqrt{n\var(X_1)}}\cvlawn \mathcal N(0,1)\\
                    \iff& \forall x\in \bb R\\
                    &\bb P\left(\frac{S_n-n\bb E[X_1]}{\sqrt{n\var(X_1)}}\leq x\right)\cvn \Phi(x) = \int_{-\infty}^x\frac{e^{-\frac{t^2}{2}}}{\sqrt{2\pi}}\der t
                \end{aligned}
            \end{equation*}
        \end{enumerate}
    \end{td-sol}
}{}

\begin{td-exo}[]\,
    Pour tout entier \(n\geq 1\) on définit la fonction \(f_n\) sur \(\bb R\)
    par
    \begin{equation*}
        f_n(x) = \frac{n}{\pi\left(1+n^2 x^2\right)},\quad x\in\bb R
    \end{equation*}
    \begin{enumerate}
        \item Montrer que \(f_n\) est une densité pour tout entier \(n\geq 1\).

        \item Soit \({(X_n)}_{n\geq 1}\) une suite de variables aléatoires telle
        que \(X_n\) admet une densité \(f_n\). Les variables aléatoires \(X_n\)
        admettent-elles des moments?

        \item Etudier la convergence en loi, puis la convergence en probabilité
        de la suite \({(X_n)}_{n\geq 1}\).
    \end{enumerate}
\end{td-exo}

\iftoggle{showsolutions}{
    \begin{td-sol}\,
        \begin{enumerate}
            \item Soit \(n\geq 1\). La fonction \(f_n\) est
            mesurable et positive. De plus, on a
            \begin{equation*}
                \begin{aligned}
                    \int_{\bb R}f_n(x)\der x 
                    &= \int_{-\infty}^{+\infty}\frac{n}{\pi\left(1+n^2 x^2\right)}\der x\\
                    &= \frac{1}{\pi}\int_{-\infty}^{+\infty}\frac{n}{1+n^2 x^2}\der x\\
                    &= \frac{1}{\pi}\int_{-\infty}^{+\infty}\frac{1}{1+t^2}\der t\\
                    &= \frac{1}{\pi}{\left[\arctan(t)\right]}_{-\infty}^{+\infty}\\
                    &= \frac{1}{\pi}\left(\frac{\pi}{2}-\left(-\frac{\pi}{2}\right)\right)\\
                    &= 1
                \end{aligned}
            \end{equation*}
            Donc \(f_n\) est une densité.

            \item Soit \(n\geq 1\). On a l'équivalent
            \begin{equation*}
                x f_n(x) \underset{x\to\pm\infty}{\sim} \frac{1}{\pi n x}
            \end{equation*}
            Or cette dernière quantité n'est pas intégrable (Riemann) 
            donc \(x f_n(x)\) n'est pas intégrable, donc
            \(X_n\) n'admet pas d'espérance. Par conséquent, \(X_n\)
            n'admet aucun moment d'ordre \(p\geq 1\).

            \item Soit \(n\geq 1\). On a
            \begin{equation*}
                \begin{aligned}
                    F_n(x)
                    &= \bb P(X_n\leq x)\\
                    &= \int_{-\infty}^x f_n(t)\der t\\
                    &= \int_{-\infty}^x\frac{n}{\pi\left(1+n^2 t^2\right)}\der t\\
                    &= \frac{1}{\pi}{\left[\arctan(nt)\right]}_{-\infty}^x\\
                    &= \frac{\arctan(nx)}{\pi}+\frac{1}{2}
                \end{aligned}
            \end{equation*}
            On a
            \begin{equation*}
                \frac{\arctan(nx)}{\pi}+\frac{1}{2} \cvn
                \begin{cases}
                    1,&\text{ si }x>0\\
                    \frac{1}{2},&\text{ si }x=0\\
                    0,&\text{ si }x<0
                \end{cases}\quad =F(x)
            \end{equation*}
            \begin{remark}
                Si \(Y\sim \delta_0\), alors
                \begin{equation*}
                    F_Y(x) = \bb P(Y\leq x) = \begin{cases}
                        0,&\text{ si }x<0\\
                        1,&\text{ si }x\geq 0
                    \end{cases}
                \end{equation*}
            \end{remark}
            Ainsi, on a
            \begin{equation*}
                F_n(x)\cvn F(x)
            \end{equation*}
            pour tout point de continuité \(x\) de \(F\).
            Donc \(X_n\cvlawn X\), où \(X\sim \delta_0\).

            On a montré que \(X_n\cvlawn 0\). Comme \({(X_n)}_{n\geq 1}\)
            converge en loi vers une constante, elle converge en probabilité vers
            0. Alors
            \begin{equation*}
                X_n\cvpn 0
            \end{equation*}
        \end{enumerate}
    \end{td-sol}
}{}

\begin{td-exo}[]\,
    On considère deux suites de variables aléatoires \({(X_n)}_{n\geq 1}\)
    et \({(Y_n)}_{n\geq 1}\) définies sur le même espace probabilisé
    et qui convergent en loi respectivement vers les variables aléatoires
    \(X\) et \(Y\). 
\end{td-exo}

\iftoggle{showsolutions}{
    \begin{td-sol}\,
        fill
    \end{td-sol}
}{}


\begin{td-exo}[]\,
    Soit \(X_1, X_2,\ldots\) une suite \iid{} de variables aléatoires
    telle que
    \begin{equation*}
        \bb P(X_1=1) = \bb P(X_1=-1) = \frac{1}{2}
    \end{equation*}
    On pose \(S_n= X_1+\cdots X_n\). En utilisant le théorème central
    limite, déterminer la limite de \(\bb P(S_n\geq 0)\) quand \(n\to\infty\).
\end{td-exo}

\iftoggle{showsolutions}{
    \begin{td-sol}\,
        On a
        \begin{equation*}
            \begin{aligned}
                \bb E[X_1] = -1\times\frac12 + 1\times\frac12 = 0\\
                \var(X_1) = \bb E[X_1^2] = {\left(-1\right)}^2\times\frac12+1^2\times\frac12=1
            \end{aligned}
        \end{equation*}
        Les \(X_n\) sont \iid{} de variance finie non nulle, le théorème
        central limite donne alors
        \begin{equation*}
            \frac{S_n - n\bb E[X_1]}{\sqrt{n\var(X_1)}} 
            = \frac{S_n}{\sqrt{n}}\cvlawn N\sim \mathcal N(0,1)
        \end{equation*}
        On a donc, pour tout \(x\in\bb R\),
        \begin{equation*}
            \bb P\left(\frac{S_n}{\sqrt{n}}\leq x\right) \cvn \bb P(N\leq x) 
            = \int_{-\infty}^x \frac{e^{-\frac{t^2}{2}}}{\sqrt{2\pi}}\der t
        \end{equation*}
        En particulier, en \(x=0\), on a
        \begin{equation*}
            \bb P(S_n\leq 0) = \bb P\left(\frac{S_n}{\sqrt{n}} \leq 0\right)
            \cvn \bb P(N\leq 0) = \frac12
        \end{equation*}
        On en déduit
        \begin{equation*}
            \begin{aligned}
                \bb P(S_n\geq 0) 
                &= \bb P(S_n>0) + \bb P(S_n = 0)\\
                &= 1 - \underbrace{\bb P(S_n\leq 0)}_{\cv \frac12} + \underbrace{\bb P(S_n = 0)}_{\cv \begin{cases}
                    =0,&\text{ si }n\text{ impair}\\
                    \sim \frac{1}{\sqrt{2\pi n}},&\text{ si }n\text{ pair}
                \end{cases}}\\
                &= 0
            \end{aligned}
        \end{equation*}
    \end{td-sol}
}{}