\begin{td-exo}[]
    Soit \({(X_n)}_{n \geq 1}\) une suite de variables aléatoires \iid{}
    de loi \(\mathcal N(0,1)\). Montrer que la suite de terme général
    \begin{equation*}
        Y_n = \frac{1}{n} \sum_{k=1}^n X_k e^{X_k}
    \end{equation*}
    converge presque sûrement lorsque \(n\) tend vers l'infini vers
    une limite que l'on précisera.
\end{td-exo}

\iftoggle{showsolutions}{
    \begin{td-sol}[]
        Les variables aléatoires \(Z_k = X_k e^{X_k}\) sont
        \iid{} car les \(X_k\) le sont.\\
        Par ailleurs, les \(Z_k\) sont intégrables car
        \begin{equation*}
            x\mapsto |x| e^{|x|} \frac{e^{-\frac{x^2}{2}}}{\sqrt{2\pi}}
        \end{equation*}
        est intégrable en \(\pm \infty\). D'après la loi
        des grands nombres, on a 
        \begin{equation*}
            Y_n = \frac{1}{n} \sum_{k=1}^n Z_k \cvpsn \bb E[Z_1] 
        \end{equation*}
        et
        \begin{equation*}
            \begin{aligned}
                \bb E[Z_1] 
                &= \bb E[X_1 e^{X_1}]\\
                &= \int_{-\infty}^{+\infty} x e^x \frac{e^{-\frac{x^2}{2}}}{\sqrt{2\pi}} \der x\\
                &=\int_{-\infty}^{+\infty} \frac{x}{\sqrt{2\pi}} e^{-\frac{1}{2}{\left(x-1\right)}^2+\frac{1}{2}} \der x\\
                &= \sqrt{e} \int_{-\infty}^{+\infty} \frac{xe^{-\frac{1}{2}{\left(x-1\right)}^2}}{\sqrt{2\pi}} \der x\\
                \smol{\(t=x-1\)}&= \sqrt{e} \int_{-\infty}^{+\infty} \left(t+1\right) \frac{e^{-\frac{t^2}{2}}}{\sqrt{2\pi}} \der t\\
                &= \sqrt{e} \left(\color{red}\underbrace{\color{black}\int_{-\infty}^{+\infty} t \frac{e^{-\frac{t^2}{2}}}{\sqrt{2\pi}} \der t}_{\bb E[X_1]=0}\color{black} + \color{blue}\underbrace{\color{black}\int_{-\infty}^{+\infty} \frac{e^{-\frac{t^2}{2}}}{\sqrt{2\pi}} \der t}_{1}\color{black}\right)\\
                &= \sqrt{e}.
            \end{aligned}
        \end{equation*}
    \end{td-sol}
}{}

\begin{td-exo}[]
    On considère une suite de lancers d'un dé équilibré
    et on désigne par \(X_k\) le résultat du \(k\)-ième lancer.
    \begin{enumerate}
        \item On note \(Y_n\) la variable aléatoire donnant le
        plus grand résultat observé au cours des \(n\) premiers lancers.
        Etudier la convergence de la suite \({(Y_n)}_{n \geq 1}\).

        \item On note \(N_n\) la variable aléatoire donnant le
        nombre de 6 obtenus lors des \(n\) premiers lancers.
        Etudier la convergence de la suite \({(N_n/n)}_{n \geq 1}\).
    \end{enumerate}
\end{td-exo}

\iftoggle{showsolutions}{
    \begin{td-sol}[]\,
        \begin{enumerate}
            \item On a \(Y_n\) une suite croissante majorée par 6.
            Donc,
            \begin{equation*}
                \begin{aligned}
                    \bb P(Y_n \neq 6)
                    &= \bb P(\forall k \in \llbracket 1,n \rrbracket, X_k \neq 6)\\
                    &= {\left(\frac{5}{6}\right)}^n
                \end{aligned}
            \end{equation*}
            Alors,
            \begin{equation*}
                \begin{aligned}
                    \bb P(Y_n\cvn 6)
                    &= \bb P(\exists n_0 \in \bb N, Y_{n_0} = 6)\\
                    &= 1 - \bb P(\forall n \in \bb N, Y_n \neq 6)\\
                    &= 1 - \lim_{n\to\infty} \bb P(Y_n \neq 6)\\
                    &= 1.
                \end{aligned}
            \end{equation*}
            car \((Y_n)\) est a valeurs dans \(\{1,2,3,4,5,6\}\) et croissante.

            \item On a
            \begin{equation*}
                N_n = \sum_{k=1}^n \one_{X_k=6} \sim \mathcal B(n,1/6)
            \end{equation*}
            Les variables aléatoires \(\one_{X_k=6}\) sont \iid{} et
            intégrables (car bornées) donc la loi forte des grands nombres
            assure que
            \begin{equation*}
                \frac{N_n}{n} = \frac{1}{n} \sum_{k=1}^n \one_{X_k=6} \cvpsn \bb E[\one_{X_1=6}] = \frac{1}{6}.
            \end{equation*}
        \end{enumerate}
    \end{td-sol}
}{}

\begin{td-exo}[]
    On suppose que le sexe d'un nouveau-né est équiréparti entre fille et
    garçon. Un pays propose la politique de natalité suivante:
    tous les couples ont des enfants jusqu'à obtenir une fille.
    \begin{enumerate}
        \item Soit \(X\) le nombre d'enfants d'un couple pris au hasard
        dans la population. Donner la loi de la variable aléatoire \(X\)
        et son espérance.

        \item On considère une génération en âge de procréer constituée
        de \(n\) couples. On note \(X_1,\ldots,X_n\) le nombre d'enfants
        respectifs de chaque couple. On désigne par \(P_n\) la variable
        aléatoire donnant la proportion de filles issues de cette génération.
        Exprimer \(P_n\) en fonction de \(X_1,\ldots,X_n\), puis déterminer
        la limite de \(P_n\) quand \(n\) tend vers l'infini. La politique
        de natalité du pays a-t-elle un effet?
    \end{enumerate}
\end{td-exo}

\iftoggle{showsolutions}{
    \begin{td-sol}[]\,
        \begin{enumerate}
            \item On répète une expérience de Bernoulli de paramètre \(1/2\)
            de manière indépendante jusqu'à obtenir un succès. Le nombre
            d'enfants \(X\) suit alors une loi géométrique \(\mathcal G\left(\frac{1}{2}\right)\).
            \begin{itemize}[\ptr{}]
                \item \(X\) est à valeurs dans \(\bb N^*\),
                \item \(\bb P(X=k) = {\left(\frac{1}{2}\right)}^{k-1} \frac{1}{2} = \frac{1}{2^k}\),
                \item \(\bb E[X] = \sum_{k=1}^{+\infty} k \frac{1}{2^k} = 2\).
            \end{itemize}
            
            \item On a
            \begin{equation*}
                \begin{aligned}
                    P_n
                    &= \frac{\text{nombre de filles}}{\text{nombre total d'enfants}}\\
                    &= \frac{n}{X_1+\cdots+X_n}\\
                \end{aligned}
            \end{equation*}
            Les \(X_k\) sont \iid{} et intégrables donc la loi forte des
            grands nombres assure que
            \begin{equation*}
                \frac{X_1+\cdots+X_n}{n} \cvpsn \bb E[X_1] = 2
            \end{equation*}
            et donc
            \begin{equation*}
                P_n \cvpsn \frac{1}{2}.
            \end{equation*}
        \end{enumerate}
    \end{td-sol}
}{}

\begin{td-exo}[Pour aller plus loin]\,\\
    On considère une suite \iid{} \({(X_n)}_{n \geq 1}\) de variables aléatoires
    réelles et on pose \(S_n = \sum_{k=1}^n X_k\).
    \begin{enumerate}
        \item En utilisant le théorème de Fubini, montrer que
        pour toute variable aléatoire positive \(Y\),
        \begin{equation*}
            \bb E[Y] = \int_0^{+\infty} \bb P(Y \geq t) \der t.
        \end{equation*}
        puis en déduire que 
        \begin{equation*}
            \bb E[|X_1|] \leq 1 + \sum_{n=1}^{+\infty} \bb P(|X_n| \geq n).
        \end{equation*}

        \item En déduire que si \(X_1\) n'est pas intégrable,
        alors la suite \({(S_n/n)}_{n \geq 1}\) diverge presque sûrement.
    \end{enumerate}
\end{td-exo}

\iftoggle{showsolutions}{
    \begin{td-sol}[]\,
        \begin{enumerate}
            \item On a
            \begin{equation*}
                \begin{aligned}
                    \int_0^{+\infty} \bb P(Y \geq t) \der t
                    &= \int_0^{+\infty} \left(\int_{\Omega} \one_{Y(\omega) \geq t} \bb P(\der \omega)\right) \der t\\
                    \smol{Fubini}&= \int_{\Omega} \left(\int_0^{+\infty} \one_{Y(\omega) \geq t} \der t\right) \der \bb P(\omega)\\
                    &= \int_{\Omega} \left(\int_0^{Y(\omega)} 1\der t\right) \der \bb P(\omega)\\
                    &= \int_{\Omega} Y(\omega) \der \bb P(\omega)\\
                    &= \bb E[Y].
                \end{aligned}
            \end{equation*}

            On en déduit
            \begin{equation*}
                \begin{aligned}
                    \bb E[|X_1|]
                    &= \int_0^{+\infty} \bb P(|X_1| \geq t) \der t\\
                    &= \int_0^{+\infty} \sum_{n=0}^{+\infty} \bb P(|X_1| \geq t)\one_{n \leq t < n+1} \der t\\
                    &\leq \int_0^{+\infty} \sum_{n=0}^{+\infty} \bb P(|X_1| \geq n) \one_{n \leq t < n+1} \der t\\
                    \smol{TCM/Fubini}&= \sum_{n=0}^{+\infty} \bb P(|X_1| \geq n) \int_{0}^{\infty} \one_{n \leq t < n+1} \der t\\
                    &= \sum_{n=0}^{+\infty} \bb P(|X_1| \geq n)\\
                    &\leq 1 + \sum_{n=1}^{+\infty} \bb P(|X_n| \geq n).
                \end{aligned}
            \end{equation*}
            en résumé, on a montré que
            \begin{equation*}
                \bb E[|X_1|] \leq 1 + \sum_{n=1}^{+\infty} \bb P(|X_n| \geq n).
            \end{equation*}
        \end{enumerate}
    \end{td-sol}
}{}


% template:
% \begin{td-exo}
%     
% \end{td-exo}

% \iftoggle{showsolutions}{
%     \begin{td-sol}[]\,
% 
%     \end{td-sol}
% }{}