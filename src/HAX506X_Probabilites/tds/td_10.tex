\begin{td-exo}[]\,
    \begin{enumerate}
        \item Rappeler la définition de la convergence en loi,
        puis sa caractérisation via les fonctions caractéristiques
        et les fonctions de répartition (dans le cas de variables
        aléatoires réelles).

        \item On suppose que \(X_n\cvlawn X\) avec \((X_n)\) et \(X\)
        des variables aléatoires réelles. Si \(f;\bb R\to\bb R\) est
        une fonction continue, a-t-on
        \begin{equation*}
            f(X_n)\cvlawn f(X)\quad\text{?}
        \end{equation*}
    \end{enumerate}
\end{td-exo}

\iftoggle{showsolutions}{
    \begin{td-sol}\,
        \begin{enumerate}
            \item Soit \({(X_n)}_{n\geq 1}\) une suite de variables
            aléatoires réelles et \(X\) une variable aléatoire réelle.
            On a
            \begin{equation*}
                X_n \cvlawn X
            \end{equation*}
            si et seulement si
            \begin{itemize}[\ptr{}]
                \item On a
                \begin{equation*}
                    \forall h\in C_b^0(\bb R,\bb R),\quad \bb E[h(X_n)]\cvn \bb E[h(X)]
                \end{equation*}

                \item On a
                \begin{equation*}
                    \varphi_{X_n}(t)\cvn \varphi_X(t),\quad\forall t\in\bb R
                \end{equation*}

                \item On a
                \begin{equation*}
                    F_{X_n}(t)\cvn F_X(t),\quad \forall \text{ point de continuité de }F_X
                \end{equation*}
            \end{itemize}

            \item Soit \(h: \bb R\to\bb R\) continue bornée. Montrons que
            \begin{equation*}
                \bb E\ff{h\color{verdant}\left(\color{black}f\left(X_n\right)\color{verdant}\right)\color{black}} 
                \cvn \bb E\ff{h\color{verdant}\left(\color{black}f\left(x\right)\color{verdant}\right)\color{black}}
            \end{equation*}
            Comme \(X_n\cvlawn X\) et \(h\circ f\) est continue bornée, on a
            \begin{equation*}
                \bb E\ff{h\left(f\color{verdant}\left(\color{black}X_n\color{verdant}\right)\color{black}\right)}
                \cvn \bb E\ff{h\left(f\color{verdant}\left(\color{black} X\color{verdant}\right)\color{black}\right)}
            \end{equation*}
        \end{enumerate}
    \end{td-sol}
}{}

\begin{td-exo}[]
    Soit \({(X_n)}_{n\geq 1}\) une suite de variables aléatoires telle que
    pour tout \(n\geq 1\), \(X_n\) suit une loi uniforme sur l'ensemble
    \(\{k/n,1\leq k\leq n\}\). Montrer de trois manières différentes que
    \(X_n\) converge en loi vers une loi uniforme sur \(\ff{0,1}\).
\end{td-exo}

\iftoggle{showsolutions}{
    \begin{td-sol}\,
        Soit \({(X_n)}_{n\geq 1}\) une suite de variables aléatoires telle 
        que pour tout \(n\geq 1\), \(X_n\) suit une loi uniforme sur 
        l'ensemble \(\{k/n,1\leq k\leq n\}\).\\
        On montre par différentes méthodes:
        \begin{itemize}[\ptr{}]
            \item Première méthode: Définition\\
            Soit \(h\in C_b^0(\bb R)\), on a
            \begin{equation*}
                \begin{aligned}
                    \bb E\ff{h(X_n)}
                    &= \sum_{k=0}^{n}h(\frac{k}{n})\\
                    &\cvn \int_{0}^{1} h(x)\der x\\
                    &= \bb E\ff{h(X)}
                \end{aligned}
            \end{equation*}

            \item Deuxième méthode: Fonction caractéristique\\
            On a
            \begin{equation*}
                \varphi_{X_n}(t) = \sum_{k=1}^{n} {e^{\left(\frac{it}{n}\right)}}^n\times\frac{1}{n}
            \end{equation*}
            On calcule pour \(t\neq 0\)
            \begin{equation*}
                \begin{aligned}
                    &\frac{e^{\frac{it}{n}}}{n}\times \frac{1-e^{it}}{1-e^{\frac{it}{n}}}\\
                    \sim& \frac{1}{n\times\left(-\frac{it}{n}\right)}\times \left(1-e^{it}\right)
                \end{aligned}
            \end{equation*}
            ce qui correspond bien à
            \begin{equation*}
                \varphi_U(t) = \frac{e^{it}-1}{it}
            \end{equation*}
            avec \(U\sim \mathcal U([0,1])\)

            Pour \(t=0\), on a
            \begin{equation*}
                \varphi(X_n)(0)=1 \cvn \varphi_U(0)
            \end{equation*}


            \item Troisième méthode: Fonction de répartition\\
            On a
            \begin{equation*}
                F_X(t) = \bb P(X\leq t) = \begin{cases}
                    0&\text{ si }t\leq0\\
                    t&\text{ si }t\in\oo{0,1}\\
                    1&\text{ si }t\geq 1
                \end{cases}
            \end{equation*}
            Comme \(F_X\) est continue, on veut montrer que
            \begin{equation*}
                \forall t\in\bb R,\quad F_{X_n}(t) \cvn F_X(t)
            \end{equation*}
            On procède par cas:
            \begin{itemize}[\ptr{}]
                \item Si \(t\leq 0\), on a
                \begin{equation*}
                    F_{X_n}(t) = 0\cvn 0=F_X(t)
                \end{equation*}

                \item Si \(t\geq 1\), on a
                \begin{equation*}
                    F_{X_n}(t) = 1\cvn 1=F_X(t)
                \end{equation*}

                \item Si \(t\in\oo{0,1}\), on a
                \begin{equation*}
                    \begin{aligned}
                        F_{X_n}(t)
                        &= \bb P(X_n\leq t)\\
                        &= \sum_{k=1}^n \frac{1}{n} \one_{\left\{\frac{k}{n}\leq t\right\}}\\
                        &=\frac{1}{n} \sum_{k=1}^n\one_{\left\{k\leq tn \right\}}\\
                    \end{aligned}
                \end{equation*}
                On conclut avec
                \begin{equation*}
                    \begin{aligned}
                        & tn-1< \lfloor tn\rfloor \leq tn\\
                        & t-\frac{1}{n}<\frac{\lfloor tn\rfloor}{n}\leq t
                    \end{aligned}
                \end{equation*}
                Donc
                \begin{equation*}
                    F_{X_n}(t) = \frac{\lfloor tn\rfloor}{n} \cvn t
                \end{equation*}
            \end{itemize}
        \end{itemize}
    \end{td-sol}
}{}

\begin{td-exo}[]\,
    Pour tout entier \(n\) on considère la variable aléatoire \(T_n\)
    de loi géométrique de paramètre \(\lambda/n\), avec \(\lambda>0\)
    fixé. Montrer que \(T_n/n\) converge en loi vers une limite à déterminer
\end{td-exo}

\iftoggle{showsolutions}{
    \begin{td-sol}\,
        On a \(T_n\sim \mathcal G\left(\frac{\lambda}{n}\right)\).
        Soit \(t\in\bb R\).
        \begin{equation*}
            \begin{aligned}
                \varphi_{\frac{T_n}{n}}(t)
                &=\bb E\ff{e^{it\frac{T_n}{n}}}\\
                &= \sum_{k=1}^\infty e^{it\frac{k}{n}} {\left(1-\frac{\lambda}{n}\right)}^{k-1}\frac{\lambda}{n}\\
                &= \frac{\lambda}{n} e^{i\frac{t}{n}}\sum_{k=1}^\infty e^{i\frac{t}{n}\left(k-1\right)}{\left(1-\frac{\lambda}{n}\right)}^{k-1}\\
                &= \frac{\lambda}{n}e^{i\frac{t}{n}} \sum_{k=0}^\infty {\left[e^{i\frac{t}{n}\left(1-\frac{\lambda}{n}\right)}\right]}^k\\
                &= \frac{\lambda e^{i\frac{t}{n}}}{n\left(1-e^{i\frac{t}{n}}\left(1-\frac{\lambda}{n}\right)\right)}
            \end{aligned}
        \end{equation*}
        où
        \begin{equation*}
            \begin{aligned}
                1-e^{i\frac{t}{n}}\left(1-\frac{\lambda}{n}\right)
                &= 1-\left(\color{orange} 1\color{black} + \color{verdant}\frac{it}{n}\color{black}+o\left(\frac{1}{n}\right)\right)\left(\color{verdant}1\color{black}\color{orange}-\frac{\lambda}{n}\color{black}\right)\\
                &= \color{verdant}-\frac{it}{n}\color{black}+\color{orange}\frac{\lambda}{n}\color{black}+o\left(\frac{1}{n}\right)
            \end{aligned}
        \end{equation*}
        et alors
        \begin{equation*}
            \varphi_{\frac{T_n}{n}}(t) \cvn \frac{\lambda}{\lambda - it}
        \end{equation*}
        Si \(X\sim \mathcal E(\lambda)\), alors
        \begin{equation*}
            \begin{aligned}
                \varphi_X(t) 
                &= \bb E\ff{e^{itX}} \\
                &= \int_{\bb R} e^{itx}\lambda e^{-\lambda x}\one_{\bb R_+}(x)\der x\\
                &= \lambda \int_0^\infty e^{\left(it-\lambda\right) x}\der x\\
                &= \left[\lambda\frac{e^{\left(it-\lambda\right)x}}{it-\lambda}\right]_0^\infty\\
                &= \frac{\lambda}{\lambda-it}
            \end{aligned}
        \end{equation*}
        Donc
        \begin{equation*}
            \frac{T_n}{n}\cvlawn \mathcal E(\lambda)
        \end{equation*}
    \end{td-sol}
}{}

\begin{td-exo}[]\,
    Soit \({(U_n)}_{n\geq 1}\) une suite de variables aléatoires indépendantes
    de loi uniforme sur \(\ff{0,1}\). On pose
    \begin{equation*}
        M_n = \max\left(U_1,\ldots,U_n\right)
        \quad\text{et}\quad
        X_n = n(1-M_n)
    \end{equation*}
    \begin{enumerate}
        \item Déterminer la fonction de répartition de \(M_n\),
        puis celle de \(X_n\).

        \iteù Etudier la convergence en loi de la suite \({(X_n)}_{n\geq 1}\).
    \end{enumerate}
\end{td-exo}

\iftoggle{showsolutions}{
    \begin{td-sol}\,
        A recuperer sur le Moodle
    \end{td-sol}
}{}