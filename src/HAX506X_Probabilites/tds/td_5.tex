\begin{td-exo}[] % 1
    Soit \(X\) une variable aléatoire de loi exponentielle
    de paramètre \(\lambda\). On pose
    \begin{equation*}
        Y = \lfloor X\rfloor + 1
    \end{equation*}
    où \(\lfloor x\rfloor\) désigne la partie entière
    d'un réel \(x\). Déterminer la loi de \(Y\).
\end{td-exo}
% ----- Solutions exo 1
\iftoggle{showsolutions}{
    \begin{td-sol}[]\, % 1
        \(\triangleright\) On peut remarquer que \(Y\) est
        une variable aléatoire discrète prenant à valeurs
        dans \(\bb N^*\).

        \(\triangleright\) Soit \(k \in \bb N^*\). On veut 
        déterminer \(\bb P(Y = k)\):
        \begin{equation*}
            \begin{aligned}
                \bb P(Y = k) 
                &= \bb P(\lfloor X\rfloor + 1 = k)\\
                &= \bb P(k - 1 \leq X < k)\\
                &= \int_{k-1}^k \lambda e^{-\lambda x} \der x\\
                &= \n{-e^{-\lambda x}}_{k-1}^k\\
                &= e^{-\lambda(k-1)} - e^{-\lambda k}\\
                &= e^{-\lambda (k-1)}(1 - e^{-\lambda})\\
                &= \color{verdant}{(e^{-\lambda})}^{k-1}(1 - e^{-\lambda})
            \end{aligned}
        \end{equation*}

        \(\triangleright\) On reconnait alors la loi géométrique de paramètre
        \(p = 1 - e^{-\lambda}\). Ainsi, \(Y \sim \mathcal G(p)\).
    \end{td-sol}
}{}

\begin{td-exo}[]\, % 2
    \begin{enumerate}
        \item Déterminer la fonction caractéristique de la loi
        de Bernoulli, de la loi binomiale et de la loi de
        Poisson. Rappeler également (sans calcul) la
        fonction caractéristique de la loi normale.

        \item Soit \(X\) et \(Y\) deux variables aléatoires
        indépendantes. Montrer que
        \begin{equation*}
            \varphi_{X+Y} = \varphi_X\varphi_Y
        \end{equation*}

        \item \,
        \begin{enumerate}
            \item On considère deux variables aléatoires 
            \(X_1 \sim \mathcal N(\mu,1,\sigma_1^2)\) et
            \(X_2 \sim \mathcal N(\mu,2,\sigma_2^2)\) indépendantes.
            Déterminer la loi de \(X_1 + X_2\).

            \item Montrer que la somme de \(n\) variables aléatoires
            indépendantes de loi de Bernoulli de paramètre \(p\)
            suit une loi binomiale de paramètres \(n\) et \(p\).

            \item Montrer que la somme de deux variables aléatoires
            indépendantes \(X\) et \(Y\) de loi de Poisson de paramètres
            respectifs \(\lambda\) et \(\mu\) suit une loi de Poisson
            de paramètre \(\lambda + \mu\).
        \end{enumerate}

    \end{enumerate}
\end{td-exo}
% ----- Solutions exo 2
\iftoggle{showsolutions}{
    \begin{td-sol}[]\, % 2
        \begin{enumerate}
            \item Les fonctions caractéristiques sont
            \begin{itemize}
                \item Pour \(X \sim \mathcal B(p)\),
                \begin{equation*}
                    \begin{aligned}
                        \bb P(X = 1) &= p\\
                        \bb P(X = 0) &= 1 - p\\
                        \varphi_X(t)
                        &= \bb E\ff{e^{itX}}\\
                        &= \int_{\Omega} e^{itX(\omega)} \bb P(\der \omega)\\
                        &= \sum_{k=0}^1 e^{itk} \bb P(X = k)\\
                        &= e^{it} \bb P(X = 1) + \bb P(X = 0)\\
                        &= e^{it}p + 1 - p
                    \end{aligned}
                \end{equation*}

                \item Pour \(X \sim \mathcal B(n,p)\),
                \begin{equation*}
                    \begin{aligned}
                        \bb P(X=k)
                        &= \binom{n}{k} p^k{(1-p)}^{n-k}\\
                        \varphi_X(t)
                        &= \bb E\ff{e^{itX}}\\
                        &= \int_{\Omega} e^{itX(\omega)} \bb P(\der \omega)\\
                        &= \sum_{k=0}^n e^{itk} \bb P(X = k)\\
                        &= \sum_{k=0}^n e^{itk} \binom{n}{k} p^k{(1-p)}^{n-k}\\
                        &= \sum_{k=0}^n \binom{n}{k} {(pe^{it})}^k{(1-p)}^{n-k}\\
                        &= {(pe^{it} + 1 - p)}^n
                    \end{aligned}
                \end{equation*}

                \item Pour \(X \sim \mathcal P(\lambda)\),
                \begin{equation*}
                    \begin{aligned}
                        \bb P(X=k)
                        &= \frac{e^{-\lambda}\lambda^k}{k!}\\
                        \varphi_X(t)
                        &= \bb E\ff{e^{itX}}\\
                        &= \sum_{k=0}^{\infty} e^{itk} \frac{e^{-\lambda}\lambda^k}{k!}\\
                        &= e^{-\lambda} \sum_{k=0}^{\infty} \frac{{(e^{it}\lambda)}^k}{k!}\\
                        &= e^{-\lambda} e^{e^{it}\lambda}\\
                        &= e^{\lambda(e^{it} - 1)}
                    \end{aligned}
                \end{equation*}

                \item Pour \(X \sim \mathcal N(\mu,\sigma^2)\),
                \begin{equation*}
                    \begin{aligned}
                        Y = \frac{X - \mu}{\sigma} \sim \mathcal N(0,1)\\
                        \varphi_X(t)
                        &= \bb E\ff{e^{itX}}\\
                        &= \bb E\ff{e^{it(\mu + \sigma Y)}}\\
                        &= e^{it\mu} \bb E\ff{e^{it\sigma Y}}\\
                        &= e^{it\mu} \varphi_Y(\sigma t)\\
                        &= e^{it\mu - \frac{1}{2}\sigma^2t^2}
                    \end{aligned}
                \end{equation*}
            \end{itemize}

            \item Soit \(t \in \bb R\),
            \begin{equation*}
                \begin{aligned}
                    \varphi_{X+Y}(t)
                    &= \bb E\ff{e^{it(X+Y)}}\\
                    &= \bb E\ff{e^{itX}e^{itY}}\\
                    {}_{\text{par indep de }X,Y}&= \bb E\ff{e^{itX}}\bb E\ff{e^{itY}}\\
                    &= \varphi_X(t)\varphi_Y(t)
                \end{aligned}
            \end{equation*}

            \item \,
            \begin{enumerate}
                \item Soit \(t \in \bb R\),
                \begin{equation*}
                    \begin{aligned}
                        \varphi_{X_1+X_2}(t)
                        &= \varphi_{X_1}(t)\varphi_{X_2}(t)\\
                        &= e^{it\mu_1 - \frac{1}{2}\sigma_1^2t^2}e^{it\mu_2 - \frac{1}{2}\sigma_2^2t^2}\\
                        &= e^{it(\mu_1 + \mu_2) - \frac{1}{2}(t^2(\sigma_1^2 + \sigma_2^2))}
                    \end{aligned}
                \end{equation*}
                Donc, \(X_1 + X_2 \sim \mathcal N(\mu_1 + \mu_2, \sigma_1^2 + \sigma_2^2)\).

                \item Soit \(X_1, \ldots, X_n\) des variables aléatoires
                indépendantes de loi de Bernoulli de paramètre \(p\).
                On a
                \begin{equation*}
                    \begin{aligned}
                        \varphi_{X_1 + \cdots + X_n}(t)
                        &= \varphi_{X_1}(t)\cdots\varphi_{X_n}(t)\\
                        &= (pe^{it} + 1 - p) \times \cdots \times (pe^{it} + 1 - p)\\
                        &= {(pe^{it} + 1 - p)}^n
                    \end{aligned}
                \end{equation*}
                Donc, \(X_1 + \cdots + X_n \sim \mathcal B(n,p)\).

                \item Soit \(X \sim \mathcal P(\lambda)\) et \(Y \sim \mathcal P(\mu)\).
                On a
                \begin{equation*}
                    \begin{aligned}
                        \varphi_{X+Y}(t)
                        &= \varphi_X(t)\varphi_Y(t)\\
                        &= e^{\lambda(e^{it} - 1)}e^{\mu(e^{it} - 1)}\\
                        &= e^{(\lambda + \mu)(e^{it} - 1)}
                    \end{aligned}
                \end{equation*}
                Donc, \(X + Y \sim \mathcal P(\lambda + \mu)\).
            \end{enumerate}
        \end{enumerate}
    \end{td-sol}
}{}

\begin{td-exo}[] % 3
    On considère deux variables aléatoires indépendantes
    \(X \sim \mathcal N(0,1)\) et \(Y \sim \mathcal N(0,\sigma^2)\)
    avec \(\sigma > 0\). 
    \begin{enumerate}
        \item On pose \(T = X^2\). Calculer \(\bb E\ff{h(t)}\) pour
        toute fonction \(h: \bb R \to \bb R\) mesurable bornée.
        En déduire la loi de \(T\).\\
        On pourra varifier qu'on retrouve bien le résultat de
        l'exercice 2 du TD 4.

        \item On pose \(U = \one_{\{X \geq 0\}} - \one_{\{X < 0\}}\).
        Déterminer la loi de \(U\).

        \item Montrer que \(Z=UY\) suit une loi normale de paramètres
        à déterminer.

        \item Déterminer la loi de \(Y+Z\).
    \end{enumerate}
\end{td-exo}
% ----- Solutions exo 3
\iftoggle{showsolutions}{
    \begin{td-sol}[]\, % 3
        \begin{enumerate}
            \item Soit \(h: \bb R \to \bb R\) une fonction mesurable
            bornée. On a
            \begin{equation*}
                \begin{aligned}
                    \bb E\ff{h(t)}
                    &= \bb E\ff{h(X^2)}\\
                    {}_{g(t)=t^2}&= \bb E\ff{(h\circ g)(X)}\\
                    % on applique la formule de transfert par rapport à la loi de X
                    &= \int_{\bb R} (h\circ g)(x) \frac{e^{-\frac{x^2}{2}}}{\sqrt{2\pi}} \der x\\
                    &= \int_{\bb R} h(x^2) \frac{e^{-\frac{x^2}{2}}}{\sqrt{2\pi}} \der x\\
                    &= 2 \int_{0}^{+\infty} h(x^2) \frac{e^{-\frac{x^2}{2}}}{\sqrt{2\pi}} \der x\\
                    {}_{t=x^2}&= 2 \int_{0}^{+\infty} h(t) \frac{e^{-\frac{t}{2}}}{\sqrt{2\pi}} \der \frac{t}{2\sqrt t}\\
                    &= \int_{\bb R} h(t) \frac{e^{-\frac{t}{2}}}{\sqrt{2\pi t}} \one_{\bb R_+}(t) \der t
                \end{aligned}
            \end{equation*}
            Donc \(T\) est une variable aléatoire à densité donnée par
            \begin{equation*}
                f_T(t) = \frac{e^{-\frac{t}{2}}}{\sqrt{2\pi t}} \one_{\bb R_+}(t)
            \end{equation*}

            \item On a
            \begin{equation*}
                U =
                \begin{cases}
                    1 & \text{si } X \geq 0\\
                    -1 & \text{si } X < 0
                \end{cases}
            \end{equation*}
            Donc, \(U\) est une variable aléatoire discrète prenant
            les valeurs \(-1\) et \(1\) avec
            \begin{equation*}
                \begin{aligned}
                    \bb P(U = 1)
                    &= \bb P(X \geq 0)\\
                    &= \frac{1}{2}\\
                    \bb P(U = -1)
                    &= \bb P(X < 0)\\
                    &= \frac{1}{2}
                \end{aligned}
            \end{equation*}
            Donc, \(U \sim \mathcal B\ff{\frac{1}{2}}\). Une
            variante de Bernoulli appelée \textit{Rademacher}.

            \item Montrons que \(Z \sim \mathcal N(0,\sigma^2)\).
            \begin{equation*}
                \begin{aligned}
                    \bb P(Z \leq x)
                    &= \bb P(UY \leq x)\\
                    &= \bb P(UY\leq x \mid U=1) \bb P(U=1) 
                    + \bb P(UY\leq x \mid U=-1) \bb P(U=-1)\\
                    {}_{\text{par indep de }U}&= \bb P(Y\leq x) \times \frac{1}{2} + \bb P(-Y\leq x) \times \frac{1}{2}\\
                    &= \frac{1}{2} \left(\bb P(Y\leq x) + \bb P(Y\geq -x)\right)\\
                    &= \frac{1}{2} \left(\bb P(Y\leq x) + \bb P(Y\leq x)\right)\\
                    &= \bb P(Y\leq x)
                \end{aligned}
            \end{equation*}
            Donc \(Z \sim \mathcal N(0,\sigma^2)\).

            \item \(Y\) et \(Z\) étant indépendantes, on a
            \begin{equation*}
                \begin{aligned}
                    \bb E\ff{e^{it(Y+Z)}}
                    &= \bb E\ff{e^{itY}e^{itZ}}\\
                    &= \bb E\ff{e^{itY}}\bb E\ff{e^{itZ}}\\
                    &= e^{-\frac{t^2}{2}}e^{-\frac{\sigma^2t^2}{2}}\\
                    &= e^{-\frac{(1+\sigma^2)t^2}{2}}
                \end{aligned}
            \end{equation*}
            Donc, \(Y + Z \sim \mathcal N(0,1+\sigma^2)\).
        \end{enumerate}
    \end{td-sol}
}{}
