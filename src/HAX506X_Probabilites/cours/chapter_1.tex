\subsection{Espaces probabilisés}\label{subsec:1}
%\addcontentsline{toc}{subsection}{\nameref{subsec:1}}


\subsubsection{Probabilité}\label{subsubsec:1}
\setcounter{subsection}{0}
%\addcontentsline{toc}{subsubsection}{\nameref{subsubsec:1}}
\begin{definition}
    Soit \((\Omega, \scr F)\) un espace mesurable. Une
    \defemph{mesure} sur \((\Omega, \scr F)\) est une application
    \begin{equation*}
        \begin{aligned}
            \mu\colon \scr F &\to \ff{0,+\infty}\\
            A &\mapsto \mu(A)
        \end{aligned}
    \end{equation*}
    qui vérifie les propriétés suivantes:
    \begin{enumerate}
        \item \(\mu(\emptyset) = 0\)
        \item \(\mu\) est \(\sigma\)-additive, c'est-à-dire que pour
        toute suite \({(A_n)}_{n\in\N}\) d'éléments 2 à 2
        disjoints de \(\scr F\), on a
        \begin{equation*}
            \mu\left(\bigcup_{n\in\N}A_n\right) = \sum_{n\in\N}\mu(A_n)
        \end{equation*}
    \end{enumerate}

    On dit alors que \((\Omega, \scr F,\mu)\) est un \defemph{espace
    mesuré}.

    Si de plus \(\mu(\Omega) = 1\), on dit que \((\Omega, \scr F,\mu)\)
    est un \defemph{espace probabilisé} et \(\mu\) est une \defemph{probabilité}.

    On notera alors \(\mu = \bb P\).
\end{definition}

\begin{remark}
    Comme \(\bb P(\Omega)=1\), une mesure de probabilité est une mesure
    dans \(\ff{0,1}\). Un événement \(A\) est dit \defemph{presque sûr}
    si \(\bb P(A) = 1\).
\end{remark}

\begin{exs}\,
    \begin{enumerate}
        \item Soit \((\Omega, \scr F)\) un espace mesurable et
        \(\omega\) un élément fixé dans \(\Omega\). La mesure
        (ou masse) de Dirac en \(\omega\) est la mesure définie
        pour tout \(A\in\scr F\) par
        \begin{equation*}
            \delta_\omega(A) = \begin{cases}
                1 & \text{si } \omega\in A\\
                0 & \text{sinon}
            \end{cases}= \one_{A}(\omega)
        \end{equation*}
        On vérifie facilement que c'est bien une probabilité.

        \item Sur le segment \(\ff{0,1}\) muni de sa tribu 
        borélienne, la mesure de Lebesgue est une probabilité.

        \item Si \((\Omega,\scr F,\mu)\) est un espace mesuré avec
        \(0<\mu(\Omega)<+\infty\), alors on obtient une probabilité
        en considérant la mesure
        \begin{equation*}
            \bb P = \frac{\mu(\cdot)}{\mu(\Omega)}
        \end{equation*}
    \end{enumerate}
\end{exs}

\begin{interp}
    Un espace probabilisé est donc un cas particulier d'espace mesuré
    pour lequel la masse totale de la mesure est égale à 1. En fait,
    le point de vue diffère de la théorie de l'intégration: dans
    le cadre de la théorie des probabilités, on cherche à fournir
    un modèle mathématique pour une ``expérience aléatoire''.

    \begin{itemize}
        \item L'ensemble \(\Omega\) est appelé \defemph{univers}:
        il représente l'ensemble de toutes les éventualiés possibles,
        toutes les déterminations du hasard dans l'expérience considérée.
        Les éléments \(\omega\) de \(\Omega\), parfois appelés
        \defemph{événements élémentaires}, correspondent donc aux issues
        possibles de l'expérience aléatoire.

        \item La tribu \(\scr F\) correspond à l'ensemble des
        \defemph{événements}:  ce sont les parties de \(\Omega\) dont
        on peut évaluer la probabilité. Il faut voir un événement
        \(A\) de \(\scr F\) comme un sous-ensemble de \(\Omega\) 
        contenant toutes les éventualités \(\omega\) pour
        lesquelles une certaine propriété est vérifiée.

        \item On associe à chaque événement \(A\in\scr F\) un réel
        \(\bb P(A) \in \ff{0,1}\) qui donne la plausibilité que
        le résultat de l'expérience soit dans \(A\).
    \end{itemize}
\end{interp}

\subsubsection{Exemples d'espaces probabilisés}\label{subsubsec:2}
\setcounter{subsection}{0}
%\addcontentsline{toc}{subsubsection}{\nameref{subsubsec:2}}
Suivent quelques exemples classiques d'espaces probabilisés.

\begin{exs}
    fin
\end{exs}

