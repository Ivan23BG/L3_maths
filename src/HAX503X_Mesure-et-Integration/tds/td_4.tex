\begin{td-exo}[]
    exo 6.
\end{td-exo}
% ----- Solutions exo 6
\iftoggle{showsolutions}{
    \begin{nntd-sol}[]\,
        \begin{enumerate}
            \item Rapide à faire. On veut appliquer TCD donc on vérifie les hypothèses:
            \begin{itemize}
                \item On a une convergence simple de la fraaction vers \(\frac1{k^2}\)

                \item On a domination de la valeur absolue par \(\frac{1}{k^2}\) qui est
                sommable.
            \end{itemize}
            On applique alors le théorème de convergence dominée pour conclure.

            \item On verifie encore les hypothèses:
            \begin{itemize}
                \item On a une convergence simple de la fraction vers 0.

                \item On a domination de la valeur absolue par \(\frac{1}{k^2}\) qui est
            \end{itemize}
            On applique alors le théorème de convergence dominée pour conclure.
        \end{enumerate}
    \end{nntd-sol}
}{}

\begin{td-exo}[]
    exo 9.
\end{td-exo}
% ----- Solutions exo 9
\iftoggle{showsolutions}{
    \begin{nntd-sol}[]\,
        % partie 1:
        % step 1: lister les outils utiles, ici croissances comparées pour
        % convergences simples
        % step 2: identifier les potentiels problèmes, ici 0 est exclu
        % donc le problème est uniquement en y->infini qui est un O(1/y^2)
        % qui est intégrable en y->infini
        % step 3: on majore la |.| par une fonction de R+*
        % qui est \one_{0,1}(y)+\one_{1,+\infty}(y)/y^2
        % qui est L1
        % step 4: on applique le théorème de convergence dominée
        % step 5: on conclut

        % partie 2:
        % step 1: lister les outils utiles, ici decomposition en elements simples
        % step 2: on multiplie et on identifie
        % step 3: on prolonge par continuité en x=1
        % step 4: on montre que ca vaut 2 f ( x )
        % step 5: on utilise le dl de ln pour observer le comportement en -1 (qui est 1/2)
        % step 6: etude aux bornes
        % setp 7: -ln(x) en 0, integrable par le critère de Bertrand
        % step 8: 1/(x^2 ln(x)) en +infini, integrable par le critère de Bertrand
        % step 9: on pose I l'intégrale de F et J l'intégrale^2 de F
        % step 10: on utilise Fubini-Tonelli pour intégrer comme on veut (car F est positive)
        % step 11: on retrouve l'intégrale de F dans l'autre sens qui vaut pi/2 -> pi^2/2
        % step 12: on conclut

        % partie 3: I = int ln(x)/(x^2-1) = int_(0,1) ln(x)/(x^2-1) + int_(1,+\infty) ln(x)/(x^2-1)
        % step 1: on a la première partie qui colle, on s'occupe de la seconde
        % step 2: on pose u = 1/x -> dx = -du/u^2
        % step 3: on a I = 2 int_(0,1) ln(x)/(x^2-1)
        % step 4: developpement en serie entiere avec -1/(x^2-1) = 1 + x^2 + x^4 + ...
        % step 5: on applique fubini tonelli a la mesure produit lebesgue et comptage
        % pour inverser les integrales
        % step 6: IPP pour montrer que l'intégrale est égale à pi^2/8
        % step 7: on separe en somme de pairs et impairs
        
    \end{nntd-sol}
}{}

