\subsection*{Exercices en TD}\label{sec:tutorial_3}
\addcontentsline{toc}{subsection}{\nameref{sec:tutorial_3}}

\begin{td-exo}[Groupes] % 1
    Ces choses-ci sont-elles des groupes?
    \begin{itemize}
        \item \(\left(2\Z,+\right)\)
        \item \(\left(2\Z,\times\right)\)
        \item L'ensemble des fonctions de \(\ff{0,1}\) dans \(\R\),
        muni de l'addition des fonctions.
        \item L'ensemble des fonctions continuesde \(\ff{0,1}\) dans \(\R\),
        muni de l'addition des fonctions.
        \item L'ensemble des matrices \(n\times n\) inversibles et
        à coefficients entiers, muni du produit matriciel.
        \item L'ensemble des parties d'un ensemble \(E\), muni de
        l'union.
        \item L'ensemble des permutations \(\sigma\in\mathfrak{S}_6\)
        telles que \(\sigma^2=\id\), muni de la composition.
    \end{itemize}
\end{td-exo}
% ----- Solutions exo 1
\iftoggle{showsolutions}{
    \begin{td-sol}[]\, % 1
        \begin{itemize}
            \item \(\left(2\Z,+\right)\) est un groupe car:
            \begin{itemize}
                \item Il est non vide: \(0\in 2\Z\).
                \item Il est stable par l'addition: si \(a,b\in 2\Z\), alors \(a+b\in 2\Z\).
                \item Il est stable par l'opposé: si \(a\in 2\Z\), alors \(-a\in 2\Z\).
            \end{itemize}
            \item \(\left(2\Z,\times\right)\) n'est pas un groupe car \(0\notin 2\Z\).
            \item L'ensemble des fonctions de \(\ff{0,1}\) dans \(\R\),
            muni de l'addition des fonctions, est un groupe.
            \item L'ensemble des fonctions continues de \(\ff{0,1}\) dans \(\R\),
            muni de l'addition des fonctions, est un groupe.
            \item L'ensemble des matrices \(n\times n\) inversibles et
            à coefficients entiers, muni du produit matriciel, n'est pas un groupe
            car l'inverse d'une matrice à coefficients entiers n'est pas forcément
            à coefficients entiers.
            \item L'ensemble des parties d'un ensemble \(E\), muni de
            l'union, n'est pas un groupe car il n'est pas stable par l'opposé.
            \item L'ensemble des permutations \(\sigma\in\mathfrak{S}_6\)
            telles que \(\sigma^2=\id\), muni de la composition, n'est un groupe
            car il n'est pas commutatif.
        \end{itemize}
    \end{td-sol}
}{}

\begin{td-exo}[Tarte à la crème] % 2
    Soit \(G\) un groupe tel que tout \(x\in G\) vérifie \(x^2=e\).
    Démontrer que \(G\) est abélien.
\end{td-exo}
% ----- Solutions exo 2
\iftoggle{showsolutions}{
    \begin{td-sol}[]\, % 2
        On veut montrer que pour tout \(x,y\in G\), \(xy=yx\).
        Soient \(x,y\in G\). Alors:
        \begin{equation*}
            \begin{aligned}
                {(xy)}^2=e
                &\implies xyxy=e\\
                &\implies xyxyy = y\\
                &\implies xyx = y\\
                &\implies xyxx = yx\\
                &\implies xy=yx
            \end{aligned}
        \end{equation*}
        Comme \({(xy)}^2=e\), on a bien \(xy=yx\).
    \end{td-sol}
}{}

\begin{td-exo}[Petits groupes] % 3
    Déterminer toutes les tables de multiplication possibles pour des
    groupes d'ordre \(\le 5\). (On se gardera d'utiliser le théorème de
    Lagrange.)
    \begin{itemize}
        \item Vous devez trouver (au nom des éléments près) un seul groupe
        d'ordre 1, un seul d'ordre 2, un seul d'ordre 3, deux de l'ordre 4 et
        un seul d'ordre 5.
        \item Remarquez que tous ces groupes sont abéliens.
    \end{itemize}
\end{td-exo}
% ----- Solutions exo 3
\iftoggle{showsolutions}{
    \begin{td-sol}[]\, % 3
        On va dresser les tables de multiplication pour les groupes d'ordre 1, 2, 3, 4 et 5.
        On peut retrouver ces tables à partir des ``règles de sudoku''.
        \begin{itemize}
            \item Pour l'ordre 1, il n'y a qu'un groupe: \(\{e\}\), de table:
            \begin{equation*}
                \begin{array}{c|c} % chktex 44
                    \times & \cellcolor{blue!20}e \\
                    \hline % chktex 44
                    \cellcolor{blue!20}e & \cellcolor{blue!20}e \\
                \end{array}
            \end{equation*}
            \item Pour l'ordre 2, il n'y a qu'un groupe: \(\{e,a\}\), de table:
            \begin{equation*}
                \begin{array}{c|cc} % chktex 44
                    \times & \cellcolor{blue!20}e & \cellcolor{gray!20}a \\
                    \hline % chktex 44
                    \cellcolor{blue!20}e & \cellcolor{blue!20}e & \cellcolor{gray!20}a \\
                    \cellcolor{gray!20}a & \cellcolor{gray!20}a & \cellcolor{blue!20}e \\
                \end{array}
            \end{equation*}
            \item Pour l'ordre 3, il n'y a qu'un groupe: \(\{e,a,b\}\), de table:
            \begin{equation*}
                \begin{array}{c|ccc} % chktex 44
                    \times & \cellcolor{blue!20}e & \cellcolor{gray!20}a & \cellcolor{gray!30}b \\
                    \hline % chktex 44
                    \cellcolor{blue!20}e & \cellcolor{blue!20}e & \cellcolor{gray!20}a & \cellcolor{gray!30}b \\
                    \cellcolor{gray!20}a & \cellcolor{gray!20}a & \cellcolor{gray!30}b & \cellcolor{blue!20}e \\
                    \cellcolor{gray!30}b & \cellcolor{gray!30}b & \cellcolor{blue!20}e & \cellcolor{gray!20}a \\
                \end{array}
            \end{equation*}
            \item Pour l'ordre 4, il y a deux groupes:
            \begin{equation*}
                \begin{array}{c|cccc} % chktex 44
                    \times & \cellcolor{blue!20}e & \cellcolor{gray!20}a & \cellcolor{gray!30}b & \cellcolor{gray!40}c \\
                    \hline % chktex 44
                    \cellcolor{blue!20}e & \cellcolor{blue!20}e & \cellcolor{gray!20}a & \cellcolor{gray!30}b & \cellcolor{gray!40}c \\
                    \cellcolor{gray!20}a & \cellcolor{gray!20}a & \cellcolor{gray!30}b & \cellcolor{gray!40}c & \cellcolor{blue!20}e \\
                    \cellcolor{gray!30}b & \cellcolor{gray!30}b & \cellcolor{gray!40}c & \cellcolor{blue!20}e & \cellcolor{gray!20}a \\
                    \cellcolor{gray!40}c & \cellcolor{gray!40}c & \cellcolor{blue!20}e & \cellcolor{gray!20}a & \cellcolor{gray!30}b \\
                \end{array}
                \quad \text{et} \quad
                \begin{array}{c|cccc} % chktex 44
                    \times & \cellcolor{blue!20}e & \cellcolor{gray!20}a & \cellcolor{gray!30}b & \cellcolor{gray!40}c \\
                    \hline % chktex 44
                    \cellcolor{blue!20}e & \cellcolor{blue!20}e & \cellcolor{gray!20}a & \cellcolor{gray!30}b & \cellcolor{gray!40}c \\
                    \cellcolor{gray!20}a & \cellcolor{gray!20}a & \cellcolor{blue!20}e & \cellcolor{gray!40}c & \cellcolor{gray!30}b \\
                    \cellcolor{gray!30}b & \cellcolor{gray!30}b & \cellcolor{gray!40}c & \cellcolor{blue!20}e & \cellcolor{gray!20}a \\
                    \cellcolor{gray!40}c & \cellcolor{gray!40}c & \cellcolor{gray!30}b & \cellcolor{gray!20}a & \cellcolor{blue!20}e \\
                \end{array}
            \end{equation*}

            \item Pour l'ordre 5, il y a un seul groupe:
            \begin{equation*}
                \begin{array}{c|ccccc} % chktex 44
                    \times & \cellcolor{blue!20}e & \cellcolor{gray!20}a & \cellcolor{gray!30}b & \cellcolor{gray!40}c & \cellcolor{gray!50}d \\
                    \hline % chktex 44
                    \cellcolor{blue!20}e & \cellcolor{blue!20}e & \cellcolor{gray!20}a & \cellcolor{gray!30}b & \cellcolor{gray!40}c & \cellcolor{gray!50}d \\
                    \cellcolor{gray!20}a & \cellcolor{gray!20}a & \cellcolor{gray!30}b & \cellcolor{gray!40}c & \cellcolor{gray!50}d & \cellcolor{blue!20}e \\
                    \cellcolor{gray!30}b & \cellcolor{gray!30}b & \cellcolor{gray!40}c & \cellcolor{gray!50}d & \cellcolor{blue!20}e & \cellcolor{gray!20}a \\
                    \cellcolor{gray!40}c & \cellcolor{gray!40}c & \cellcolor{gray!50}d & \cellcolor{blue!20}e & \cellcolor{gray!20}a & \cellcolor{gray!30}b \\
                    \cellcolor{gray!50}d & \cellcolor{gray!50}d & \cellcolor{blue!20}e & \cellcolor{gray!20}a & \cellcolor{gray!30}b & \cellcolor{gray!40}c \\
                \end{array}
            \end{equation*}
        \end{itemize}
    \end{td-sol}
}{}

\begin{td-exo}[Sous-groupes] % 4
    Lister tous les sous-groupes du groupe symétrique \(\mathfrak{S}_3\).
\end{td-exo}

\iftoggle{showsolutions}{
    \begin{td-sol}[]\, % 4
        Commençons par expliciter tous les éléments de \(\mathfrak{S}_3\):
        \begin{equation*}
            \mathfrak{S}_3 = \left\{e, (1~2), (1~3), (2~3), (1~2~3), (1~3~2)\right\}
        \end{equation*}

        Les sous-groupes de \(\mathfrak{S}_3\) sont:
        \begin{itemize}
            \item Les groupes triviaux:
            \begin{itemize}
                \item \(\{e\}\), le groupe trivial.
                \item \(\mathfrak{S}_3\), le groupe symétrique.
            \end{itemize}
            \item Les sous-groupes engendrés par un élément:
            \begin{itemize}
                \item \(\langle (1~2) \rangle = \{e, (1~2)\}\),
                \item \(\langle (1~3) \rangle = \{e, (1~3)\}\),
                \item \(\langle (2~3) \rangle = \{e, (2~3)\}\),
                \item \(\langle (1~2~3) \rangle = \{e, (1~2~3), (1~3~2)\}\),
                \item \(\langle (1~3~2) \rangle = \langle (1~2~3) \rangle\).
            \end{itemize}
            \item Les sous-groupes engendrés par plus d'un élément
            engendrent tout le groupe.
        \end{itemize}
    \end{td-sol}
}{}

\begin{td-exo}[24 heures chrono (contrôle continu 2023--2024)]\, % 5
    \begin{enumerate}
        \item Montrer que le groupe \(\Z/24\Z\) a autant de générateurs que de 
        sous-groupes. Faire la liste des générateurs. Faire la liste des
        sous-groupes, en décrivant chaque sous-groupe de la manière la plus
        explicite possible.

        \item Montrer que le groupe \({\left(\Z/24\Z\right)}^\times\) peut
        être engendré par 3 de ses éléments.

        \item Lister les sous-groupes d'ordre 2 du groupe \({\left(\Z/24\Z\right)}^\times\).

        \item Donner un exemple de sous-groupe d'ordre 4 du groupe
        \({\left(\Z/24\Z\right)}^\times\). (Bonus: lister tous les sous-groupes d'ordre 4 de \({\left(\Z/24\Z\right)}^\times\).)
    \end{enumerate}
\end{td-exo}

\iftoggle{showsolutions}{
    \begin{td-sol}[]\,
        \begin{enumerate}
            \item Les générateurs de \(\Z/24\Z\) sont exactement:
            \begin{equation*}
                \ol 1, \ol 5, \ol 7, \ol{11}, \ol{-11}, \ol{-7}, \ol{-5}, \ol{-1}
            \end{equation*}

            Les sous-groupes de \(\Z/24\Z\) sont:
            \begin{equation*}
                \begin{aligned}
                    \langle\ol{1}\rangle &= \Z/24\Z,\\
                    \langle\ol{2}\rangle &= \{\ol{0},\ol{2},\ol{4},\ldots,\ol{22}\},\\
                    \langle\ol{3}\rangle &= \{\ol{0},\ol{3},\ol{6},\ldots,\ol{21}\},\\
                    \langle\ol{4}\rangle &= \{\ol{0},\ol{4},\ol{8},\ol{12},\ol{16},\ol{20}\},\\
                    \langle\ol{6}\rangle &= \{\ol{0},\ol{6},\ol{12},\ol{18}\},\\
                    \langle\ol{8}\rangle &= \{\ol{0}, \ol{8}, \ol{16}\}
                    \langle\ol{12}\rangle &= \{\ol{0}, \ol{12}\},\\
                    \langle\ol{24}\rangle &= \{\ol{0}\}.
                \end{aligned}
            \end{equation*}
            Il y en a autant.

            \item On a
            \begin{equation*}
                {(\Z/24\Z)}^\times=\langle\ol 5, \ol 7, \ol{-1}\rangle
            \end{equation*}
            car
            \begin{equation*}
                \begin{aligned}
                    \ol 5\times \ol 7 = \ol{11}\\
                    \ol 5\times \ol{-1} = \ol{-5}\\
                    \ol 7\times \ol{-1} = \ol{-7}\\
                    \ol 5\times \ol 7\times \ol{-1} = \ol{-11}
                \end{aligned}
            \end{equation*}

            \item Un sous-groupe d'ordre 2 est de la forme
            \begin{equation*}
                \langle a\rangle = \{\ol 1, a\}
            \end{equation*}
            avec \(a\ne\ol 1, a^2 = \ol 1\). Or, \(\forall a\in{(\Z/24\Z)}^\times\),
            on a \(a^2=1\). Alors, les sous-groupes d'ordre 2 de \({(\Z/24\Z)}^\times\)
            sont:
            \begin{equation*}
                \begin{aligned}
                    \langle\ol 5\rangle &= \{\ol 1, \ol 5\},\\
                    \langle\ol 7\rangle &= \{\ol 1, \ol 7\},\\
                    \langle\ol{11}\rangle &= \{\ol 1, \ol{11}\},\\
                    \langle\ol{-11}\rangle &= \{\ol 1, \ol{-11}\},\\
                    \langle\ol{-7}\rangle &= \{\ol 1, \ol{-7}\},\\
                    \langle\ol{-5}\rangle &= \{\ol 1, \ol{-5}\},\\
                    \langle\ol{-1}\rangle &= \{\ol 1, \ol{-1}\}.
                \end{aligned}
            \end{equation*}

            \item Un exemple de sous-groupe d'ordre 4 est
            \begin{equation*}
                \langle\ol 5, \ol 7\rangle = \{\ol 1, \ol 5, \ol{7}, \ol{11}\}
            \end{equation*}
        \end{enumerate}
    \end{td-sol}
}{}

\begin{td-exo}[Union de sous-groupes (contrôle continu 2023--2024)]\, % 6
    \begin{enumerate}
        \item Soit \(G\) un groupe et \(H,K\)  deux sous-groupes de \(G\).
        Montrer que \(H\cup K\) est un sous-groupe de \(G\) si et seulement
        si \(H\subseteq K\) ou \(K\subseteq H\).

        \item Donner un exemple de groupe \(G\) et de trois sous-groupes
        \(H,K,L\) de \(G\) qui sont tous les trois différents de \(G\) et
        tels que \(G=H\cup K\cup L\).
    \end{enumerate}
\end{td-exo}

\begin{td-exo}[Morphismes de groupes?] % 7
    Ces choses-là sont-elles des morphismes de groupes?
    \begin{enumerate}
        \item \(f\colon\Z\to\Z^\ast, n\mapsto 2^n\).
        \item \(g\colon\C^\ast\to\C^\ast, z\mapsto z^3\).
        \item \(h\colon\Z/10\Z\to\Z, \ol k\mapsto k\).
        \item \(i\colon\glx2(\R)\to\R,A\mapsto\tr(A)\).
        \item \(j\colon\Z/3\Z\to\Z/6\Z,\ol k\mapsto \tilde k\).
        \item \(k\colon\mathfrak{S}_4\to\mathfrak{S}_4,\sigma\mapsto\sigma(1~2)\).
        \item \(l\colon\mathfrak{S}_4\to\mathfrak{S}_4,\sigma\mapsto(1~2)\sigma(1~2)\).
    \end{enumerate}
\end{td-exo}

\iftoggle{showsolutions}{
    \begin{td-sol}[]\, % 7
        \begin{enumerate}
            \item \(f\) n'est pas définie.
            \item \(g\) est un morphisme de groupes car
            \begin{equation*}
                g(z_1z_2) = {(z_1z_2)}^3 = z_1^3z_2^3 = g(z_1)g(z_2).
            \end{equation*}
            \item \(h\) n'est pas définie
            \item \(i\) n'est pas un morphisme de groupes car \(\tr(I_2)=2\ne 0\).
            \item \(j\) n'est pas un morphisme de groupes car \(j(\ol 1+\ol 2)=\tilde 0\ne\tilde 1+\tilde 2=j(\ol 1)+j(\ol 2)\).
            \item \(k\) n'est pas un morphisme de groupes car \(k(\id)=(1~2)\ne\id\).
            \item \(l\) est un morphisme de groupes car
            \begin{equation*}
                \begin{aligned}
                    l(\sigma_1\sigma_2) &= (1~2)(\sigma_1\sigma_2)(1~2)\\
                    &= (1~2)\sigma_1(1~2)(1~2)\sigma_2(1~2)\\
                    &= l(\sigma_1)l(\sigma_2).
                \end{aligned}
            \end{equation*}
        \end{enumerate}
    \end{td-sol}
}{}

\begin{td-exo}[Exo 15] % 8

\end{td-exo}

\iftoggle{showsolutions}{
    \begin{td-sol}[]\, % 8
        \begin{enumerate}
            \item Le sous groupe engendré par \(\ol 3\) est
            \begin{equation*}
                H = \{\ol 3, \ol 9, \ol 5, \ol 4, \ol 1\}
            \end{equation*}
            et le sous-groupe engendré par \(\ol 10\) est
            \begin{equation*}
                K = \{\ol{10}, \ol{1}\}
            \end{equation*}
            On vérifie alors les trois propriétés:
            \begin{itemize}[label=\((i)\)]
                \item Si \(x\in H\), alors le couple \((x,\ol 1)\) convient. Sinon,
                \(x\in\{\ol 2, \ol 6, \ol 7, \ol 8, \ol 10\}\). On a alors
                \begin{equation*}
                    \begin{aligned}
                        x = \ol 2 \equiv x = \ol 90 \implies (\ol 9, \ol 10)\\
                        x = \ol 6 \equiv x = \ol 50 \implies (\ol 5, \ol 10)\\
                        x = \ol 7 \equiv x = \ol 40 \implies (\ol 4, \ol 10)\\
                        x = \ol 8 \equiv x = \ol 30 \implies (\ol 3, \ol 10)\\
                        x = \ol{10} \implies (\ol{1}, \ol{10})
                    \end{aligned}
                \end{equation*}
                La première propriété est donc vérifiée pour tout \(x\in G\).

                \item On a bien \(H\cap K = \{1\}\).

                \item Le groupe \(\bb Z/p\bb Z\) est toujours abélien donc
                on a bien que tous les éléments de \(H\) et \(K\) commutent entre eux.
            \end{itemize}

            \item Commençons par montrer que les ensembles 
            \begin{equation*}
                H = G_1\times\{e_2\},\quad  K = \{e_1\}\times G_2
            \end{equation*}
            sont bien des sous-groupes de \(G\), soit qu'ils vérifient
            les trois propriétés:
            \begin{itemize}
                \item \(e_G\in H\),
                \item Si \(x,y\in H\), alors \(xy\in H\),
                \item Si \(x\in H\), alors \(x^{-1}\in H\).
            \end{itemize}
            Montrons-les:
            \begin{itemize}
                \item \(e_G = (e_{G_1}, e_{G_2})\in H\) car \(e_{G_1}\in G_1\).

                \item Soient \(x=(x_1, e_{G_2})\) et \(y=(y_1, e_{G_2})\) alors
                \begin{equation*}
                    xy = (x_1y_1, e_{G_2})\in H
                \end{equation*}
                car \(x_1y_1\in G_1\).

                \item Soit \(x=(x_1, e_{G_2})\), alors
                \begin{equation*}
                    x^{-1} = (x_1^{-1}, e_{G_2})\in H
                \end{equation*}
                car \(x_1^{-1}\in G_1\).
            \end{itemize}
            De même pour \(K\).

            On a montré que ces deux ensembles sont bien des sous-groupes
            de \(G\), montrons maintenant qu'ils vérifient les trois propriétés
            demandées:
            \begin{itemize}[label=\((i)\)]
                \item Soit \(x=(x_1, x_2)\in G\), alors 
                le couple \((x_1, e_{G_2}), (e_{G_1}, x_2)\) convient.

                \item Il est clair que \(H\cap K = \{(e_{G_1}, e_{G_2})\} = e_G\).

                \item Comme le produit se fait terme à terme et que tout 
                élément de \(G_i\) commute avec \(\{e_i\}\), on a bien que
                tous les éléments de \(H\) et \(K\) commutent entre eux.
            \end{itemize}

            \item On note \(H\) et \(K\) les sous-groupes de \(G\).
            \begin{enumerate}
                \item On procède par l'absurde. Supposons qu'il existe deux
                écritures de \(x=yz=y'z'\). Alors on a
                \begin{equation*}
                    \begin{aligned}
                        &yz = y'z'\\
                        \equiv &y^{-1}yz = y^{-1}y'z'\\
                        \equiv &z = y^{-1}y'z'\\
                    \end{aligned}
                \end{equation*}
                Or \(z\in K\) donc \(y^{-1}y'z'\in K\). Comme \(z'\in K\),
                on a \(y^{-1}y'\in K\). Or \(y,y'\in H\) donc \(y^{-1}y'\in H\).
                On a donc \(y^{-1}y'\in H\cap K = \{e\}\) donc \(y=y'\).
                Il en est de même pour \(z\).

                L'écriture est donc unique.

                \item On a déjà montré au point précédent que l'écriture
                est unique et donc tout élément de \(G\) peut s'écrire
                de manière unique comme \(yz\) avec \(y\in H\) et \(z\in K\)
                donc on peut considérer l'application
                \begin{equation*}
                    \begin{aligned}
                        f\colon H\times K &\to G\\
                        (x,y) &\mapsto xy
                    \end{aligned}
                \end{equation*}
                qui est bijective d'après \((i)\) et le point précédent.

                Montrons maintenant que c'est un morphisme de groupes:
                Soient \((x,y), (x',y')\in H\times K\), on a:
                \begin{equation*}
                    \begin{aligned}
                        f((x,y)(x',y')) &= f((xx',yy'))\\
                        &= xx'yy'\\
                        &= xyx'y'\\
                        &= f(x,y)f(x',y')
                    \end{aligned}
                \end{equation*}
                donc \(f\) est un morphisme de groupes.

                En conclusion, \(G\) est isomorphe à \(H\times K\).

                \begin{remark}
                    Via l'isomorphisme \(G = H\times K\), on a correspondance
                    entre les sous-groupes
                    \begin{equation*}
                        H \leftrightarrow H\times\{e\},\quad K\leftrightarrow\{e\}\times K
                    \end{equation*}
                    On retrouve donc le cas de la question 2.
                \end{remark}

                \begin{remark}
                    En appliquant à la question 1, on obtient un isomorphisme
                    de groupes:
                    \begin{equation*}
                        \begin{aligned}
                            {\left(\bb Z/11\bb Z\right)}^\times 
                            &\simeq \langle \ol 3\rangle\times\langle\ol{10}\rangle\\
                            &\simeq \left(\bb Z/5\bb Z\right)\times\left(\bb Z/2\bb Z\right)\\
                        \end{aligned}
                    \end{equation*}
                    On verra plus tard que pour \(p\) premier, le
                    groupe \({\left(\bb Z/p\bb Z\right)}^\times\) est cyclique (d'ordre \(p-1\)),
                    donc \({\left(\bb Z/p\bb Z\right)}^\times\simeq\bb Z/(p-1)\bb Z\). Notamment,
                    \begin{equation*}
                        \begin{aligned}
                            {\left(\bb Z/11\bb Z\right)}^\times
                            &\simeq\bb Z/10\bb Z\\
                            &\simeq\bb Z/5\bb Z\times\bb Z/2\bb Z
                        \end{aligned}
                    \end{equation*}
                    d'après le théorème des restes chinois (car \(5\) et \(2\) sont premiers entre eux).
                \end{remark}
                \begin{remark}
                    Il existe des groupes finis qui ne sont pas isomorphes à un produit direct 
                    de groupes cycliques, par exemple \(\mathfrak{S}_3\), car un produit
                    de groupes cycliques est abélien.

                    En revance, c'est un fait (pas trivial) que tout groupe fini abélien
                    est isomorphe à un produit de groupes cycliques.
                \end{remark}
            \end{enumerate}

            \item On se place dans le groupe \(\mathfrak{S}_3\). On considère alors
            \begin{equation*}
                \begin{aligned}
                    H = \langle (1~2)\rangle = \{(1~2), \id\},\\
                    K = \langle (1~2~3)\rangle = \{(1~2~3), (1~3~2), \id\}
                \end{aligned}
            \end{equation*}
            On a tout de suite \((ii)\) qui est vérifié.

            Il manque \((1~3), (2~3)\) à ``former''. Or
            \begin{equation*}
                \begin{aligned}
                    (1~3) &= (1~2)(1~2~3),\\
                    (2~3) &= (1~2)(1~3~2)
                \end{aligned}
            \end{equation*}
            Donc \((i)\) est vérifié.

            En revanche, on a \((1~2)(1~2~3) \neq (1~2~3)(1~2)\) donc 
            \((iii)\) n'est pas vérifié.
        \end{enumerate}
    \end{td-sol} % 12 13 18 20
}{}

\begin{td-exo}[Exo 12] % 12
    Enoncé à remplir.
\end{td-exo}

\iftoggle{showsolutions}{
    \begin{td-sol}[]\, % 12
        \begin{enumerate}
            \item On veut montrer l'égalité
            \begin{equation*}
                \color{orange}\overbrace{\color{black}\sigma (i_1~i_2~\cdots~i_k)\sigma^{-1}}^{\alpha}\color{black}
                = \color{orange}\overbrace{\color{black}(\sigma(i_1)~\sigma(i_2)~\ldots~\sigma(i_k))}^{\beta}\color{black}
            \end{equation*}
            Soit \(j\in\{1,\ldots,n\}\). Procédons par cas:

            \(\triangleright\) Cas 1: \(j\notin\{\sigma(i_1),\ldots,\sigma(i_k)\}\).

            Alors \(\beta(j) = j\). On calcule
            \begin{equation*}
                \begin{aligned}
                    \alpha(j)
                    &= \sigma(i_1~i_2~\cdots~i_k)\sigma^{-1}(j)\\
                    &= \sigma\left(\sigma^{-1}(j)\right)\\
                \end{aligned}
            \end{equation*}
            Donc \(\alpha(j) = \beta(j)\).

            \(\triangleright\) Cas 2: \(j = \sigma(i_r)\) pour un \(r\in\{1,\ldots,k\}\).

            Alors
            \begin{equation*}
                \begin{aligned}
                    \alpha(j)
                    &= \sigma(i_1~i_2~\cdots~i_k)\underbrace{\sigma^{-1}\left(\sigma(i_r)\right)}_{i_r}\\
                    &= \sigma(i_{r+1})\\
                \end{aligned}
            \end{equation*}

            D'autre part,
            \begin{equation*}
                \begin{aligned}
                    \beta(j)
                    &= (\sigma(i_1)~\sigma(i_2)~\ldots~\sigma(i_k))(\sigma(i_r))\\
                    &= \sigma(i_{r+1})\\
                \end{aligned}
            \end{equation*}
            Donc \(\alpha(j) = \beta(j)\).

            Conclusion: on a \(\forall j\in\{1,\ldots,n\}, \alpha(j) = \beta(j)\) donc \(\alpha = \beta\).

            \begin{remark}[Ca sort d'où?]
                Soient deux ensembles \(E\) et \(F\) de cardinal fini \(n\).
                Soit \(\sigma\colon E\to F\) une bijection. 

                Soit \(f\in \text{Bij}(E)\). Alors, \(\sigma f \sigma^{-1}\) est une bijection de \(F\).

                Slogan: ``\(\sigma f \sigma^{-1}\), c'est comme \(f\), après avoir renommé les éléments''.

                En effet, notons \(E = \{x_1,\ldots,x_n\}\) et \(F = \{y_1,\ldots,y_n\}\) avec \(\sigma(x_i) = y_i\).

                Si \(f(x_i) = x_j\) alors
                \begin{equation*}
                    (\sigma f \sigma^{-1})(y_i) = (\sigma f)(x_i) = \sigma(f(x_i)) = \sigma(x_j) = y_j
                \end{equation*}
                et
                \begin{equation*}
                    (\sigma f \sigma^{-1})(y_i) = y_j
                \end{equation*}
            \end{remark}

            \item Il suffit de montrer que toutes les transpositions adjacentes 
            \((1~2), (2~3), \ldots, (n-1~n)\) sont dans
            \begin{equation*}
                \langle (1~2), (1~2~\cdots~n)\rangle
            \end{equation*}
            (car par le cours, les transpositions adjacentes engendrent \(\mathfrak{S}_n\)).

            On a
            \begin{equation*}
                \begin{aligned}
                    &(1~2~\cdots~n)(1~2){(1~2~\cdots~n)}^{-1} = (2~3)\\
                    &(1~2~\cdots~n)(2~3){(1~2~\cdots~n)}^{-1} = (3~4)\\
                    \text{etc.}
                \end{aligned}
            \end{equation*}
            donc toutes les transpositions adjacentes sont dans \(\langle (1~2), (1~2~\cdots~n)\rangle\).

            Conclusion: \(\mathfrak{S}_n = \langle (1~2), (1~2~\cdots~n)\rangle\).
        \end{enumerate}
    \end{td-sol}
}{}

\begin{td-exo}[Exo 13] % 13
    Enoncé à remplir.
\end{td-exo}

\iftoggle{showsolutions}{
    \begin{td-sol}[]\, % 13
        Soient \(s_1,s_2\) deux réflexions de \(\bb R^2\).

        Sens indirect: Si \(s_1=s_2\) alors \(s_1s_2 =s_2s_1\).

        Si \(s_1=-s_2\) alors \(s_1s_2 = s_1(-s_1) = s_1\circ(-s_1) = -s_1\circ s_1 = -\id\).
        et \(s_2s_1 =(-s_1)\circ s_1 = -s_1\circ s_1 = -\id\).

        Sens direct: On suppose que \(s_1s_2 = s_2s_1\).

        Ecrivons \(s_1 = s_{\Delta_1}\) et \(s_2 = s_{\Delta_2}\) avec \(\Delta_1\) et \(\Delta_2\) deux droites linéaires de \(\bb R^2\).

        On sait par le cours que 
        \begin{equation*}
            s_1s_2=r_{-2\theta}
        \end{equation*}
        où \(\theta\) est l'angle orient de \(\Delta_1\) vers \(\Delta_2\).

        De même, on a
        \begin{equation*}
            s_2s_1=r_{+2\theta}
        \end{equation*}

        On obtient donc que \(r_{-2\theta} = r_{+2\theta}\) donc il existe \(k\in\bb Z\) tel que
        \(-2\theta = 2\theta + 2k\pi\) donc \(\theta = -k\frac{\pi}{2}\).

        On a donc soit \(\Delta_1 = \Delta_2\), soit \(\Delta_1 = \Delta_2^\perp\) (orthogonalité).

        Le premier donne \(s_1=s_2\), le second donne \(s_1=-s_{\Delta_2^\perp}\) donc \(s_1=-s_{\Delta_2}=s_2\).

        En effet, c'est un fait générale que
        \begin{equation*}
            s_{\Delta^\perp} = -s_\Delta
        \end{equation*}

        Pour se convaincre, choisissons une base \(e,f\) avec \(e\) dans \(\Delta\) et \(f\) dans \(\Delta^\perp\). Alors
        la matrice de \(s_\Delta\) est
        \begin{equation*}
            \begin{pmatrix}
                1 & 0\\
                0 & -1
            \end{pmatrix}
        \end{equation*}
        et la matrice de \(-s_{\Delta}\) est
        \begin{equation*}
            \begin{pmatrix}
                -1 & 0\\
                0 & 1
            \end{pmatrix}
        \end{equation*}
        c'est la matrice de \(s_{\Delta^\perp}\).
    \end{td-sol}
}{}

\begin{td-exo}[Exo 13] % 13
    Enoncé à remplir.
\end{td-exo}

\iftoggle{showsolutions}{
    \begin{td-sol}[]\, % 13
        solution à remplir
    \end{td-sol}
}{}
    