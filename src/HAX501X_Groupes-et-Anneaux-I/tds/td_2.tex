\subsection*{Etude de \(\Z/n\Z\)}\label{sec:tutorial_2}
\addcontentsline{toc}{subsection}{\nameref{sec:tutorial_2}}

\begin{td-exo}[En cercle] % 1
    Soit \(n\in\N^\ast\). On note
    \begin{equation*}
        \bb U_n=\left\{z\in\bb C\mid z^n=1\right\}
    \end{equation*}
    l'ensemble des racines \(n\)-ièmes de l'unité. Montrer que l'application
    \begin{equation*}
        \begin{aligned}
            f\colon\bb Z&\to \bb U_n\\
            k&\mapsto \exp\left(\frac{2ik\pi}{n}\right)
        \end{aligned}
    \end{equation*}
    passe au quotient par la relation de congruence modulo \(n\) et
    induit l'application
    \begin{equation*}
        \begin{aligned}
            g\colon\Z/n\Z&\to \bb U_n\\
            \ol k&\mapsto \exp\left(\frac{2ik\pi}{n}\right).
        \end{aligned}
    \end{equation*}
    Montrer que \(g\) est bijective.
\end{td-exo}
% ----- Solutions exo 1
\iftoggle{showsolutions}{
    \begin{td-sol}[] % 1
        On a \(g\) bijective par construction.
    \end{td-sol}
}{}

\begin{td-exo}[Inversibles] % 2
    Faire la liste des éléments inversibles de \(\Z/14\Z\) et
    calculer leurs inverses. Même chose avec \(\Z/20\Z\).
\end{td-exo}
% ----- Solutions exo 2
\iftoggle{showsolutions}{
    \begin{td-sol}[] % 2
        Pour \(\Z/14\Z\), les inversibles sont les éléments qui ne divisent pas \(14\), soit:
        \begin{equation*}
            \left\{\ol 1,\ol 3,\ol 5,\ol 9,\ol 11,\ol 13\right\} \text{ ou } \left\{\ol 1,\ol 3,\ol 5,\ol{-5},\ol{-3},\ol{-1}\right\}
        \end{equation*}
        On a donc:
        \begin{equation*}
            \begin{aligned}
                1^{-1}&\equiv 1\pmod{14}\\
                3^{-1}&\equiv 5\pmod{14}\\
                5^{-1}&\equiv 3\pmod{14}\\
                -5^{-1}&\equiv -3\pmod{14}\\
                -3^{-1}&\equiv -5\pmod{14}\\
                -1^{-1}&\equiv -1\pmod{14}\\
            \end{aligned}
        \end{equation*}
        Pour \(\Z/20\Z\), les inversibles sont les éléments qui ne divisent pas \(20\), soit:
        \begin{equation*}
            \left\{\ol 1,\ol 3,\ol 7,\ol 9,\ol 11,\ol 13,\ol 17,\ol 19\right\} \text{ ou } \left\{\ol 1,\ol 3,\ol 7,\ol{9},\ol{-9},\ol{-7},\ol{-3},\ol{-1}\right\}
        \end{equation*}
        On a donc:
        \begin{equation*}
            \begin{aligned}
                1^{-1}&\equiv 1\pmod{20}\\
                3^{-1}&\equiv 7\pmod{20}\\
                7^{-1}&\equiv 3\pmod{20}\\
                9^{-1}&\equiv 9\pmod{20}\\
                -9^{-1}&\equiv 9\pmod{20}\\
                -7^{-1}&\equiv -3\pmod{20}\\
                -3^{-1}&\equiv -7\pmod{20}\\
                -1^{-1}&\equiv -1\pmod{20}\\
            \end{aligned}
        \end{equation*}
    \end{td-sol}
}{}

\begin{td-exo}[Puissance] % 3
    On se place dans \(\Z/41\Z\). Calculer
    \(\ol 2^{2023}\)
\end{td-exo}
% ----- Solutions exo 3
\iftoggle{showsolutions}{
    \begin{td-sol}[] % 3
        Comme \(41\) est premier, \(2\) est inversible dans \(\Z/41\Z\).
        Par le petit théorème de Fermat, on a
        \begin{equation*}
            2^{40}\equiv 1\pmod{41}
        \end{equation*}
        On a donc
        \begin{equation*}
            2^{2023} = 2^{50\times 40 + 23} = {(2^{40})}^{50} \times 2^{23} \equiv 2^{23} \pmod{41}
        \end{equation*}
        Pour calculer \(2^{23}\), on utilise la méthode piétonne:
        \begin{equation*}
            \begin{aligned}
                \ol{2}^1&=\ol{2}\\
                \ol{2}^2&=\ol{4}\\
                \ol{2}^3&=\ol{8}\\
                \ol{2}^4&=\ol{16}\\
                \ol{2}^5&=\ol{32}\\
                \ol{2}^6&=\ol{23}\\
                \ol{2}^7&=\ol{5}\\
                \ol{2}^8&=\ol{10}\\
                \ol{2}^9&=\ol{20}\\
                \ol{2}^{10}&=\ol{40}\\
            \end{aligned}
        \end{equation*}
        On en conclut alors que \(\ol 2^{20}=\ol 1\) et alors
        \begin{equation*}
            \ol 2^{2023} = \ol 2^{23} = \ol 2^3 = \ol 8
        \end{equation*}
    \end{td-sol}
}{}

\begin{td-exo}[Sous-groupes] % 4
    Quels sous-groupes de \(\Z/1000\Z\) contiennent \(\ol{120}\)?
\end{td-exo}
% ----- Solutions exo 4
\iftoggle{showsolutions}{
    \begin{td-sol}[] % 4
        Les sous-groupes de \(\Z/1000\Z\) sont
        les \(\langle \ol{d}\rangle\) avec \(d\in\N^\ast\)
        un diviseur de \(1000\).

        On sait que \(1000=2^3 5^3\). Alors,
        les diviseurs positifs de \(1000\) sont les
        \(2^a 5^b\) avec \(a,b\in\left\{0,1,2,3\right\}\).

        Or 
        \begin{equation*}
            \langle \ol d\rangle = \left\{
                \ol{0},\ol{d},\ol{2d},\dots,\ol{(e-1)d}
                \right\}
        \end{equation*}
        avec \(e=\frac{1000}{d}\). Donc
        \begin{equation*}
            \begin{aligned}
                \ol{120}\in\langle \ol d\rangle 
                &\iff 120\in\left\{0,d,2d,\dots,(e-1)d\right\}\\
                &\iff d \mid 120
            \end{aligned}
        \end{equation*}
        On sait que \(d\) divise \(1000\) et \(120\) donc
        il divise \(1000\wedge 120 = 40\).

        On a donc les sous-groupes de \(\Z/1000\Z\) contenant \(\ol{120}\) sont
        \begin{equation*}
            \langle \ol{1}\rangle, \langle \ol{2}\rangle, \langle \ol{4}\rangle, \langle \ol{5}\rangle, \langle \ol{8}\rangle, \langle \ol{10}\rangle, \langle \ol{20}\rangle, \langle \ol{40}\rangle
        \end{equation*}
    \end{td-sol}
}{}

\begin{td-exo}[Théorème de Wilson] % 5
    Le but de cet exercice est de démontrer le théorème de Wilson:
    pour un entier \(n\geq 2\), on a:
    \begin{equation*}
        n\text{ est premier} \iff (n-1)!\equiv -1 \pmod{n}
    \end{equation*}
    \begin{enumerate}
        \item Soit \(p\) un nombre premier.
        \begin{enumerate}
            \item Quels éléments \(x\in(\Z/p\Z)\setminus\{\ol 0\}\) sont égaux à leur inverse?
            \item En calculant le produit de tous les éléments de \((\Z/p\Z)\setminus\{\ol 0\}\), montrer qu'on a:
            \begin{equation*}
                (p-1)!\equiv -1 \pmod{p}
            \end{equation*}
        \end{enumerate}
        \item Soit \(n\) un nombre composé. Montrer que:
        \begin{equation*}
            (n-1)!\not\equiv -1 \pmod{n}
        \end{equation*}
        En déduire le théorème de Wilson.
    \end{enumerate}
\end{td-exo}
% ----- Solutions exo 5
\iftoggle{showsolutions}{
    \begin{td-sol}[]\, % 5
        \begin{enumerate}
            \item Soit \(p\) un nombre premier.
            \begin{enumerate}
                \item Soit \(x\in(\Z/p\Z)\setminus\{\ol 0\}\). On a:
                \begin{equation*}
                    \begin{aligned}
                        x=x^{-1} &\iff x^2=\ol 1\\
                        &\iff x^2-\ol 1=\ol 0\\
                        &\iff \left(x-\ol 1\right)\left(x+\ol 1\right)=\ol 0\\
                        &\iff x-\ol 1=\ol 0 \text{ ou } x+\ol 1=\ol 0 \text{ (car \(p\) est premier)}\\
                        &\iff x=\ol 1 \text{ ou } x=\ol{-1}
                    \end{aligned}
                \end{equation*}
                Conclusion: les \(x\in\Z/p\Z\) égaux à leur inverse sont \(\ol 1\) et \(\ol{-1}\).

                \item Dans le produit:
                \begin{equation*}
                    \ol 1\times\ol 2\times\cdots\times\ol{p-1}
                \end{equation*}
                tous les éléments se simplifient avec leur inverse par paires
                \emph{sauf} les éléments égaux à leur propre inverse.

                Par la question précédente, on a donc:
                \begin{equation*}
                    \ol 1\times\ol 2\times\cdots\times\ol{p-1} 
                    = \ol{1}\times\ol{-1} = \ol{-1}
                \end{equation*}
                On en déduit:
                \begin{equation*}
                    (p-1)!\equiv -1 \pmod{p}
                \end{equation*}
            \end{enumerate}

            \item Soit \(n\) un nombre composé.
            
            On procède par l'absurde. Supposons que \((n-1)!\equiv -1 \pmod{n}\).

            Il existe donc \(k\in\Z\) tel que:
            \begin{equation*}
                (n-1)! = kn - 1
            \end{equation*}
            On en déduit que:
            \begin{equation}
                kn - (n-1)! = 1 \label{eq:wilson_contradiction}
            \end{equation}
            Comme \(n\) est composé, on peut écrire \(n=ab\) avec
            \(1<a,b<n\).

            Comme \(a | (n-1)!\), pn peut voir~\eqref{eq:wilson_contradiction} comme
            une relation de Bézout entre \(n\) et \(a\) qui montre que
            \(n\wedge a = 1\), ce qui est absurde car \(n\wedge a=a\neq 1\).

            Autre raisonnement:
            
            On écrit \(n=ab\) avec \(1<a,b<n\).

            \(\triangleright\) Cas 1: \(a\neq b\)

            Dans ce cas, on voit apparaître \(a\) et \(b\) à
            des places différentes dans le produit \((n-1)!\) et donc
            \(ab\) divise \((n-1)!\). Alors \((n-1)!\equiv 0\pmod{n}\).

            \(\triangleright\) Cas 2: \(a=b\), soit \(n=a^2\).

            Dans ce cas, on voit apparaître \(a\) et \(2a\) à
            des places différentes dans le produit \((n-1)!\) (à
            condition que \(a>2\)) et donc \(a^2\) divise \((n-1)!\).
            Alors \((n-1)!\equiv 0\pmod{n}\).

            Si \(a=2\), on a \(n=4\) et alors \((3!)=6\not\equiv -1\pmod{4}\).
            
        \end{enumerate}
    \end{td-sol}
}{}

\begin{td-exo}[Une formule de Gauss] % 6
    Soit un entier \(n\in\N^\ast\). On veut montrer qu'on a:
    \begin{equation*}
        \sum_{d\mid n}\varphi(d) = n
    \end{equation*}
    \begin{enumerate}
        \item Vérifier que la formule est vraie pour \(n=12\).
        \item Soit \(d\) un diviseur de \(n\). On note \(e=\frac{n}{d}\).
        Montrer qu'il y a \(\varphi(e)\) entiers \(a\in\left\{1,\dots,n\right\}\)
        tels que \(a\wedge n=d\).
        \item Conclure.
    \end{enumerate}
\end{td-exo}
% ----- Solutions exo 6
\iftoggle{showsolutions}{
    \begin{td-sol}[] % 6
        \begin{enumerate}
            \item Pour \(n=12\), on a:
            \begin{equation*}
                \begin{aligned}
                    \sum_{d\mid 12}\varphi(d) 
                    &= \underset{1}{\varphi(1)} + \underset{1}{\varphi(2)} + \underset{2}{\varphi(3)} + \underset{2}{\varphi(4)} + \underset{2}{\varphi(6)} + \underset{4}{\varphi(12)} \\
                    &= 12
                \end{aligned}
            \end{equation*}
            Donc la formule est vraie pour \(n=12\).

            \item Pour \(a\in\left\{1,\dots,n\right\}\) tel que
            \(a\wedge n=d\), on a \(d\mid a\) et donc il
            existe un entier \(k\) tel que \(a=kd\).

            Alors, on a
            \begin{equation*}
                (kd)\wedge(ed)=d
            \end{equation*}
            d'où \(k\wedge e=1\).

            De plus, comme \(a\in\left\{1,\dots,n\right\}\), on a
            nécessairement \(k\in\left\{1,\dots,e\right\}\).

            On a montré que:
            \begin{equation*}
                \left(a\in\left\{1,\dots,n\right\}\mid a\wedge n=d\right)
                \implies \left(\exists k\in\left\{1,\dots,e\right\}\mid k\wedge=1,a=kd\right)
            \end{equation*}

            La réciproque \(\left(\impliedby\right)\) est aussi vraie,
            il suffit de faire le même raisonnement.

            Il y a donc autant d'entiers \(a\in\left\{1,\dots,n\right\}\)
            tels que \(a\wedge n=d\) que d'entiers \(k\in\left\{1,\dots,e\right\}\).

            Or, il y a \(\varphi(e)\) entiers \(k\in\left\{1,\dots,e\right\}\)
            tels que \(k\wedge e=1\).

            Alors, il y a \(\varphi(e)\) entiers \(a\in\left\{1,\dots,n\right\}\)
            tels que \(a\wedge n=d\).

            \item On partitionne l'ensemble \(\left\{1,\dots,n\right\}\)
            suivant le pgcd avec \(n\), ce pgcd est alors un 
            diviseur \(d\) de \(n\).

            On a donc:
            \begin{equation*}
                \left\{1,\dots,n\right\} = \bigsqcup_{d\mid n}\left\{a\in\left\{1,\dots,n\right\}\mid a\wedge n=d\right\}
            \end{equation*}

            Par le point précédent, on sait que les ensembles
            \(\left\{a\in\left\{1,\dots,n\right\}\mid a\wedge n=d\right\}\)
            sont disjoints et leur réunion est \(\left\{1,\dots,n\right\}\).

            On peut réécrire cette somme dans un autre ordre en 
            utilisant l'involution \(d\mapsto \frac{n}{d}\) de 
            l'ensemble des diviseurs de \(n\). On obtient:
            \begin{equation*}
                \sum_{d\mid n}\varphi(d) = n
            \end{equation*}
        \end{enumerate}
    \end{td-sol}
}{}

\subsection*{Exercices supplémentaires, et approfondissement}

\begin{td-exo}[Equations]\, % 7
    \begin{enumerate}
        \item Résoudre dans \(\Z/37\Z\) l'équation \(\ol 7x+ \ol 5=\ol 1\).
        \item Résoudre dans \(\Z/37\Z\) l'équation \(x^2- \ol 6x+ \ol{10}=\ol 0\).
    \end{enumerate}
\end{td-exo}

\begin{td-exo}[Un exercice de baccalauréat (filière C, académie de Paris, juin 1978)] % 8
    Dans l'anneau \(\Z/91\Z\) (dont les éléments sont notés \(\ol 0, \ol 1, \dots, \ol{90}\)), 
    \begin{enumerate}
        \item discuter, suivant la valeur du paramètre \(a\in\Z/91\Z\), l'équation
        \begin{equation*}
            ax=\ol 0
        \end{equation*}
        
        \item résoudre l'équation
        \begin{equation*}
            x^2+\ol 2x-\ol 3=\ol 0
        \end{equation*}
    \end{enumerate}
\end{td-exo}
