\subsection*{Exercices en TD}\label{sec:tutorial_4}
\addcontentsline{toc}{subsection}{\nameref{sec:tutorial_4}}

\begin{td-exo}[Un corps exotique] % 1
    On définit sur l'ensemble \(\bb R^2\) une addition
    \begin{equation*}
        (x,y) + (x',y') = (x+x',y+y')
    \end{equation*}
    et une multiplication
    \begin{equation*}
        (x,y) \cdot (x',y') = (xx'-yy',xy'+x'y.)
    \end{equation*}
    Montrer que \(\bb R^2\) muni de ces deux lois est un corps.
\end{td-exo}

\iftoggle{showsolutions}{
    \begin{td-sol}[]\, % 1
        Il faut montrer que \(\bb R^2\) muni de ces deux lois est un corps, soit
        que c'est un anneau commutatif dans lequel tout élément non nul est inversible.

        \ptr{} Commençons par montrer que c'est un anneau.\\
        Il faut vérifier les 4 axiomes:
        \begin{enumerate}
            \item On sait que \(\bb R^2\) muni de l'addition par composantes est un groupe abélien.

            \item On vérifie que la multiplication est associative:
            \begin{equation*}
                \begin{aligned}
                    ((x,y) \cdot (x',y')) \cdot (x'',y'') &= (xx'-yy',xy'+x'y) \cdot (x'',y'')\\
                    &= ((xx'-yy')x''-(xy'+x'y)y'',(xx'-yy')y''+(xy'+x'y)x'')\\
                    &= (x(x'x''-y'y'')-y(y'x''+x'y''),x(x'y''+x'y'')+y(x'x''-y'y''))\\
                    &= (x(x'x''-y'y'')-y(y'x''+x'y''),x(x'y''+x'y'')+y(x'x''-y'y''))\\
                    &= (x,y) \cdot (x',y') \cdot (x'',y'').
                \end{aligned}
            \end{equation*}

            \item On vérifie qu'il existe un élément neutre pour la multiplication:
            \begin{equation*}
                \begin{aligned}
                    (x,y) \cdot (1,0) &= (x\cdot 1-y\cdot 0,x\cdot 0+y\cdot 1)\\
                    &= (x,y).
                \end{aligned}
            \end{equation*}

            \item On vérifie la distributivité de la multiplication par rapport à l'addition:
            \begin{equation*}
                \begin{aligned}
                    (x,y) \cdot \left((x',y')+(x'',y'')\right) &= (x,y) \cdot (x'+x'',y'+y'')\\
                    &= (x(x'+x'')-y(y'+y''),x(y'+y'')+y(x'+x''))\\
                    &= (xx'-yy'+xx''-yy'',xy'+xy''+yx'+yx'')\\
                    &= (xx'-yy',xy'+x'y) + (xx''-yy'',xy''+x'y'')\\
                    &= (x,y) \cdot (x',y') + (x,y) \cdot (x'',y'').
                \end{aligned}
            \end{equation*}
        \end{enumerate}

        \ptr{} Montrons maintenant que cet anneau est commutatif.\\
        Il faut vérifier que la multiplication est commutative:
        \begin{equation*}
            \begin{aligned}
                (x,y) \cdot (x',y') &= (xx'-yy',xy'+x'y)\\
                &= (x'x-y'y,x'y+xy')\\
                &= (x',y') \cdot (x,y).
            \end{aligned}
        \end{equation*}

        \ptr{} Montrons enfin que tout élément non nul est inversible.\\
        Soit \((x,y) \in \bb R^2\) tel que \(x^2+y^2 \neq 0\). On a
        \begin{equation*}
            \begin{aligned}
                (x,y) \cdot \left(\frac{x}{x^2+y^2},-\frac{y}{x^2+y^2}\right) &= \left(x\cdot \frac{x}{x^2+y^2}-y\cdot \left(-\frac{y}{x^2+y^2}\right),x\cdot \left(-\frac{y}{x^2+y^2}\right)+y\cdot \frac{x}{x^2+y^2}\right)\\
                &= \left(\frac{x^2}{x^2+y^2}+\frac{y^2}{x^2+y^2},\frac{xy}{x^2+y^2}-\frac{xy}{x^2+y^2}\right)\\
                &= \left(1,0\right).
            \end{aligned}
        \end{equation*}
        Donc tout élément non nul est inversible.

        \ptr{} Conclusion: \(\bb R^2\) muni de ces deux lois est un corps.
    \end{td-sol}
}{}

% \begin{td-exo}[Exo 1]\, % 1
%     Enoncé à remplir.
% \end{td-exo}
% 
% \iftoggle{showsolutions}{
%     \begin{td-sol}[]\, % 1
%         solution à remplir
%     \end{td-sol}
% }{}
