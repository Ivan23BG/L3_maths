\subsection*{Exercices en TD}\label{sec:tutorial_4}
\addcontentsline{toc}{subsection}{\nameref{sec:tutorial_4}}

\begin{td-exo}[Un corps exotique] % 1
    On définit sur l'ensemble \(\bb R^2\) une addition
    \begin{equation*}
        (x,y) + (x',y') = (x+x',y+y')
    \end{equation*}
    et une multiplication
    \begin{equation*}
        (x,y) \cdot (x',y') = (xx'-yy',xy'+x'y.)
    \end{equation*}
    Montrer que \(\bb R^2\) muni de ces deux lois est un corps.
\end{td-exo}

\iftoggle{showsolutions}{
    \begin{td-sol}[]\, % 1
        Il faut montrer que \(\bb R^2\) muni de ces deux lois est un corps, soit
        que c'est un anneau commutatif dans lequel tout élément non nul est inversible.

        \ptr{} Commençons par montrer que c'est un anneau.\\
        Il faut vérifier les 4 axiomes:
        \begin{enumerate}
            \item On sait que \(\bb R^2\) muni de l'addition par composantes est un groupe abélien.

            \item On vérifie que la multiplication est associative:
            \begin{equation*}
                \begin{aligned}
                    ((x,y) \cdot (x',y')) \cdot (x'',y'') &= (xx'-yy',xy'+x'y) \cdot (x'',y'')\\
                    &= ((xx'-yy')x''-(xy'+x'y)y'',(xx'-yy')y''+(xy'+x'y)x'')\\
                    &= (x(x'x''-y'y'')-y(y'x''+x'y''),x(x'y''+x'y'')+y(x'x''-y'y''))\\
                    &= (x(x'x''-y'y'')-y(y'x''+x'y''),x(x'y''+x'y'')+y(x'x''-y'y''))\\
                    &= (x,y) \cdot (x',y') \cdot (x'',y'').
                \end{aligned}
            \end{equation*}

            \item On vérifie qu'il existe un élément neutre pour la multiplication:
            \begin{equation*}
                \begin{aligned}
                    (x,y) \cdot (1,0) &= (x\cdot 1-y\cdot 0,x\cdot 0+y\cdot 1)\\
                    &= (x,y).
                \end{aligned}
            \end{equation*}

            \item On vérifie la distributivité de la multiplication par rapport à l'addition:
            \begin{equation*}
                \begin{aligned}
                    (x,y) \cdot \left((x',y')+(x'',y'')\right) &= (x,y) \cdot (x'+x'',y'+y'')\\
                    &= (x(x'+x'')-y(y'+y''),x(y'+y'')+y(x'+x''))\\
                    &= (xx'-yy'+xx''-yy'',xy'+xy''+yx'+yx'')\\
                    &= (xx'-yy',xy'+x'y) + (xx''-yy'',xy''+x'y'')\\
                    &= (x,y) \cdot (x',y') + (x,y) \cdot (x'',y'').
                \end{aligned}
            \end{equation*}
        \end{enumerate}

        \ptr{} Montrons maintenant que cet anneau est commutatif.\\
        Il faut vérifier que la multiplication est commutative:
        \begin{equation*}
            \begin{aligned}
                (x,y) \cdot (x',y') &= (xx'-yy',xy'+x'y)\\
                &= (x'x-y'y,x'y+xy')\\
                &= (x',y') \cdot (x,y).
            \end{aligned}
        \end{equation*}

        \ptr{} Montrons enfin que tout élément non nul est inversible.\\
        Soit \((x,y) \in \bb R^2\setminus\{(0,0)\}\). On cherche \((x',y') \in \bb R^2\) tel que \((x,y) \cdot (x',y') = (1,0)\). On a donc
        \begin{equation*}
            \begin{aligned}
                &\begin{cases}
                    xx'-yy' &= 1\\
                    xy'+x'y &= 0
                \end{cases}\\
                \iff &\begin{pmatrix}
                    x & -y\\
                    y & x
                \end{pmatrix}
                \begin{pmatrix}
                    x'\\
                    y'
                \end{pmatrix}
                = \begin{pmatrix}
                    1\\
                    0
                \end{pmatrix}\\
                \iff & \frac{1}{x^2+y^2}\begin{pmatrix}
                    x & y\\
                    -y & x
                \end{pmatrix}
                \begin{pmatrix}
                    x'\\
                    y'
                \end{pmatrix}
                = \begin{pmatrix}
                    1\\
                    0
                \end{pmatrix}\\
                \iff & \begin{pmatrix}
                    x'\\
                    y'
                \end{pmatrix}
                = \frac{1}{x^2+y^2}\begin{pmatrix}
                    x & y\\
                    -y & x
                \end{pmatrix}
                \begin{pmatrix}
                    1\\
                    0
                \end{pmatrix}
                = \begin{pmatrix}
                    \frac{x}{x^2+y^2}\\
                    \frac{-y}{x^2+y^2}
                \end{pmatrix}
            \end{aligned}
        \end{equation*}
        On pose alors
        \begin{equation*}
            (x',y') = \left(\frac{x}{x^2+y^2},\frac{-y}{x^2+y^2}\right).
        \end{equation*}
        et on vérifie que \((x,y) \cdot (x',y') = (1,0)\).

        Ce corps est \(\bb C\):
        \begin{equation*}
            \begin{aligned}
                &\bb R^2&\bb C\\
                &(x,y) \longleftrightarrow& x+iy
            \end{aligned}
        \end{equation*}
    \end{td-sol}
}{}

\begin{td-exo}[Entiers de Gauss]\, % 3
    On définit l'ensemble des entiers de Gauss:
    \begin{equation*}
        \bb Z[i] = \{a+ib \mid a,b \in \bb Z\}.
    \end{equation*}
    \begin{enumerate}
        \item Montrer que \(\bb Z[i]\) est un sous-anneau de \(\bb C\).
        Est-il commutatif? Est-il un intègre? Est-il un corps?

        \item Pour \(z = a+ib \in \bb Z[i]\), on pose
        \begin{equation*}
            N(z) = \n{z}^2 = a^2+b^2
        \end{equation*}
        qu'on appelle la norme. Montrer qu'on a
        \begin{equation*}
            \forall z,z' \in \bb Z[i],\quad N(zz') = N(z)N(z').
        \end{equation*}

        \item Montrer que si \(z\in\bb Z[i]\) est inversible
        si et seulement si \(N(z) = 1\). Identifier le groupe
        des inversibles de \(\bb Z[i]\).

        \item Soient \(m,n\in\bb N\). Montrer que si \(m\) et \(n\) 
        peuvent être écrits comme somme de deux carrés, alors leur
        produit \(mn\) aussi.

        \item Soit maintenant l'ensemble des rationnels de Gauss:
        \begin{equation*}
            \bb Q(i) = \left\{a+ib \in \bb C \mid a,b \in \bb Q\right\}.
        \end{equation*}
        Montrer qu'il s'agit d'un sous-corps de \(\bb C\).
    \end{enumerate}
\end{td-exo}

\iftoggle{showsolutions}{
    \begin{td-sol}[]\, % 3
        On se place dans
        \begin{equation*}
            \bb Z[i] = \{a+ib \mid a,b \in \bb Z\}.
        \end{equation*}
        \begin{enumerate}
            \item On montre facilement que c'est un sous-anneau de \(\bb C\)
            en vérifiant les axiomes.
            \begin{itemize}[\ptr{}]
                \item On a bien \(0\in\bb Z[i]\)

                \item On a bien \(\bb Z[i]\) stable par addition.

                \item On a bien \(\bb Z[i]\) stable par passage à l'opposé.

                \item On a bien \(1\in\bb Z[i]\)

                \item On a bien \(\bb Z[i]\) stable par multiplication.
            \end{itemize}
            Donc \(\bb Z[i]\) est un sous-anneau de \(\bb C\).

            Cet anneau est bien commutatif car \(\bb C\) l'est. 
            De même pour l'intégrité.\\
            Par contre, \(\bb Z[i]\) n'est pas un corps car \(2\in\bb Z[i]\)
            n'est pas inversible.

            \item On sait que 
            \begin{equation*}
                \forall z,z' \in \bb C,\quad \n{zz'} = \n{z}\n{z'}.
            \end{equation*}
            et donc
            \begin{equation*}
                N(zz') = N(z)N(z').
            \end{equation*}

            \item \ptr{} Montrons d'abord le sens direct.\\
            On suppose que \(z\in\bb Z[i]\) est inversible.\\
            Alors, il existe \(z'\in\bb Z[i]\) tel que \(zz' = 1\).\\
            En utilisant la question précédente, on a
            \begin{equation*}
                N(zz') = N(z)N(z') = 1.
            \end{equation*}
            Comme \(N(z)\in\bb N\) et \(N(z')\in\bb N\), on a
            \begin{equation*}
                N(z) = N(z') = 1.
            \end{equation*}

            \ptr{} Montrons maintenant le sens réciproque.\\
            On suppose que \(N(z) = 1\).\\
            Alors \(z = a+ib\) avec \(a,b\in\bb Z\) et \(a^2+b^2 = 1\).
            Alors
            \begin{equation*}
                (a+ib)(a-ib) = 1.
            \end{equation*}
            Donc \(a+ib\) est inversible dans \(\bb Z[i]\),
            d'inverse \(a-ib\).

            Pour \(a,b\in\bb Z\), on a
            \begin{equation*}
                a^2+b^2 = 1 \iff (a,b) \in \left\{(1,0),(0,1),(-1,0),(0,-1)\right\}.
            \end{equation*}
            Donc les inversibles de \(\bb Z[i]\) sont
            \begin{equation*}
                \bb {Z[i]}^\times = \{1,-1,i,-i\} = \bb U_4.
            \end{equation*}
            (groupe cyclique d'ordre 4, engendré par \(i\) ou \(-i\))

            \item Formule magique vue à la question 2:
            \begin{equation*}
                (a^2+b^2)(a'^2+b'^2) = {(aa'-bb')}^2 + {(ab'+ba')}^2.
            \end{equation*}

            \item C'est clair que \(\bb Q(i)\) est un sous-corps de \(\bb C\),
            montrons simplement l'axiome de stabilité par passage à l'inverse.\\
            Pour \(a,b\in\bb Q\), si \(a+ib\neq 0\), alors
            \begin{equation*}
                \frac{1}{a+ib} = \frac{a}{a^2+b^2} - i\frac{b}{a^2+b^2} \in \bb Q(i).
            \end{equation*}
        \end{enumerate}
    \end{td-sol}
}{}

\begin{td-exo}[Nilpotents]\, % 4
    Soit \(A\) un anneau commutatif. On dit qu'un élément
    \(x\in A\) est nilpotent s'il existe \(n\in\bb N\) tel que
    \(x^n = 0\).
    \begin{enumerate}
        \item Donner un exemple d'un anneau commutatif \(A\)
        et d'un élément nilpotent \(x\in A\) non nul.

        \item Montrer que si \(x\) est nilpotent et que \(y\)
        est n'importe quel élément de \(A\), alors \(xy\) est nilpotent.

        \item Montrer que l'ensemble des éléments nilpotents de \(A\)
        est un sous-groupe de \(A\). Est-ce un sous-anneau?

        \item Montrer que si \(x\) est nilpotent alors \(1-x\)
        est inversible.

        \item Soient \(u,x\in A\) tel que \(u\) est inversible et
        \(x\) est nilpotent. Montrer que \(u+x\) est inversible.
    \end{enumerate}
\end{td-exo}

\iftoggle{showsolutions}{
    \begin{td-sol}[]\, % 4
        \begin{itemize}
            \item On considère
            \begin{equation*}
                A = \bb Z/4\bb Z
            \end{equation*}
            alors \(x=\ol 2\neq \ol 0\) est nilpotent car \(x^2 = \ol 4 = \ol 0\).

            \item Soit \(x\in A\) nilpotent (d'ordre \(n\)) et \(y\in A\). Alors
            \begin{equation*}
                {(xy)}^n = x^n y^n = 0.
            \end{equation*}

            \item Soit \(N\) l'ensemble des éléments nilpotents de \(A\).
            Vérifions les axiomes:
            \begin{itemize}[\ptr{}]
                \item On a bien \(0\in N\) car \(0^1 = 0\).

                \item Soit \(x\in N\). Alors il existe \(n\in\bb N\) tel que \(x^n = 0\).
                Alors \({(-x)}^n = (-1)^n x^n = 0\), donc \(-x\in N\).

                \item Soit \(x,y\in N\). Alors il existe \(n,m\in\bb N\) tels que \(x^n = 0\) et \(y^m = 0\).
                Alors
                \begin{equation*}
                    \begin{aligned}
                        {(x+y)}^{n+m} 
                        &= \sum_{k=0}^{n+m} \binom{n+m}{k} x^k y^{n+m-k}\\
                        &= \sum_{k=0}^{n} \binom{n+m}{k} x^k y^{n+m-k} + \sum_{k=n+1}^{n+m} \binom{n+m}{k} x^k y^{n+m-k}\\
                        &= 0.
                    \end{aligned}
                \end{equation*}
                Comme \(1\) n'est pas nilpotent, \(N\) n'est pas un sous-anneau.
                \begin{remark}
                    C'est un idéal de \(A\) par la question 2.
                \end{remark}
            \end{itemize}
        \end{itemize}
    \end{td-sol}
}{}


% \begin{td-exo}[Exo 1]\, % 1
%     Enoncé à remplir.
% \end{td-exo}
% 
% \iftoggle{showsolutions}{
%     \begin{td-sol}[]\, % 1
%         solution à remplir
%     \end{td-sol}
% }{}
