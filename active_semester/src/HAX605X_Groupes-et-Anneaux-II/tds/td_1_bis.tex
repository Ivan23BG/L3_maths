% exos jusqu'au 4 A REMPLIR
%%%%%%%%%%%%%%%%%%%%%%%%%%%%%%%%%%%%%%%%

% --- Consignes exo 5
\begin{td-exo} % 5
    % fill
\end{td-exo}
% --- Solution exo 5
\iftoggle{showsolutions}{
    \begin{td-sol}
        On a \(\glx{2}(\bb F_2)\act\bb P^1(\bb F_2)\).

        On rappelle que \(\bb F_p\coloneqq \bb Z/p\bb Z\) et
        \begin{equation*}
            \begin{aligned}
                &\n{\GLn(\bb F_p)} = \prod_{k=0}^{n-1} (p^n - p^k)\\
                \implies &\n{\glx{2}(\bb F_2)} = (2^2 - 2^0)(2^2 - 2^1) = 6.
            \end{aligned}
        \end{equation*}
        D'autre part,
        \begin{equation*}
            \bb P^1(\bb F_2) = \{[e_1], [e_2], [e_1 + e_2]\} \simeq \bb F_2^2 \setminus \{0\}.
        \end{equation*}
        On peut alors trouver un morphisme de groupe
        \begin{equation*}
            \rho\from \glx{2}(\bb F_2) \to \mathfrak{S}_{\bb P^1(\bb F_2)} \simeq \mathfrak{S}_3
        \end{equation*}
        On sait aussi que 
        \begin{equation*}
            \begin{aligned}
                A\in\ker(\rho) &\implies \begin{cases}
                    A[e_1] = [e_1]\\
                    A[e_2] = [e_2]\\
                    A[e_1 + e_2] = [e_1 + e_2]
                \end{cases}\\
                &\implies A = \begin{pmatrix}
                    \lambda & 0\\
                    0 & \lambda
                \end{pmatrix}.
            \end{aligned}
        \end{equation*}
        Avec \(\lambda\in\bb F_2^\times\). Donc \(A = \id_2\).

        Donc \(\rho\) est fidèle (injective) et comme
        \begin{equation*}
            \n{\glx{2}(\bb F_2)} = 6,
        \end{equation*}
        on a \(\rho \) isomorphisme.

        
        On a \(\scr D_3\act \mu_3 = \{\zeta^1,\zeta^2,\zeta^3\}\) où \(\zeta = e^{2i\pi/3}\). Alors
        \begin{equation*}
            \rho\from \scr D_3 \to \mathfrak{S}_{\mu_3} \simeq \mathfrak{S}_3.
        \end{equation*}
        Observons ce qui arrive quand on agit sur \(\mu_3\). On a
        \begin{equation*}
            R^i S^j \cdot \zeta^k =
            \begin{cases}
                \zeta^{i+k} &\text{si } j = 0\\
                \zeta^{i-k} &\text{si } j = 1
            \end{cases}
        \end{equation*}
        On remarque alors que
        \begin{equation*}
            R^i S^j \in\ker(\rho)
            \implies i = 0 \text{ et } j = 0
        \end{equation*}
        pour \(0 \leq i \leq n-1, 0 \leq j \leq 1\). Donc \(\ker(\rho) = \{e\}\) et \(\rho\) est fidèle.

        Comme \(\n{\scr D_3} = 6\), on a \(\rho\) isomorphisme.
    \end{td-sol}
}{}

% --- Consignes exo 6
\begin{td-exo} % 6
    On a
    \begin{equation*}
        \bb R^\times = G\act X = \bb R^2\setminus \{0\}
    \end{equation*}
    par multiplication par scalaire. Trouver une bijection
    \begin{equation*}
        X/G = \bb P^1(\bb R) \to S^1 = \{z\in\bb C\mid |z| = 1\}
    \end{equation*}
\end{td-exo}
% --- Solution exo 6
\iftoggle{showsolutions}{
    \begin{td-sol}
        On considère l'application
        \begin{equation*}
            \begin{aligned}
                f\from &X \to S^1\\
                &z \mapsto \frac{z^2}{\n{z^2}}.
            \end{aligned}
        \end{equation*}
        Elle est surjective car \(e^{i\theta} = f(e^{i\theta/2})\) pour tout \(\theta\in\bb R\). 

        Elle est injective car
        \begin{equation*}
            f(x) = f(y) \iff [x] = [y]\in X/G.
        \end{equation*}
        et
        \begin{equation*}
            \begin{aligned}
                \frac{x^2}{\n{x^2}} = \frac{y^2}{\n{y^2}}
                \iff \frac{x}{\n x} = \pm \frac{y}{\n y}\\
                \iff [x] = [y].
            \end{aligned}
        \end{equation*}
        Grâce à la propriété universelle des quotients, 
        \begin{equation*}
            \begin{aligned}
                \exists! \ol f\from &X/G \to S^1\\
                & [z] \mapsto \frac{z^2}{\n{z^2}}.
            \end{aligned}
        \end{equation*}
        qui est en plus surjective et injective, donc bijective.
    \end{td-sol}
}{}

% --- Consignes exo 7
\begin{td-exo}\, % 7
    \begin{enumerate}[label=(\roman*)] % chktex 36
        \item Trouver une bijection de \(\bb R/\bb Z\to S^1\). (Bonus: 
        trouver une bijection \(\bb R\)-équivalente.)

        \item Trouver une bijection de \(O_n(\bb R)/i(O_{n-1}(\bb R))\to S^{n-1}\).
    \end{enumerate}
\end{td-exo}
% --- Solution exo 7
\iftoggle{showsolutions}{
    \begin{td-sol}\, % 7
        \begin{enumerate}[label=(\roman*)] % chktex 36
            \item On considère l'application
            \begin{equation*}
                \begin{aligned}
                    \alpha \from &\bb R \times S^1 \to S^1\\
                    &(\lambda, z) \mapsto e^{2i\pi\lambda}z.
                \end{aligned}
            \end{equation*}
            C'est bien une action de \(\bb R\) sur \(S^1\). De plus, comme
            \begin{equation*}
                e^{\theta i} = \frac{\theta}{2\pi} \cdot 1,
            \end{equation*}
            \(\alpha\) est transitive et donc \(S^1 = G\cdot 1\).

            Aussi,
            \begin{equation*}
                \begin{aligned}
                    G_1
                    &= \left\{g\in G \mid g\cdot 1 = 1\right\}\\
                    &= \left\{\lambda\in\bb R \mid e^{2i\pi\lambda} = 1\right\}\\
                    &= \bb Z.
                \end{aligned}
            \end{equation*}
            Donc
            \begin{equation*}
                \begin{aligned}
                    \bb R/\bb Z &\to S^1\\
                    [\lambda] &\mapsto e^{2i\pi\lambda}
                \end{aligned}
            \end{equation*}
            est une bijection.

            \item Dans l'exercice 2 on a vu que
            \begin{equation*}
                O_n(\bb R) = G\act X = S^{n-1}.
            \end{equation*}
            transitivement et on a calculé par \(x = e_1\in S^{n-1}\)
            \begin{equation*}
                G_{e_1} = \left\{
                    \begin{pmatrix}
                        1 & 0\\
                        0 & D
                    \end{pmatrix}
                    \bigg \mid
                    D\in O_{n-1}(\bb R)
                \right\}.
            \end{equation*}
            donc on obtient \og{} gratuitement \fg{} une bijection
            \(G\)-équivalente
            \begin{equation*}
                \ol{\alpha}_x\from O_n(\bb R)/i(O_{n-1}(\bb R)) \to S^{n-1}.
            \end{equation*}
            %tout simplement en envoyant
            %\begin{equation*}
            %    \begin{pmatrix}
            %        1 & 0\\
            %        0 & D
            %    \end{pmatrix}
            %    \mapsto D\cdot e_1.
            %\end{equation*}
        \end{enumerate}
    \end{td-sol}
}{}