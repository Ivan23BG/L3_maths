\begin{td-exo}[]\, % 1
    \begin{enumerate}
        \item Soit \(\Omega\) un ensemble muni d'une tribu \(\scr F\) et 
        \(x\in\Omega\). Montrer que
        \begin{equation*}
            \delta_x(A) = \one_A(x)
        \end{equation*}
        définit une probabilité sur \((\Omega,\scr F)\).

        \item Soit \({(\bb P_n)}_{n\ge 1}\) une suite de mesures
        de probabilité sur un espace mesurable \((\Omega,\scr F)\) et
        \({(a_n)}_{n\ge 1}\) une suite de réels dans \(\ff{0,1}\) telle que
        \begin{equation*}
            \sum_{n=1}^{\infty} a_n = 1.
        \end{equation*}
        Montrer que
        \begin{equation*}
            \sum_{n=1}^{\infty} a_n \bb P_n
        \end{equation*}
        est une probabilité sur \((\Omega,\scr F)\).

        \item Soit \(I\) un intervalle de \(\R\) de mesure de Lebesgue
        \(\lambda(I)\) finie et strictement positive. Montrer que
        \begin{equation*}
            \bb P(A) = \frac{\lambda(A)}{\lambda(I)}
        \end{equation*}
        définit une probabilité sur \((I,\scr B(I))\).

        \item Soit \((\Omega,\scr F,\mu)\) un espace mesuré (pas forcément
        de probabilité) et \(f\colon \Omega\to\fo{0,\infty}\) une fonction
        mesurable telle que
        \begin{equation*}
            \int_{\Omega} f(\omega) \der\mu(\omega) = 1.
        \end{equation*}
        Montrer que l'application
        \begin{equation*}
            \begin{aligned}
                \bb P\colon \scr F &\to \bb R\\
                A &\mapsto \int_{\Omega} f(\omega) \one_A(\omega) \der\mu(\omega)
            \end{aligned}
        \end{equation*}
        est une probabilité sur \((\Omega,\scr F)\).
    \end{enumerate}
\end{td-exo}
% ----- Solutions exo 1
\iftoggle{showsolutions}{
    \begin{td-sol}[]\, % 1
        test
    \end{td-sol}
}{}