\section{Exemples importants de groupes}\label{sec:exemples-importants-de-groupes}

\subsection*{\(A^\times\)}\label{subsec:ax}

Soit \(A\) un anneau et \(A^\times\) l'ensemble des éléments inversibles de \(A\).
L'ensemble \(A^\times\) est un groupe pour la multiplication.

Si \(A = \bb K\) est un corps, alors pour tout \(n\in\bb N\),
l'ensemble
\begin{equation*}
    \mu_n(\bb K) = \left\{z\in\bb K\ \mid\ z^n=1\right\}
\end{equation*}
est un groupe pour la multiplication.

\begin{remark}
    On a \(\mu_n \simeq \bb Z/n\bb Z\) via l'isomorphisme de groupes
    \begin{equation*}
        \begin{aligned}
            \bb Z/n\bb Z &\to \mu_n\\
            \overline{k} &\mapsto e^{2i\pi k/n}
        \end{aligned}
    \end{equation*}
\end{remark}

\subsection*{\(\GLn(\bb K)\)}\label{subsec:glnk}

Soit \(\bb K\) un corps et \(\GLn(\bb K)\) l'ensemble des matrices carrées inversibles
de taille \(n\) à coefficients dans \(\bb K\).
L'ensemble \(\GLn(\bb K)\) est un groupe pour la multiplication des matrices.

\begin{remark}
    Si \(\bb K = \bb F_p\), c'est-à-dire un \(\bb Z/p\bb Z\) avec \(p\) premier,
    alors \(| \GLn(\bb F_p) |\) est fini. Pour le calculer, considérons
    \(X \in \GLn(\bb F_p)\). On a
    \begin{equation*}
        X = \begin{pmatrix}
            X_1 & X_2 & \cdots & X_n
        \end{pmatrix}
    \end{equation*}
    avec \(X_i \in \bb F_p^n\). On a \(X_1 \neq 0\), donc on a \(p^n - 1\) choix
    pour \(X_1\). \\
    Ensuite, on a \(X_2\notin \bb F_p X_1 = \vect_{\bb F_p}(X_1)\), donc on a
    \(p^n - p\) choix pour \(X_2\). \\
    En général, on a \(p^n - p^{i-1}\) choix pour \(X_i\). \\
    On a donc
    \begin{equation*}
        | \GLn(\bb F_p) | = (p^n - 1)(p^n - p)\cdots(p^n - p^{n-1}) = \prod_{k=0}^{n-1} (p^n - p^k)
    \end{equation*}
\end{remark}

\subsection*{\(\scr S_n\)}\label{subsec:sn}

Considérons les éléments suivants:
\begin{itemize}
    \item \(n > 1\) un entier naturel,
    \item \(R \in \GLn(\bb R)\) la rotation d'angle \(\frac{2\pi}{n}\)
    dans le plan (dans le sens anti-horaire),
    \item \(S \in \GLn(\bb R)\) la réflexion par rapport à l'axe des abscisses.
\end{itemize}

Si on identifie \(\bb R^2\) à \(\bb C\), alors pour tout \(z \in \bb C\),
    \begin{equation*}
        R(z) = e^{\frac{2i\pi}{n}}z \quad\text{et}\quad S(z) = \overline{z}
    \end{equation*}
    et alors pour tout \(k \in \bb Z\),
    \begin{equation*}
        S R^k S = R^{-k}
    \end{equation*}
    Alors, le groupe
    \begin{equation*}
        \scr D_n = \left\{ \id, R, \ldots, R^{n-1}, S, SR, \ldots, SR^{n-1} \right\}
    \end{equation*}
    est un sous-groupe de \(\GLn(\bb R)\), c'est le groupe diédral à \(2n\) éléments.


\section{Action de groupe}\label{sec:action-de-groupe}

Soit \(G\) un groupe et \(X\) un ensemble.

\begin{definition}
    Une \defemph{action} de \(G\) sur \(X\) est une application
    \begin{equation*}
        \begin{aligned}
            \alpha : G \times X &\to X\\
            (g,x) &\mapsto g \cdot x
        \end{aligned}
    \end{equation*}
    telle que
    \begin{enumerate}[label=(\roman*)] % chktex 36
        \item pour tout \(x \in X\), on a \(e \cdot x = x\),
        \item pour tout \(g,h \in G\) et \(x \in X\), on a \(g \cdot (h \cdot x) = (gh) \cdot x\).
    \end{enumerate}
\end{definition}

\begin{notation}\,
    \begin{itemize}
        \item On notera \(g \cdot x\) pour signifier \(\alpha(g,x)\).
        \item On notera \(G \act X\) pour signifier que \(G\) agit sur \(X\).
    \end{itemize}
\end{notation}

\begin{definition}
    Un \defemph{\(G\)-ensemble} est un ensemble muni de l'action de \(G\).
\end{definition}


\begin{definition}
    Une \defemph{représentation} de \(G\) dans \(X\) est un
    morphisme de groupes
    \begin{equation*}
        \rho : G \to {\mathfrak{S}}_X
    \end{equation*}
    où \({\mathfrak{S}}_X\) est le groupe des permutations/bijections de \(X\).
\end{definition}

\begin{notation}
    On notera alors pour tout \(g \in G\)
    \begin{equation*}
        \rho_g\coloneq \rho(g)
    \end{equation*}
    et pour tout \(x \in X\)
    \begin{equation*}
        \rho_g(x)\coloneq \rho(g(x))
    \end{equation*}
\end{notation}

\begin{exo}\,
    \begin{itemize}
        \item Montrer que si \(\alpha : G\times X \to X\) est une action alors
        il existe \(\rho : G\to {\mathfrak{S}}_X\) telle que, 
        pour tout \(g\in G\), on a
        \begin{equation*}
            \begin{aligned}
                \rho(g) :X &\to X\\
                x &\mapsto g\cdot x
            \end{aligned}
        \end{equation*}
    
        \item Réciproquement, montrer que si \(\rho : G\to {\mathfrak{S}}_X\) est une représentation
        alors \(\alpha : G\times X \to X\) définie pour tout \(g\in G\) et \(x\in X\) par
        \begin{equation*}
            \alpha(g,x)\coloneq \rho_g(x)
        \end{equation*}
        est une action.
    \end{itemize}
\end{exo}

\begin{example}\,
    \begin{itemize}
        \item Soit \(n\in\bb N\). Le groupe symétrique \({\mathfrak{S}}_n\) agit sur \(\{1,\ldots,n\}\)
        par permutation, c'est-à-dire
        \begin{equation*}
            \begin{aligned}
                {\mathfrak{S}}_n \times \{1,\ldots,n\} &\to \{1,\ldots,n\}\\
                (\sigma,k) &\mapsto \sigma(k)
            \end{aligned}
        \end{equation*}
        
        \item Soit \(\bb K\) un corps. Le groupe \(\GLn(\bb K)\) agit sur \(\bb K^n\) par multiplication,
        c'est-à-dire
        \begin{equation*}
            \begin{aligned}
                \GLn(\bb K) \times \bb K^n &\to \bb K^n\\
                (A,x) &\mapsto A x
            \end{aligned}
        \end{equation*}

        \item Soit \(n\in\bb N\). Le groupe diédral \(\scr D_n\) agit sur \(\mu_n\) par multiplication,
        c'est-à-dire
        \begin{equation*}
            \begin{aligned}
                \scr D_n \times \mu_n &\to \mu_n\\
                (g,\zeta) &\mapsto g(\zeta)
            \end{aligned}
        \end{equation*}
        On peut vérifier que cette action est bien définie pour les générateurs \(R\) et \(S\).

        \item Soit \(H < G\) (sous-groupe de \(G\)). On a
        \begin{enumerate}
            \item L'action par translation à gauche:
            \begin{equation*}
                H \act G \quad\text{par}\quad \rho^L : H \to {\mathfrak{S}}_G
            \end{equation*}
            avec \(\rho^L_h(g) = hg\)

            \item L'action par translation à droite:
            \begin{equation*}
                H \act G \quad\text{par}\quad \rho^R : H \to {\mathfrak{S}}_G
            \end{equation*}
            avec \(\rho^R_h(g) = gh^{-1}\)
        \end{enumerate}
    \end{itemize}
    \begin{remark}
        Attention, en général \(\rho_h(g)\coloneq gh\) ne définit pas une action de \(H\) sur \(G\).
    \end{remark}
\end{example}

% FIN DU PREMIER COURS, NOUVELLE PARTIE

\begin{definition}
    Soient \(X\) et \(Y\) des \(G\)-ensembles. On dit que
    \begin{equation*}
        f\ \colon\ X\to Y
    \end{equation*}
    est \defemph{\(G\)-équivariante} si pour tout \(x\in X\) et tout \(g\in G\), on a
    \begin{equation*}
        f(g\cdot x) = g\cdot f(x)
    \end{equation*}
\end{definition}

\begin{exo}
    On considère \(G\) un groupe et \(H\) un sous-groupe de \(G\).
    On note \(G^L\) (respectivement \(G^R\)) l'ensemble \(G\) muni de l'action de \(H\)
    par translation à gauche (respectivement à droite). Montrer que
    \begin{equation*}
        \begin{aligned}
            {(\cdot)}^{-1}\ \colon\ G^L &\to G^R\\
            g &\mapsto g^{-1}
        \end{aligned}
    \end{equation*}
    est une bijection \(H\)-équivariante.
\end{exo}

\begin{definition}\,
    Soient \(G\) et \(\Gamma\) des groupes et \(V\) un \(\bb K\)-espace vectoriel.
    \begin{enumerate}[label=(\roman*)] % chktex 36
        \item Si \(G\act\Gamma\), les assertions suivantes sont équivalentes:
        \begin{itemize}
            \item \(G\) \defemph{agit par homomorphismes} sur \(\Gamma\),
            \item pour tout \(g\in G\) et tout \(\gamma_1,\gamma_2\in\Gamma\), on a
            \begin{equation*}
                g\cdot(\gamma_1\gamma_2) = (g\cdot\gamma_1)(g\cdot\gamma_2)
            \end{equation*}
            \item Il existe un morphisme de groupes
            \begin{equation*}
                \rho\ \colon\ G\to\aut(\Gamma) < {\mathfrak{S}}_\Gamma
            \end{equation*}
            tel que pour tout \(g\in G\), on a \(\rho_g\) est un morphisme de groupes.
        \end{itemize}
        
        \item Si \(G\act V\), les assertions suivantes sont équivalentes:
        \begin{itemize}
            \item \(G\) \defemph{agit linéairement} sur \(V\) (l'action est linéaire),
            \item pour tout \(g\in G\) et tout \(v_1,v_2\in V\) et tout \(\lambda_1,\lambda_2\in\bb K\), on a
            \begin{equation*}
                g\cdot(\lambda_1 v_1 + \lambda_2 v_2) = \lambda_1(g\cdot v_1) + \lambda_2(g\cdot v_2)
            \end{equation*}
            \item Il existe un morphisme de groupes
            \begin{equation*}
                \rho\ \colon\ G\to\glx{\bb K}(V) < {\mathfrak{S}}_V
            \end{equation*}
            tel que pour tout \(g\in G\), on a \(\rho_g\) est une application linéaire.
        \end{itemize}
    \end{enumerate}
\end{definition}

\begin{example}\ 
    \begin{enumerate}
        \item Avec \(H < G\), l'action de \(H\) par translation à gauche sur \(G\) est
        une action par homomorphismes si et seulement si \(H = \{e\}\).

        En effet, si \(H = \{e\}\), alors l'action est triviale. Réciproquement, si
        l'action est par homomorphismes, on a
        \begin{equation*}
            \begin{aligned}
                & h \cdot (gg') = (h \cdot g)(h \cdot g')\\
                \iff & hgg' = hghg'\\
                \iff & e = h
            \end{aligned}
        \end{equation*}
        pour tout \(g,g'\in G\), donc \(H = \{e\}\).

        \item L'action de \(\GLn(\bb K)\) sur \(\bb K^n\) est linéaire.

        \item L'action par conjugaison:\\
        Si \(H < G\) alors \(H\act G\) par \(\rho^C\ \colon\ H\to\aut(G) < {\mathfrak{S}}_G\)
        et \(\rho^C_h(g) = hgh^{-1}\).\\
        Il s'agit d'une action par homomorphismes.
    \end{enumerate}
\end{example}

\begin{theorem}[de Cayley]
    Si \(G\) est un groupe d'ordre \(n\), alors il est isomorphe à un sous-groupe
    de \({\mathfrak{S}}_n\).
\end{theorem}

\begin{proof}
    On sait que \(G\) agit sur lui meme par translation à gauche
    \(\rho^L\ \colon\ G\to{\mathfrak{S}}_G \simeq {\mathfrak{S}}_n\). 
    Donc
    \begin{equation*}
        g\in \ker(\rho^L) \implies \rho^L_g(e) = g\cdot e = e \implies g = e
    \end{equation*}
    Donc \(\rho^L\) est injectif et
    \begin{equation*}
        \rho^L : G \to \rho^L(G) < {\mathfrak{S}}_G
    \end{equation*}
    est un isomorphisme de groupes.
\end{proof}

\begin{example}
    \(\mu_n\) est isomorphe au sous-groupe de \({\mathfrak{S}}_n\) engendré par \((1~2~\cdots~u)\).
    \begin{equation*}
        \zeta_n = e^{2i\pi/n}, \quad \mu_n = \{\zeta^1,\ldots,\zeta^{n}\} \simeq \{1,2,\ldots,n\}
    \end{equation*}
    et
    \begin{equation*}
        \begin{aligned}
            \rho^L : \mu_n &\to {\mathfrak{S}}_{\mu_n} \simeq {\mathfrak{S}}_{n}\\
            \zeta_n^k &\mapsto {(1~2~\cdots~n)}^k
        \end{aligned}
    \end{equation*}
\end{example}

\begin{definition}
    On prend \(G \act X\). 
    \begin{enumerate}
        \item On dit que \(Y\sub X\) est \defemph{stable} par \(G\) si 
        \begin{equation*}
            \{g\cdot y\ \mid\ g\in G,\ y\in Y\} = G\cdot Y = Y
        \end{equation*}

        \item L'\defemph{orbite} de \(x\in X\) est
        \begin{equation*}
            \orb(x) = G\cdot x= \{g\cdot x\ \mid\ g\in G\}
        \end{equation*}
        qui est stable par \(G\).

        \item Le \defemph{stabilisateur} de \(x\in X\) est
        \begin{equation*}
            \stab(x) = G_x = \{g\in G\ \mid\ g\cdot x = x\}
        \end{equation*}
        qui est un sous-groupe de \(G\).

        \item On dit que \(x\in X\) est un \defemph{point fixe} de \(g\in G\) si
        \begin{equation*}
            g\cdot x = x
        \end{equation*}
        c'est à dire si \(g\in\stab(x)\).
        L'ensemble des points fixes de \(g\) est noté
        \begin{equation*}
            X^g = \{x\in X\ \mid\ g\cdot x = x\}
        \end{equation*}
        De plus, \(x\in X\) est un point fixe de \(G\) si et seulement si
        \begin{equation*}
            x\in X^g,\quad\forall g\in G
        \end{equation*}
        c'est à dire si et seulement si \(G_X = G\). L'ensemble des points fixes de \(G\) est
        noté
        \begin{equation*}
            X^G = \{x\in X\ \mid\ g\cdot x = x,\ \forall g\in G\}
        \end{equation*}

        \item L'action est \defemph{transitive} si il existe \(x \in X\)
        tel que \(\orb(x) = G\cdot x = X\) (dans ce cas, \(X = G\cdot x,\forall x\in X\))
        Dans ce cas, on dit que \(X\) est un \defemph{\(G\)-espace homogène}.

    \end{enumerate}
\end{definition}


%%%%%%%%%%%%%%%%%%%%%%%%%%%%%%%%%%%%%%%%%%%%%%%%%%%%%%%%%%%%%%%%%
%% Remplir manquant
%%%%%%%%%%%%%%%%%%%%%%%%%%%%%%%%%%%%%%%%%%%%%%%%%%%%%%%%%%%%%%%%%



% def de l'andice





\begin{proposition}
    Soit \(X\) un \(G\)-ensemble et \(Y\) un ensemble.

    Pour toute application \(f\from X\to Y\) constante sur les orbites, il existe une
    unique fonction \(\ol f\from X/G \to Y\) telle que
    \begin{equation*}
        \forall x\in X,\quad \ol f(\orb (x)) = f(x)
    \end{equation*}
\end{proposition}

\begin{proof}
    Par définition du quotient.
\end{proof}

\begin{lemma}
    Soit \(X\) un \(G\)-ensemble et \(x\in X\). On a les propriétés suivantes:
    \begin{enumerate}[label=(\roman*)] %chktex 36
        \item Il existe une bijection
        \begin{equation*}
            \begin{aligned}
                \ol{\alpha}_x\from &G/G_x\to G\cdot x\\
                &gG_x\mapsto g\cdot x
            \end{aligned}
        \end{equation*}

        \item \({\ol \alpha}_x\) est \(G\)-équivariante. C'est-à-dire que \(G\act G/G_x\)
        par translation à gauche, soit \(g\cdot g' G_x \coloneqq gg' G_x\).

        \item Pour tout \(g\in G\), on a
        \begin{equation*}
            G_{g\cdot x} = gG_x g^{-1}
        \end{equation*}
    \end{enumerate}
\end{lemma}

\begin{proof}\,
    \begin{enumerate}[label=(\roman*)] %chktex 36
        \item L'application \({\ol \alpha}_x\) est bien définie:
        \begin{equation*}
            \begin{aligned}
                g'G_x = gG_x &\implies s\in G_x\from g' = gs\\
                &\implies g'\cdot x = (gs)\cdot x &= g\cdot(s\cdot x)\\
                & & g\cdot x
            \end{aligned}
        \end{equation*}

        L'application \({\ol \alpha}_x\) est aussi surjective par
        définition de l'orbite

        L'application \({\ol \alpha}_x\) est injective:
        \begin{equation*}
            \begin{aligned}
                &{\ol \alpha}_x (gG_x) = {\ol \alpha}_x (g'G_x)\\
                \Leftrightarrow & g\cdot x = g'\cdot x\\
                \Leftrightarrow & g^{-1}\cdot(g\cdot x) = g^{-1} \cdot (g' \cdot x)\\
                \Leftrightarrow & g^{-1} g' \in G_x \Leftrightarrow gG_x = g'G_x
            \end{aligned}
        \end{equation*}

        \item On a
        \begin{equation*}
            \begin{aligned}
                {\ol \alpha}_x(g\cdot g'G_x) &= {\ol \alpha}_x (gg' G_x)\\
                &= (gg')\cdot x\\
                &= g\cdot (g'\cdot x)\\
                &= g\cdot {\ol \alpha}_x (g' G_x)
            \end{aligned}
        \end{equation*}

        \item Soit \(s\in G_{g\cdot x}\) Alors
        \begin{equation*}
            \begin{aligned}
                &s\cdot(g\cdot x) = g\cdot x\\
                \Leftrightarrow & g^{-1}\cdot(s\cdot(g\cdot x)) = g^{-1}\cdot (g\cdot x)\\
                \Leftrightarrow & (g^{-1}sg)\cdot x = x\\
                \Leftrightarrow & g^{-1}sg\in G_x\\
                \Leftrightarrow & s\in gG_x g^{-1}
            \end{aligned}
        \end{equation*}
    \end{enumerate}
\end{proof}

\begin{corollary}
    Soit \(X\) un \(G\)-espace homogène (c'est à dire qu'il n'y a qu'une seule orbite). Alors,
    il existe \(H < G\) et \(f\from G/H\to X\) une bijection \(G\)-équivariante.
\end{corollary}

\begin{proof}
    Soit \(x\in X\), on pose \(G = G_x\) et on applique le Lemme précédent.
\end{proof}

\begin{example}
    On sait que \(\GLn(\bb K)\act \bb P^{n-1}(\bb K)\) transitivement et
    donc on obtient une application
    \begin{equation*}
        \GLn(\bb K)/ H\to \bb P^{n-1}(\bb K)
    \end{equation*}
    bijective et \(G\)-équivariante où
    \begin{equation*}
        H = {(\GLn(\bb K))}_{\ff{e_1}} =
        \left\{
            \begin{pmatrix}
                a&b\\
                0&D
            \end{pmatrix}
            \bigg |
            \begin{matrix}
                a\in \bb K^x\\
                b^T \in\bb K^{n-1}\\
                D\in \glx{n-1}(\bb K)
            \end{matrix}
        \right\}
    \end{equation*}
\end{example}

\begin{corollary}[Formule des classes]
    Soient \(G,X\) finis et \(G\act X\). Alors, les propriétés suivantes sont vraies:
    \begin{enumerate}[label=(\roman*)] % chktex 36
        \item Pour tout \(x\in X\), on a \(\n{G\cdot x} = \ff{G\from G_x} = \n{G/G_x}\).
        \item Si on a \(X = (G\cdot x_1) \sqcup \cdot \sqcup (G\cdot x_n)\) alors
        \begin{equation*}
            \n X = \sum_{i=1}^n \n{G\cdot x_i} = \sum_{i=1}^n \frac{\n G}{\n{G_{x_i}}}
        \end{equation*}
    \end{enumerate}
\end{corollary}

\begin{proof}\,
    \begin{enumerate}[label=(\roman*)] % chktex 36
        \item On a 
        \begin{equation*}
            \begin{aligned}
                \ol{\alpha}_x\from &G/G_x\to G\cdot x\\
                &gG_x\mapsto g\cdot x
            \end{aligned}
        \end{equation*}
        bijective donc
        \begin{equation*}
            \n{G\cdot x} = \n{G/G_x} = \ff{G\from G_x}
        \end{equation*}

        \item On a \(X = \bigsqcup_{i=1}^n (G\cdot x_i)\) donc
        \begin{equation*}
            \begin{aligned}
                \n X
                &= \sum_{i=1}^n \n{G\cdot x_i}\\
                &= \sum_{i=1}^n \n{G/G_{x_i}}\\
                &= \sum_{i=1}^n \frac{\n G}{\n{G_{x_{i}}}}
            \end{aligned}
        \end{equation*}
    \end{enumerate}
\end{proof}

\begin{definition}
    Soit \(p\in N\) premier. Un groupe \(G\)
    est un \defemph{\(p\)-groupe fini} si \(\n{G} = p^n\)
    avec \(n>0\).
\end{definition}

\begin{lemma}
    Si \(G\) est un \(p\)-groupe fini et \(X\) un \(G\)-ensemble fini.
    Alors
    \begin{equation*}
        \n X \equiv \n{X^G} \pmod p
    \end{equation*}
    où \(X^G\) est l'ensemble des points fixes de \(G\act X\).
\end{lemma}

\begin{proof}
    Soit \(x\in X\setminus X^G\). Alors
    \begin{equation*}
        1 < \n{G\cdot x} = \n{G/G_x} = \frac{\n{G}}{\n{G_{x}}}
    \end{equation*}
    qui divise \(\n G\). Alors \(\n{G\cdot x}\equiv 0 \pmod p\)

    Si \(X = (G\cdot x_1)\sqcup\cdots\sqcup (G\cdot x_n)\) et 
    \(X = (G\cdot x_1)\sqcup\cdots\sqcup (G\cdot x_m)\) avec \(1\leq m\leq n\).
    Alors la formule des classes donne
    \begin{equation*}
        \n{X} = \sum_{i=1}^m \n{G\cdot x_i} + \sum_{j = m+1}^n \n{G\cdot x_j}
        \equiv m = \n{X^G}\pmod p
    \end{equation*}
\end{proof}

\begin{corollary}
    Soit \(G\) un \(p\)-groupe fini. Alors, le centre de \(G\)
    noté \(\mathsf{Z}(G) \neq \{e\}\).
\end{corollary}

\begin{proof}
    On a \(G\act G\) par conjugaison, donc
    \begin{equation*}
        \n G \equiv \n{G^G} = \n{\mathsf{Z}(G)} \pmod p
    \end{equation*}
    et donc \(\n{\mathsf{Z}(G)} > 1\).
\end{proof}

\begin{theorem}[de Cauchy]
    Soit \(p\in\bb N\) premier qui divise \(\n G\). Alors
    \(G\) admet un élément d'ordre \(p\).
\end{theorem}

\begin{proof}
    A voir sur le Moodle.
\end{proof}

\begin{lemma}[de Burnside]
    Soit \(G\) un groupe fini et \(X\) un \(G\)-ensemble fini.
    Alors
    \begin{equation*}
        \n{X/G} = \frac{1}{\n{G}}\times \sum_{g\in G}\n{X^g}
    \end{equation*}
\end{lemma}













