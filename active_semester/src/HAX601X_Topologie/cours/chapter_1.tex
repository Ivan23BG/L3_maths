\section{Espaces métriques}\label{sec:espaces-metriques}

Soit \(X\) un ensemble.

\begin{definition}
    On appellle une \defemph{distance} (ou métrique) sur \(X\)
    une application \(d\from X \times X \to \R\) telle que
    pour tout \(x, y, z \in X\),
    \begin{enumerate}[label=(\roman*)] % chktex 36
        \item la distance est \emph{positive}:
        \begin{equation*}
            d(x, y) \geq 0
        \end{equation*}

        \item la distance possède la \emph{séparation}:
        \begin{equation*}
            d(x, y) = 0 \iff x = y
        \end{equation*}

        \item la distance est \emph{symétrique}:
        \begin{equation*}
            d(x, y) = d(y, x)
        \end{equation*}

        \item la distance vérifie l'\emph{inégalité triangulaire}:
        \begin{equation*}
            d(x, z) \leq d(x, y) + d(y, z)
        \end{equation*}
    \end{enumerate}
\end{definition}

\begin{example}
    Un exemple classique de distance est la \defemph{distance euclidienne} sur \(\R^n\):
    \begin{equation*}
        d(x, y) = \sqrt{\sum_{i=1}^n {(x_i - y_i)}^2}
    \end{equation*}
\end{example}

\begin{definition}
    Soit \(E\) un \(\bb R\)-espace vectoriel.
    On appelle \defemph{norme} sur \(E\) une application \(\nn{\cdot}\from E \to \R^+\) telle que
    pour tout \(x, y \in E\) et \(\lambda \in \R\),
    \begin{enumerate}[label=(\roman*)] % chktex 36
        \item la norme possède la \emph{séparation}:
        \begin{equation*}
            \nn{x} = 0 \iff x = 0
        \end{equation*}

        \item la norme est \emph{homogène}:
        \begin{equation*}
            \nn{\lambda x} = \abs{\lambda} \nn{x}
        \end{equation*}

        \item la norme vérifie l'\emph{inégalité triangulaire}:
        \begin{equation*}
            \nn{x + y} \leq \nn{x} + \nn{y}
        \end{equation*}
    \end{enumerate}
\end{definition}

\begin{exercice}[\(\star\)]\,\\
    Montrer que si \(\nn{\cdot}\) est une norme sur \(E\), alors la fonction
    \begin{equation*}
        d(x, y) = \nn{x - y}
    \end{equation*}
    est une distance sur \(E\).
\end{exercice}

\begin{example}
    Un exemple classique est \(\bb R^n\) muni d'une norme \(\nn{\cdot}\).
\end{example}

\begin{exercice}[\(\star\)]\,\\
    Soit \(X\) et \(\delta\from X \times X \to \R\) telle que
    \begin{equation*}
        \delta(x, y) = \begin{cases}
            0 & \text{si } x = y \\
            1 & \text{sinon}
        \end{cases}
    \end{equation*}
    Montrer que \(\delta\) est une distance sur \(X\) appelée \defemph{distance discrète}.
\end{exercice}

\begin{remark}
    Si on considère \(\bb R\) muni de \(\delta\) alors \(\delta\) n'est pas une norme.
\end{remark}

\section{Ouverts d'un espace métrique}\label{sec:ouverts-dun-espace-metrique}

Soit \((X, d)\) un espace métrique.

\begin{definition}
    Pour \(\varepsilon > 0\) et \(x_0\in X\), on note
    \begin{equation*}
        \bolo{x_0}{\varepsilon} = \{x \in X \mid d(x, x_0) < \varepsilon\}
    \end{equation*}
    la \defemph{boule ouverte} de centre \(x_0\) et de rayon \(\varepsilon\).
\end{definition}

\begin{definition}
    Une partie \(U \subset X\) est dite \defemph{ouverte} si et seulement si
    pour tout \(x \in U\), il existe \(\varepsilon > 0\) tel que \(\bolo{x}{\varepsilon} \subset U\).
\end{definition}

\begin{example}\,
    \begin{itemize}
        \item Dans \(\R\) muni de la norme euclidienne, on a
        \begin{equation*}
            \bolo{x_0}{\varepsilon} = \{ x \in \R \mid \abs{x - x_0} < \varepsilon \}
        \end{equation*}
        qui est l'intervalle ouvert \(\oo{x_0 - \varepsilon, x_0 + \varepsilon}\).

        \item Un contre-exemple est l'intervalle \(\fo{0, 1}\) dans \(\R\) qui n'est pas ouvert.
    \end{itemize}
\end{example}

\begin{definition}
    On note \defemph{\(\Tau_d = \{\text{ouverts de }X\}\)}
\end{definition}

\begin{proposition}
    On a les propriétés suivantes:
    \begin{enumerate}[label=(\roman*)] % chktex 36
        \item \(X \in \Tau_d\) et \(\varnothing \in \Tau_d\),
        \item Si \({\{U_i\}}_{i \in I}\) est une famille de \(\Tau_d\), alors \(\bigcup_{i \in I} U_i \in \Tau_d\),
        \item Si  \({\{U_i\}}_{i \in \{1, \ldots, n\}}\) est une famille finie de \(\Tau_d\), alors \(\bigcap_{i = 1}^n U_i \in \Tau_d\).
    \end{enumerate}
\end{proposition}

\begin{proof}\,
    \begin{enumerate}[label=(\roman*)] % chktex 36
        \item Par convention de logique, on a \(\varnothing\in \Tau_d\).
        Soit \(x \in X\), alors \(\bolo{x}{1} \sub X\), donc \(X \in \Tau_d\).

        \item Soit \(x \in \bigcup_{i \in I} U_i\), alors il existe \(i \in I\) tel que \(x \in U_i\).
        Comme \(U_i \in \Tau_d\), il existe \(\varepsilon > 0\) tel que \(\bolo{x}{\varepsilon} \sub U_i \sub \bigcup_{i \in I} U_i\).
        Donc \(\bigcup_{i \in I} U_i \in \Tau_d\).

        \item Soit \(x \in \bigcap_{i = 1}^n U_i\), alors pour tout \(i \in \{1, \ldots, n\}\), on a \(x \in U_i\).
        Comme \(U_i \in \Tau_d\), il existe \(\varepsilon_i > 0\) tel que \(\bolo{x}{\varepsilon_i} \sub U_i\).
        Posons \(\varepsilon = \min_{i = 1}^n \varepsilon_i\), alors pour tout \(i \in \{1, \ldots, n\}\), on a \(\bolo{x}{\varepsilon} \sub U_i\).
        Donc \(\bigcap_{i = 1}^n U_i \in \Tau_d\).
    \end{enumerate}
\end{proof}

\begin{definition}
    Soit \(X\) un ensemble (pas forcément métrique). On dit que
    \(\Tau \sub\scr P(x)\) est une \defemph{topologie} sur \(X\) si
    elle vérifie les propriétés suivantes:
    \begin{enumerate}[label=(\roman*)] % chktex 36
        \item \(X \in \Tau\) et \(\varnothing \in \Tau\),
        \item Si \({\{U_i\}}_{i \in I}\) est une famille de \(\Tau\), alors \(\bigcup_{i \in I} U_i \in \Tau\),
        \item Si  \({\{U_i\}}_{i \in \{1, \ldots, n\}}\) est une famille finie de \(\Tau\), alors \(\bigcap_{i = 1}^n U_i \in \Tau\).
    \end{enumerate}

    Les éléments de \(\Tau\) sont appelés \defemph{ouverts} de \(X\).
    On dit alors que \((X, \Tau)\) est un \defemph{espace topologique}.
\end{definition}

\begin{example}
    Soit \(X\) un ensemble. On a les exemples suivants:
    \begin{enumerate}[label=(\alph*)] % chktex 36
        \item Si \((X, d)\) est un espace métrique, alors \(\Tau_d\) est une topologie sur \(X\).
        \item \(\Tau = \{\varnothing, X\}\) est une topologie sur \(X\).
        \item \(\Tau = \scr P(X) = \Tau_\delta\) est une topologie sur \(X\) où \(\delta\) est la distance discrète.
        \item Si \(X = \{a, b\}\), alors \(\Tau = \{\varnothing, \{a\}, \{a, b\}\}\) est une topologie sur \(X\).
    \end{enumerate}
\end{example}

\begin{definition}
    Soit \((X, \Tau_X)\) et \((Y, \Tau_Y)\) deux espaces topologiques
    et \(f\from X \to Y\) une application.
    On dit que \(f\) est \defemph{continue} si pour tout ouvert \(V \in \Tau_Y\),
    \(f^{-1}(V) \in \Tau_X\).
\end{definition}

\begin{definition}
    Soit \((X, \Tau)\) un espace topologique.
    On dit que \(A \subset X\) est \defemph{fermé} si \(X \setminus A\) est ouvert.
\end{definition}

\begin{remark}
    Un ensemble \(A\sub X\) peut être ouvert et fermé en même temps.
\end{remark}

\begin{example}
    Si on se place dans \(\R\) muni de la norme euclidienne, alors
    l'intervalle \(\fo{0, 1}\) n'est ni ouvert ni fermé.
\end{example}

\begin{proposition}[de relation avec la continuité de \(\R\) dans \(\R\)]\,\\
    Soient \((X, d_X)\) et \((Y, d_Y)\) deux espaces métriques.

    Une application \(f\from X \to Y\) est continue si et seulement si 
    pour tout \(x_0 \in X\) et pour tout \(\varepsilon > 0\),
    il existe \(\delta > 0\) tel que
    \begin{equation*}
        \forall x \in X, \quad d_X(x, x_0) < \delta \implies d_Y(f(x), f(x_0)) < \varepsilon
    \end{equation*}
\end{proposition}

\begin{proof}
    On commence par énoncer et démontrer un lemme qui nous sera utile:
    \begin{lemma}
        Soit \((X, d)\) un espace métrique. 
        Une boule ouverte sur \(X\) est un ouvert
        pour la topologie \(\Tau_{d}\).
    \end{lemma}
    \begin{proof}
        Soit \(x_0 \in X\) et \(\varepsilon > 0\).
        On a \(\bolo{x_0}{\varepsilon} \in \Tau_d\) par définition de la topologie.

        Soit \(x \in \bolo{x_0}{\varepsilon}\), alors \(d(x, x_0) < \varepsilon\).
        Posons \(\delta = \varepsilon - d(x, x_0)\), alors \(\delta > 0\).
        
        Soit \(y \in \bolo{x}{\delta}\), alors \(d(y, x) < \delta\).
        Par l'inégalité triangulaire, on a
        \begin{align*}
            d(y, x_0) &\leq d(y, x) + d(x, x_0) \\
            &< \delta + d(x, x_0) \\
            &= \varepsilon
        \end{align*}
        Donc \(y \in \bolo{x_0}{\varepsilon}\), donc \(\bolo{x}{\delta} \subset \bolo{x_0}{\varepsilon}\).
    \end{proof}

    \centerline{\rule{13cm}{0.4pt}}
    \bigskip

    Revenons à la preuve de la proposition.

    \ptr{} Sens direct:

    Soit \(x_0 \in X\) et \(\varepsilon > 0\).
    Montrons que \(B = \bolo{f(x_0)}{\varepsilon}\) est un ouvert
    de \(Y\), donc \(f^{-1}(B)\) est un ouvert de \(X\).

    On sait que \(x_0 \in f^{-1}(B)\), ouvert par hypothèse.
    Alors, il existe \(\delta > 0\) tel que
    \begin{equation*}
        \bolo{x_0}{\delta} \subset f^{-1}(B)
    \end{equation*}
    Donc pour tout \(x \in X\), 
    \begin{equation*}
        d_X(x, x_0) < \delta \implies d_Y(f(x), f(x_0)) < \varepsilon
    \end{equation*}

    \ptr{} Sens réciproque:

    Soit \(V \in \Tau_Y\), alors \(V\) est un ouvert de \(Y\).
    Soit \(x_0 \in f^{-1}(V)\), alors \(f(x_0) \in V\).
    Comme \(V\) est ouvert, il existe \(\varepsilon > 0\) tel que
    \(\bolo{f(x_0)}{\varepsilon} \subset V\).

    Par hypothèse, il existe \(\delta > 0\) tel que
    \begin{equation*}
        \forall x \in X, \quad d_X(x, x_0) < \delta \implies d_Y(f(x), f(x_0)) < \varepsilon
    \end{equation*}
    Donc \(\bolo{x_0}{\delta} \subset f^{-1}(V)\), donc \(f^{-1}(V)\) est ouvert.
\end{proof}

\begin{remark}
    Si \(f\) est une fonction bijective et continue,
    sa réciproque \(f^{-1}\) n'est pas forcément continue.
\end{remark}

\begin{definition}
    On dit que \(f\) est un \defemph{homéomorphisme}
    si \(f\) est bijective, continue et que sa réciproque est continue.
\end{definition}

\begin{remark}
    Une fonction continue et bijective n'est pas forcément un homéomorphisme.
\end{remark}

\begin{definition}
    Soient \((X, d_X)\) et \((Y, d_Y)\) deux espaces métriques.
    Soit \(f\from X \to Y\) une application bijective.

    On dit que \(f\) est une \defemph{isométrie} si
    \begin{equation*}
        \forall x, y \in X, \quad d_X(x, y) = d_Y(f(x), f(y))
    \end{equation*}
\end{definition}

\begin{exercice}
    Montrer les propriétés suivantes:
    \begin{enumerate}[label=(\roman*)] % chktex 36
        \item Si \(f\) est une isométrie, alors \(f\) est un homéomorphisme.

        \item Si \(f\) (non bijective) a la propriété suivante:
        \begin{equation*}
            \forall x, y \in X, \quad d_X(x, y) = d_Y(f(x), f(y))
        \end{equation*}
        alors \(f\) est injective.
    \end{enumerate}
\end{exercice}