\section{Espaces métriques}\label{sec:espaces-metriques}

Soit \(X\) un ensemble.

\begin{definition}
    On appellle une \defemph{distance} (ou métrique) sur \(X\)
    une application \(d\ \colon\ X \times X \to \R\) telle que
    pour tout \(x, y, z \in X\),
    \begin{enumerate}[label=(\roman*)]
        \item la distance est \emph{positive}:
        \begin{equation*}
            d(x, y) \geq 0
        \end{equation*}

        \item la distance possède la \emph{séparation}:
        \begin{equation*}
            d(x, y) = 0 \iff x = y
        \end{equation*}

        \item la distance est \emph{symétrique}:
        \begin{equation*}
            d(x, y) = d(y, x)
        \end{equation*}

        \item la distance vérifie l'\emph{inégalité triangulaire}:
        \begin{equation*}
            d(x, z) \leq d(x, y) + d(y, z)
        \end{equation*}
    \end{enumerate}
\end{definition}

\begin{example}
    Un exemple classique de distance est la \defemph{distance euclidienne} sur \(\R^n\):
    \begin{equation*}
        d(x, y) = \sqrt{\sum_{i=1}^n {(x_i - y_i)}^2}
    \end{equation*}
\end{example}

\begin{definition}
    Soit \(E\) un \(\bb R\)-espace vectoriel.
    On appelle \defemph{norme} sur \(E\) une application \(\nn{\cdot}\ \colon\ E \to \R^+\) telle que
    pour tout \(x, y \in E\) et \(\lambda \in \R\),
    \begin{enumerate}[label=(\roman*)]
        \item la norme possède la \emph{séparation}:
        \begin{equation*}
            \nn{x} = 0 \iff x = 0
        \end{equation*}

        \item la norme est \emph{homogène}:
        \begin{equation*}
            \nn{\lambda x} = \abs{\lambda} \nn{x}
        \end{equation*}

        \item la norme vérifie l'\emph{inégalité triangulaire}:
        \begin{equation*}
            \nn{x + y} \leq \nn{x} + \nn{y}
        \end{equation*}
    \end{enumerate}
\end{definition}

\begin{exercice}[\(\star\)]\,\\
    Montrer que si \(\nn{\cdot}\) est une norme sur \(E\), alors la fonction
    \begin{equation*}
        d(x, y) = \nn{x - y}
    \end{equation*}
    est une distance sur \(E\).
\end{exercice}

\begin{example}
    Un exemple classique est \(\bb R^n\) muni d'une norme \(\nn{\cdot}\).
\end{example}

\begin{exercice}[\(\star\)]\,\\
    Soit \(X\) et \(\delta\ \colon\ X \times X \to \R\) telle que
    \begin{equation*}
        \delta(x, y) = \begin{cases}
            0 & \text{si } x = y \\
            1 & \text{sinon}
        \end{cases}
    \end{equation*}
    Montrer que \(\delta\) est une distance sur \(X\) appelée \defemph{distance discrète}.
\end{exercice}

\begin{remark}
    Si on considère \(\bb R\) muni de \(\delta\) alors \(\delta\) n'est pas une norme.
\end{remark}

\section{Ouverts d'un espace métrique}\label{sec:ouverts-dun-espace-metrique}

Soit \((X, d)\) un espace métrique.

\begin{definition}
    Pour \(\varepsilon > 0\) et \(x_0\in X\), on note
    \begin{equation*}
        \bolo{x_0}{\varepsilon} = \{x \in X \mid d(x, x_0) < \varepsilon\}
    \end{equation*}
    la \defemph{boule ouverte} de centre \(x_0\) et de rayon \(\varepsilon\).
\end{definition}

\begin{definition}
    Une partie \(U \subset X\) est dite \defemph{ouverte} si pour tout \(x \in U\), il existe \(\varepsilon > 0\) tel que \(\bolo{x}{\varepsilon} \subset U\).
\end{definition}

\begin{example}\,
    \begin{itemize}
        \item Dans \(\R\) muni de la norme euclidienne, on a
        \begin{equation*}
            \bolo{x_0}{\varepsilon} = \{ x \in \R \mid \abs{x - x_0} < \varepsilon \}
        \end{equation*}
        qui est l'intervalle ouvert \(\oo{x_0 - \varepsilon, x_0 + \varepsilon}\).

        \item Un contre-exemple est l'intervalle \(\fo{0, 1}\) dans \(\R\) qui n'est pas ouvert.
    \end{itemize}
\end{example}

