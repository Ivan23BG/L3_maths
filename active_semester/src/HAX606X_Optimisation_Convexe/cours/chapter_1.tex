\begin{notation}\,
    \begin{itemize}
        \item On considère un espace vectoriel normé de dimension \(n\)
        noté \((E, \nn{\cdot})\) et \(U\) ouvert de \(E\).
    
        \item On considère une fonction \(f\from U\to\R\). Dans la pratique,
        \(E\) sera égal à \(\R^n\).
    
        \item Soit \(x\in U\), on note \(f'(x)\) la différentielle
        (qu'on appelera plus simplement \og{} dérivée \fg{})
        de \(f\) en \(x\). On a donc, pour tout \(h\in E\) tel que
        \(\nn{h}\) soit assez petit, 
        \begin{equation*}
            f(x+h) = f(x) + f'(x)\cdot h + \nn{h} \varepsilon(x,h)
        \end{equation*}
        avec \(\varepsilon(x,h) \underset{h\to 0}{\to} 0\)
        et \(f'(x)\in \scr L(E, \R)\), c'est à dire que \(f'(x)\) est une forme linéaire.
    
        \item Avec cette notation, si \(f\) est dérivable en \(x\),
        alors \(f\) admet des dérivées partielles en \(x\) dans toutes les
        directions, et si \((e_1,\ldots,e_n)\) est une base de \(E\),
        on note \og{}\(\partial_i f(x)\)\fg{} ou encore \og{}\(\dpar{f}{x_i}(x)\)\fg{}
        la dérivée partielle de \(f\) par rapport à la \(i\)-ième variable.
        On a alors
        \begin{equation*}
            \partial_i f(x) = f'(x)\cdot e_i,\quad i=1,\ldots,n
        \end{equation*}
        Ainsi, pour \(h\in E\) tel que \(h=\sum_{i=1}^n h_i e_i\), on a
        \begin{equation*}
            \begin{aligned}
                f'(x)\cdot h &= f'(x) \cdot \left(\sum_{i=1}^n h_i e_i\right)\\
                &= \sum_{i=1}^n h_i f'(x)\cdot e_i = \sum_{i=1}^n h_i\partial_i f(x)
            \end{aligned}
        \end{equation*}
        De même, si \(x\mapsto f'(x)\) est dérivable en \(x\), on note
        \(f''(x)\in\scr L(E, \scr L(E,\R))\) cette dérivée seconde, et on
        considère \(f''(x)\) comme une forme bilinéaire \((f''(x)\in\scr L(E\times E, \R))\)
    \end{itemize}
\end{notation}

Avec ces notations, le théorème fondamental de l'analyse (TFA) peut s'énoncer ainsi:

\begin{thm}[Fondamental de l'Analyse]
    Soit \(f\in \scr C^1(U, \R)\). Alors pour tout \((x,y)\in U\) tel que
    \begin{equation*}
        \forall t\in \ff{0,1}, x +t(y-x)\in U,
    \end{equation*}
    on a
    \begin{equation*}
        f(y) = f(x) + \int_0^1 f'(x+ t(y-x))\cdot(y-x)\der t.
    \end{equation*}
\end{thm}

\begin{fmdt}
    Soit \(f\in \scr C^2(U, \R)\) et \(x\in U\). Alors, il existe un voisinage
    \(\nu\) de \(x\) tel que, pour tout \(x+h\in\nu\), on a
    \begin{equation*}
        f(x+h) = f(x) + f'(x)\cdot h + \frac12 f''(x)\cdot(h,h) + \po(\nn{h}^2)
    \end{equation*}

    Bien entendu, cette expression peut aussi se formuler ainsi:
    \begin{equation*}
        \begin{aligned}
            f(x+h) = f(x) + \sum_{i=1}^n \partial_i f(x) h_i + \frac12 \sum_{i,j=1}^n \partial^2_{ij} f(x) h_i h_j + \nn{h}^2 \varepsilon(h)
        \end{aligned}
    \end{equation*}
    avec \(\varepsilon(h) \underset{\nn{h}\to 0}{\to} 0\), 
    de manière a bien mettre en évidence la linéarité de la dérivée et
    la bilinéarité de la dérivée seconde.

    Si on utilise la notation \(\nabla f(x)\) pour le gradient de \(f\) en \(x\),
    et \(\nabla^2 f(x)\) la matrice Hessienne de \(f\) évaluée en \(x\), on 
    a alors
    \begin{equation*}
        f(x+h) = f(x) + \nabla f(x)h + \frac 12 {}^T h\nabla^2 f(x)\cdot h + \po(\nn{h}^2)
    \end{equation*}

\end{fmdt}

%%%%%%%%%%%%%%

\section{Résultats d'existence}

Unn outil fondamental: la compacité.

\begin{theorem}
    Soit \(K\) un compact de \(\R^n\) et \(f\from K\to\R\)
    une fonction continue. Alors \(f\) est bornée et atteint ses
    bornes:
    \begin{equation*}
        \sup_{x\in K} \n{f(x)} < +\infty
    \end{equation*}
    et il existe \(\ul{x}\in K\) et \(\ol{x} \in K\) tels que
    \begin{equation*}
        \begin{aligned}
            &f(\ul x) = \inf_{x\in K} f(x) = \min_{x\in K} f(x)\\
            &f(\ol x) = \sup_{x\in K} f(x) = \max_{x\in K} f(x)
        \end{aligned}
    \end{equation*}
\end{theorem}

\begin{proof}
    Ce résultat a été démontré dans le cours de topologie / analyse
    fonctionnelle. Puisque \(f\) est continue, \(f(K)\) est une partie compacte de \(\R\),
    c'est a dire une partie fermée et bornée de \(\R\).

    Ainsi, on a \(-\infty < \inf f(K) \leq \sup f(K) < +\infty\)
    et puisque \(f(K)\) est fermée et que \(\inf f(K)\) et \(\sup f(K)\)
    sont adhérents, on a inf et min dans \(f(E)\) %%% completer preuve
\end{proof}

\begin{definition}
    Soit \(f\from \R^n\to\R\). On dit que \(f\) est \defemph{coercive}
    si \(f(x) \to +\infty\) lorsque \(\nn x\to +\infty\).
\end{definition}

\begin{theorem}
    Soit \(f\from \R^n\to\R\) continue et coercive. Alors \(f\) est minorée
    et atteint son minimum.
\end{theorem}

\begin{proof}
    Posons \(A = f(0) + 1\). Comme \(f\) est coercive, il existe \(\alpha > 0\) tel que
    \begin{equation*}
        \forall x\in\R^n, \nn x\geq \alpha \implies f(x) \geq  f(0) + 1
    \end{equation*}
    La boule \(\ol B(0, \alpha)\) est un fermé borné de \(\R^n\) %%% remplacer boule
    donc un compact de \(\R^n\) et \(f\) sur cette boule est continue %%% faire restriction symbole
    D'après le theoreme 1.1.1, \(f\) est minorée sur \(\ol B(0, \alpha)\) et atteint son minimum en un certain  %%% lien theoreme
    \(x_0\)

    Ainsi, soit \(x\in\R^n\). On a plusieurs cas
    \begin{enumerate}[label=(\alph*)]
        \item si \(x\in \ol B(0, \alpha)\), alors \(f(x)\geq f(x_0)\),

        \item si \(x\notin \ol B(0, \alpha)\), alors \(\nn x > \alpha\) et 
        donc \(f(x)\geq f(0) + 1\) et \(f(x)\geq f(x_0) + 1 \geq f(x_0)\)
        puisque \(0\in \ol B(0, \alpha)\).

        Ainsi, \(\forall x\in\R^n, f(x)\geq f(x_0)\) et \(x_0\)
        est bien le minimum de \(f\) sur \(\R^n\).
    \end{enumerate}
    %%% completer preuve
\end{proof}

\begin{remark}
    Ce dernier resultat peut eter généralisé sous les meme hypothèses

    %%% completer rem
    %%% envoyer email contacter poly
\end{remark}

\section{Caractérisation des extrema sans contraintes}

Un outil fondamental: le calcul différentiel.

\begin{theorem}
    Soit \(U\sub \R^n\) ouvert et \(f\from U\to \R\) de classe \(\scr C^1\).
    Si \(x_0\in U\) est extremum local de \(f\) sur \(U\) alors on a 
    \(f'(x_0) = O\) (ou \(\nabla f(x_0)=0\))
\end{theorem}

\begin{proof}
    %%% separer en cas simple/observateur contre general?
    Rappelons ce qu'il se passe pour une fonction \(\varphi\from I\sub \R\to\R\) qui
    admet, par exemple, un maximum local en \(0\in I\).

    On a d'une part \(\varphi'(0) = \lim_{x\to0^+}\frac{\varphi(x)-\varphi(0)}{x}\leq 0\)
    car \(x>0\) et \(\varphi(x) - \varphi(0)\geq 0\).
    D'autre part, on a l'inverse pour la limite a gauche %%% a remplacer
    Dans le cas \(E = \R^n\), supposons que \(f\) admet un maximum local en \(x_0\in U\).
    Soit \(e_i\) un vecteur de base

    On sait que \(\partial_i f(x_0) = f'(x_0)\cdot e_i\) %%% a completer

    %%% a completer

    Ainsi toutes les dérivées partielles de \(f\) sont nulles en \(x_0\) et donc
    \(f'(x_0) = 0\).
\end{proof}

\begin{remark}
    Ce résultat est bien entendu faux si \(U\) n'est pas un ouvert:

    \begin{equation*}
        \begin{aligned}
            f\from &\ff{0,1}\to\ff{0,1}\\
            &x\mapsto x
        \end{aligned}
    \end{equation*}
    où 0 est le minimum sur \(\ff{0,1}\) et pourtant \(f'(0)=1\).
\end{remark}

\begin{definition}
    Soit \(f\from U\to \R\). On dit que \(a\in U\) est un point critique de \(f\) si 
    une des conditions suivantes est vérifiée:
    \begin{enumerate}[label=(\roman*)]
        \item \(f'(a)=0\),
        \item \(\nabla f(a) = 0\),
        \item \(\partial_i f(a) = 0\) pour tout \(i=1,\ldots,n\).
    \end{enumerate}
\end{definition}

\begin{theorem}
    Soit \(U\sub\R^n\) ouvert, \(f\from U\to \R\) de classe \(\scr C^2\).

    Si \(x_0\in U\) est minimum local de \(f\) sur \(U\), alors \(f'(x_0) = 0\)
    et \(f''(x_0)\) est positive, soit, au sense des formes bilinéaires symétriques:
    \begin{equation*}
        \forall \xi\in\R^n,f''(x_0)\cdot(\xi, \xi)\geq 0
    \end{equation*}
\end{theorem}

\begin{proof}
    Soit \(x_0\in U\) et \(h\in\R^n\). Pour \(t\in\R\) assez petit, on a
    \begin{equation*}
        f(x_0+th) -f(x_0)\geq 0.
    \end{equation*}
    Or, d'après Taylor, on a
    \begin{equation*}
        f(x_0+th) -f(x_0) = \frac12 f''(x_0)\dot(th, th) + o_{t\to 0}(t^2) %%% petit o
    \end{equation*}
\end{proof}