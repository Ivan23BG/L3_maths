%% Proposition 1.4.2. Optimisation des fonctions elliptiques
% Soit \(f\in C^1(\bb R^n, \bb R)\) et \(\alpha\)-elliptique.
% Alors \(f\) est minorée et atteint son minimum 
% en un unique point \(\ul{x}\) caractérisé par \(f'(\ul{x}) = 0\)


% Preuve
% structure de la preuve:
% 1. f est minoree
% 2. f atteint son minimum
% 3. unicite du minimum
% 4. caractérisation du minimum
Soit \(f\in C^1(\bb R^n, \bb R)\) telle que \(f\) est \(\alpha\)-elliptique.
Alors, on sait que \(f\) est coercive d'après
la proposition~\ref{prop:1.3.7}. De plus, comme \(\bb R^n\) est un
fermé non borné, les hypothèses du théorème~\ref{thm:1.1.3} sont
satisfaites et donc \(f\) est minorée et atteint son minimum en un
point \(\ul{x}\) de \(\bb R^n\).

Montrons maintenant que \(\ul{x}\) est le seul point où \(f\) atteint
son minimum.

% Unicité du minimum
Comme \(f\) est \(\alpha\)-elliptique, elle est aussi strictement
convexe d'après la proposition~\ref{prop:1.3.7} et donc d'après le théorème~\ref{thm:1.4.1}, \(\ul{x}\) est
le seul point où \(f\) atteint son minimum.

% Caractérisation du minimum
Montrons maintenant que \(f'(\ul{x}) = 0\). Soit \(\ul{d}\) un vecteur
quelconque de \(\bb R^n\). On a
\begin{equation*}
    f(\ul{x} + t\ul{d}) = f(\ul{x}) + f'(\ul{x})\ul{d} + o(t)
\end{equation*}
et donc
\begin{equation*}
    f(\ul{x} + t\ul{d}) - f(\ul{x}) = f'(\ul{x})\ul{d} + o(t)
\end{equation*}
Comme \(f\) atteint son minimum en \(\ul{x}\), on a
\begin{equation*}
    f(\ul{x} + t\ul{d}) - f(\ul{x}) \geq 0
\end{equation*}
et donc
\begin{equation*}
    f'(\ul{x})\ul{d} \geq 0
\end{equation*}
Comme \(\ul{d}\) est quelconque, on a \(f'(\ul{x}) = 0\).
