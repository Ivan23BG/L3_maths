\subsection*{Introduction aux méthodes de gradient}\label{ssec:1}

Expérimenter la méthode du gradient à pas fixe pour minimiser la fonction
\begin{equation*}
    f(x) = x^4 - 7x + 8.
\end{equation*}

On testera successivement les pas \(\rho \in \{0.125, 0.1, 0.01\}\)
avec l'initialisation \(x_0 = 1\) et 15 itérations.

On représentera la suite des itérés sur la courve représentative
de la fonction \(f\) pour chaque pas \(\rho\) sur \(\ff{0.5, 1.5}\).

\subsection{Méthodes d'optimisation pour des fonctions quadratiques}\label{ssec:2}

Soit \(n\in\bb N^\ast\). On désignera par \(\langle \cdot,\cdot \rangle\)
le produit scalaire associé à la norme euclidienne sur \(\bb R^n\).

On considère la matrice \(A\in\mathcal{S}_n(\bb R)\) et le vecteur
\(b\in\bb R^n\) définis par:
\begin{equation*}
    A_n = \underbrace{
    \begin{pmatrix}
        4 & -2 & 0 & \cdots & 0 \\
        -2 & 4 & -2 & \cdots & 0 \\
        0 & -2 & 4 & \cdots & 0 \\
        \vdots & \vdots & \vdots & \ddots & \vdots \\
        0 & 0 & 0 & \cdots & 4
    \end{pmatrix}}_{\text{\(n\times n\)}},
    \qquad
    b_n = \underbrace{
    \begin{pmatrix}
        1 \\ 1 \\ \vdots \\ 1
    \end{pmatrix}}_{\text{\(n\times 1\)}}.
\end{equation*}

On cherche à minimiser à l'aide de méthodes de gradient
la fonction:
\begin{equation*}
    \begin{aligned}
        J_n\from \bb R^n &\to \bb R \\
        x &\mapsto \frac12 \langle A_n x,x \rangle - \langle b_n,x \rangle.
    \end{aligned}
\end{equation*}

On se concentrera sur le cas \(n = 2\).

\begin{td-exo}\,
    \begin{enumerate}
        \item Calculer \(\nabla J_n(x,y)\) puis \(\nabla^2 J_n(x,y)\).

        \item Cette fonction est-elle convexe? En quels
        points atteint-elle son minimum?

        \item Visualiser la fonction en 3D sur \(\ff{-2,2}^2\) puis
        visualiser les lignes de niveau sur \(\ff{-1.5,1.5}^2\).
        Qu'observe-t-on?
    \end{enumerate}
\end{td-exo}


\iftoggle{showsolutions}{
    \begin{td-sol}\,
        % a remplir
    \end{td-sol}
}{}

\subsection{Méthode de gradient à pas constant}\label{ssec:3}

\begin{td-exo}\,
    \begin{enumerate}
        \item Implémenter la méthode de gradient à pas constant et 
        l'appliquer à la fonction \(J_n\) en partant du point
        \((-1,1)\) avec un critère d'arrêt \(\varepsilon = 10^{-6}\).

        \item Afficher la trajectoire des points calculés successivement.
        Comment faut-il régler le pas pour arriver vraiment au minimum?
    \end{enumerate}
\end{td-exo}

\iftoggle{showsolutions}{
    \begin{td-sol}\,
        % a remplir
    \end{td-sol}
}{}

\subsection{Méthode de gradient à pas optimal}\label{ssec:4}

\begin{td-exo}\,
    \begin{enumerate}
        \item Implémenter la méthode de gradient à pas optimal et 
        l'appliquer à la fonction \(J_n\) en partant du point
        \((-1,1)\) avec un critère d'arrêt \(\varepsilon = 10^{-6}\).

        \item Afficher la trajectoire des points calculés successivement.
    \end{enumerate}
\end{td-exo}

\iftoggle{showsolutions}{
    \begin{td-sol}\,
        % a remplir
    \end{td-sol}
}{}

\subsection{Méthode du gradient conjugué}\label{ssec:5}

\begin{td-exo}\,
    \begin{enumerate}
        \item Implémenter la méthode du gradient conjugué et 
        l'appliquer à la fonction \(J_n\) en partant du point
        \((-1,1)\) avec un critère d'arrêt \(\varepsilon = 10^{-6}\).

        \item Afficher la trajectoire des points calculés successivement.
    \end{enumerate}
\end{td-exo}

\iftoggle{showsolutions}{
    \begin{td-sol}\,
        % a remplir
    \end{td-sol}
}{}

\subsection{Conclusion}\label{ssec:6}

\begin{td-exo}\,
    \begin{enumerate}
        \item Comparer les performances des trois méthodes pour \(n = 2\)
        en terme de nombre d'itérations et de temps de calcul.
    \end{enumerate}
\end{td-exo}