\subsection{Méthode de la dichotomie}\label{exo:1}

\begin{td-exo}\,
    \begin{enumerate}
        \item Quelle équation souhaite-t-on résoudre pour notre problème d'optimisation?
        Quelles conditions doit-on vérifier sur \(f\) pour appliquer la méthode de la dichotomie?
    
        \item Ecrire l'algorithme de dichotomie et l'appliquer pour trouver le minimum de la fonction
        \(f = x^2 - 2\sin(x)\) sur \(\ff{0,2}\) avec une précision de \(10^{-5}\).
        Comment obtient-on le nombre d'itérations à partir de la précision?
    
        \item Comparer votre code avec l'implémentation de la fonction \texttt{scipy.optimize.bisect}.
    \end{enumerate}
\end{td-exo}

\iftoggle{showsolutions}{
    \begin{td-sol}\,
        \begin{enumerate}
            \item On souhaite résoudre l'équation \(f(x) = 0\) pour notre problème d'optimisation.
            Il faut donc que \(f\) soit continue sur \(\ff{a,b}\) et que \(f(a)f(b) < 0\).

            On va alors chercher à résoudre \(f'(x) = 0\) pour trouver les points critiques de \(f\).

            \item L'algorithme de dichotomie est le suivant:
            % insert image from assets/code.png
            REMPLACER CODE ICI
            % \begin{figure}[H]
            %     \centering
            %     \includegraphics[width=0.8\textwidth]{assets/code.png}
            % \end{figure}
            Après exécution, on trouve que le minimum de \(f\) est \(x = 0.73908\) avec une précision de \(10^{-5}\).

            Pour obtenir le nombre d'itérations à partir de la précision, on utilise la formule
            \begin{equation*}
                n = \frac{\log\p{\frac{b-a}{\varepsilon}}}{\log(2)},
            \end{equation*}
            où \(n\) est le nombre d'itérations, \(a\) et \(b\) sont les bornes de l'intervalle et \(\varepsilon\) est la précision.
            
            L'implémentation est la suivante:
            % insert image from assets/code.png
            REMPLACER CODE ICI
            % \begin{figure}[H]
            %     \centering
            %     \includegraphics[width=0.8\textwidth]{assets/code.png}
            % \end{figure}
            et indique qu'il faut 18 itérations pour obtenir le résultat.

            \item En comparant avec la méthode implémentée dans \texttt{scipy.optimize.bisect}, on trouve que le résultat est le même
            et que le nombre d'itérations est identique.
        \end{enumerate}
    \end{td-sol}
}{}

\subsection{Méthode de Newton}\label{exo:2}

\begin{td-exo}\,
    \begin{enumerate}
        \item Quelle condition doit vérifier \(f\) pour appliquer la méthode de Newton
        pour le problème d'optimisation? Comment va être formulé l'itéré de Newton dans ce cas?

        \item Ecrire l'algorithme de Newton dans ce cas et l'appliquer à la fonction 
        \(f(x) = x^2 - 2\sin(x)\) avec \(x_0 = 1\).
    \end{enumerate}

    Dans les deux cas précédents, il nous faut de la régularité pour la fonction \(f\), voici 
    une autre méthode qui demande moins de régularité pour notre fonction.
\end{td-exo}

\iftoggle{showsolutions}{
    \begin{td-sol}\,
        \begin{enumerate}
            \item Pour appliquer la méthode de Newton, il faut que \(f\) soit de classe \(\mathcal{C}^2\) sur \(\ff{a,b}\).
            L'itéré de Newton est alors donné par
            \begin{equation*}
                x_{k+1} = x_k - \frac{f'(x_k)}{f''(x_k)}.
            \end{equation*}

            \item L'algorithme de Newton est le suivant:
            % insert image from assets/code.png
            REMPLACER CODE ICI
            % \begin{figure}[H]
            %     \centering
            %     \includegraphics[width=0.8\textwidth]{assets/code.png}
            % \end{figure}
            Après exécution, on trouve que le minimum de \(f\) est \(x = 0.73908\).
        \end{enumerate}
    \end{td-sol}
}{}

\subsection{Méthode de la section dorée}\label{exo:3}

\begin{td-exo}\,
    La méthode de la section dorée permet de trouver le minimum d'une fonction \(f\) continue
    et unimodale sur l'intervalle \(\ff{a,b}\sub\bb R\). On note par la suite le
    nombre d'or \(\phi = \frac{1+\sqrt{5}}{2}\).

    L'algorithme est le suivant:
    % insert bad algorithm here

    \begin{enumerate}
        \item Ecrire l'algorithme et l'appliquer à la fonction \(f(x) = x^2 - 2\sin(x)\) sur \(\ff{0,2}\)

        \item Comparer votre code avec l'implémentation de la fonction \texttt{scipy.optimize.golden}.

        \item Comparer les 3 méthodes pour \(f = -\frac1x +\cos(x)\) sur \(\ff{a,b} = \ff{2,4}\)
        ou pour \(x_0 = 2.5\) au niveau du nombre d'itérations et du temps de calcul.

        Représenter le graphique de la fonction en plaçant les résultats des itérations successives
        de Newton.
    \end{enumerate}
\end{td-exo}

\iftoggle{showsolutions}{
    \begin{td-sol}\,
        \begin{enumerate}
            \item L'algorithme de la section dorée est le suivant:
            % insert image from assets/code.png
            REMPLACER CODE ICI
            % \begin{figure}[H]
            %     \centering
            %     \includegraphics[width=0.8\textwidth]{assets/code.png}
            % \end{figure}
            Après exécution, on trouve que le minimum de \(f\) est \(x = 0.73908\).

            \item L'implémentation de \texttt{scipy.optimize.golden} donne le même résultat
            et le nombre d'itérations est identique.

            \item Pour la fonction \(f = -\frac1x +\cos(x)\) sur \(\ff{2,4}\) avec \(x_0 = 2.5\), on trouve
            que la méthode de la dichotomie est la plus rapide, suivie de la méthode de Newton et enfin
            la méthode de la section dorée.

            Le graphique de la fonction est le suivant:
            % insert image from assets/graph.png
            REMPLACER CODE ICI
            % \begin{figure}[H]
            %     \centering
            %     \includegraphics[width=0.8\textwidth]{assets/graph.png}
            % \end{figure}
        \end{enumerate}
    \end{td-sol}
}{}